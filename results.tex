% -*- root: article.tex -*-
% \lettrine[nindent=0em,lines=3]{L}orem ipsim ult assh
\subsection{Interpreta\c{c}\~ao dos Dados}

As vezes o que era indicado como boa pr\'atica para um elemento era um smell percebido em outro elemento, por exemplo, a R1 diz que \textit{"Sempre que noto ter mais de um [layout] resource usando o mesmo estilo eu tento mov\^e-lo [o estilo] para meu recurso de estilo"} (tradu\c{c}\~ao livre) ao responder sobre boa pr\'atica para o style resource, por\'em esta resposta foi considerada para definir o smell Duplicated Styles Attributes que \'e percebido em recursos de layout ou styles.


\subsection{Afirma\c{c}\~ao sobre o \textit{front-end} Android}

Uma opini\~ao que foi un\^anime em muitas respostas foi que de fato, desenvolvedores tratam \textsc{Activitys}, \textsc{Fragments} e \textsc{Adapters} como elementos do front-end Android, conforme constatamos na se\c{c}\~ao 3.1. Isso pode ser observado diversas vezes com respostas por exemplo, P25 indicou como boa pr\'atica na Activity "Nenhuma l\'ogica [de neg\'ocio] aqui" (tradu\c{c}\~ao livre), o P40 afirma sobre m\'a pr\'atica em adapter \'e \textit{"L\'ogica de neg\'ocio em adapters \'e n\~ao-n\~ao"} (tradu\c{c}\~ao livre), ao falarem sobre fragments, muitos indicaram \textit{"O mesmo da Activity"}. Ou seja, primeiramente estas respostas refor\c{c}am nossa defini\c{c}\~ao inicial sobre elementos que compo\^em o front-end Android, e por outro lado, vimos que muitas vezes fragments s\~ao tratados como Activitys, ao se falar de boas e m\'as pr\'aticas de c\'odigo.

