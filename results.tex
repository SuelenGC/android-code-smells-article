% -*- root: article.tex -*-
Durante o processo de codifica\c{c}\~ao das 18 perguntas sobre boas e m\'as pr\'aticas, 56 categorias emergiram. Agrupamos essas categorias em 6 temas: Conceitos Tradicionais, Problemas Arquiteturais, Problemas Pontuais, Ferramentas Utilit\'arias, Conhecimento e Estrutura de Arquivos. 

Classificamos as categorias de acordo com sua recorr\^encia, ou seja, a quantidade de respostas que a recebeu. Baix\'issima recorr\^encia significa menos de 3 respostas, baixa recorr\^encia significa de 3 a 7 respostas, m\'edia recorr\^encia significa de 8 a 20 respostas e alta recorr\^encia significa acima de 20 respostas.

A Tabela \ref{tab:Categories} apresenta o n\'umero de recorr\^encias, em cada quest\~ao sobre boas ou m\'as pr\'aticas, das categorias de alta, m\'edia e baixa recorr\^encia agrupadas por temas. A \'ultima linha da Tabela, \#Categorias, apresenta quantas categorias emergiram de cada quest\~ao, ou seja, \textbf{com base em um elemento Android}, quais s\~ao os pontos de aten\c{c}\~ao (boas e m\'as pr\'aticas) que devemos avaliar pensando em qualidade de c\'odigo. J\'a a \'ultima coluna da Tabela, \#Quest\~oes, apresenta em quantas quest\~oes cada categoria surgiu, ou seja, \textbf{com base na categoria} (que agrupa observa\c{c}\~oes sobre boas e m\'as pr\'aticas), quais elementos devem ser investigados pensando em qualdiade de c\'odigo. 

% Quando se criam categoriza\c{c}\~oes para auxiliar desenvolvedores a identificar pontos de aten\c{c}\~ao a serem avaliados e onde este pontos devem ser investigados especificamente no c\'odigo, pensando em qualidade de c\'odigo d\'a-se o nome de \textit{smells}. Portanto, nesta se\c{c}\~ao iremos compilar este conjunto de Android \textit{code smells} que identificamos atrav\'es das categoriza\c{c}\~oes.

% A partir das categorias foi poss\'ivel criar defini\c{c}\~oes textuais de alguns code smells Android pois as respostas indicavam os sintomas e poss\'iveis solu\c{c}\~oes, entretando, outras categorias n\~ao foi possivel extrair um smells, pois apareceram nas respostas mais como uma discuss\~ao.

\begin{table*}[t]
\centering
\caption{Lista de categorias de alta, m\'edia e baixa recorr\^encia vs. ocorr\^encia nas quest\~oes sobre boas e m\'as pr\'aticas.}
\footnotesize
\begin{tabular}{@{}p{4cm}|p{.2cm}p{.2cm}p{.2cm}p{.2cm}p{.2cm}p{.2cm}p{.2cm}p{.2cm}p{.2cm}p{.4cm}p{.4cm}p{.4cm}p{.4cm}p{.4cm}p{.4cm}p{.4cm}p{.4cm}p{.4cm}|p{1.5cm}@{}}
\toprule
\textbf{Categoria} & Q1 & Q2 & Q3 & Q4 & Q5 & Q6 & Q7 & Q8 & Q9 & Q10 & Q11 & Q12 & Q13 & Q14 & Q15 & Q16 & Q17 & Q18 &  \textbf{\#Quest\~oes} \\
\hline
No Logic In View							& \multicolumn{1}{c}{12}	& \multicolumn{1}{c}{15}	& \multicolumn{1}{c}{6}	& \multicolumn{1}{c}{8}	& \multicolumn{1}{c}{3}	& \multicolumn{1}{c}{8}	& \multicolumn{1}{c}{--}	& \multicolumn{1}{c}{--}	& \multicolumn{1}{c}{--}	& \multicolumn{1}{c}{--}	& \multicolumn{1}{c}{--}	& \multicolumn{1}{c}{--}	& \multicolumn{1}{c}{--}	& \multicolumn{1}{c}{--}	& \multicolumn{1}{c}{--}	& \multicolumn{1}{c}{--}	& \multicolumn{1}{c}{--}	& \multicolumn{1}{c}{--} & \multicolumn{1}{|c}{6} \\
Resource Name Pattern						& \multicolumn{1}{c}{1}		& \multicolumn{1}{c}{--}		& \multicolumn{1}{c}{--}	& \multicolumn{1}{c}{--}	& \multicolumn{1}{c}{--}	& \multicolumn{1}{c}{--}	& \multicolumn{1}{c}{--}	& \multicolumn{1}{c}{--}	& \multicolumn{1}{c}{3}	& \multicolumn{1}{c}{2}	& \multicolumn{1}{c}{3}	& \multicolumn{1}{c}{2}	& \multicolumn{1}{c}{8}	& \multicolumn{1}{c}{2}	& \multicolumn{1}{c}{3}	& \multicolumn{1}{c}{--}	& \multicolumn{1}{c}{--}	& \multicolumn{1}{c}{--} 		& \multicolumn{1}{|c}{8} \\
Magic [Possible] Resource					& \multicolumn{1}{c}{--}		& \multicolumn{1}{c}{--}		& \multicolumn{1}{c}{--}	& \multicolumn{1}{c}{--}	& \multicolumn{1}{c}{--}	& \multicolumn{1}{c}{--}	& \multicolumn{1}{c}{--}	& \multicolumn{1}{c}{--}	& \multicolumn{1}{c}{4}	& \multicolumn{1}{c}{2}	& \multicolumn{1}{c}{1}	& \multicolumn{1}{c}{1}	& \multicolumn{1}{c}{9}	& \multicolumn{1}{c}{6}	& \multicolumn{1}{c}{--}	& \multicolumn{1}{c}{--}	& \multicolumn{1}{c}{--}	& \multicolumn{1}{c}{--} 		& \multicolumn{1}{|c}{6} \\
Nested Layout								& \multicolumn{1}{c}{--}		& \multicolumn{1}{c}{--}		& \multicolumn{1}{c}{--}	& \multicolumn{1}{c}{--}	& \multicolumn{1}{c}{1}	& \multicolumn{1}{c}{--}	& \multicolumn{1}{c}{--}	& \multicolumn{1}{c}{--}	& \multicolumn{1}{c}{9}	& \multicolumn{1}{c}{9}	& \multicolumn{1}{c}{--}	& \multicolumn{1}{c}{--}	& \multicolumn{1}{c}{--}	& \multicolumn{1}{c}{--}	& \multicolumn{1}{c}{--}	& \multicolumn{1}{c}{--}	& \multicolumn{1}{c}{1}	& \multicolumn{1}{c}{1} 		& \multicolumn{1}{|c}{5} \\
MVP											& \multicolumn{1}{c}{8}		& \multicolumn{1}{c}{2}		& \multicolumn{1}{c}{5}	& \multicolumn{1}{c}{--}	& \multicolumn{1}{c}{--}	& \multicolumn{1}{c}{--}	& \multicolumn{1}{c}{--}	& \multicolumn{1}{c}{--}	& \multicolumn{1}{c}{--}	& \multicolumn{1}{c}{--}	& \multicolumn{1}{c}{--}	& \multicolumn{1}{c}{--}	& \multicolumn{1}{c}{--}	& \multicolumn{1}{c}{--}	& \multicolumn{1}{c}{--}	& \multicolumn{1}{c}{--}	& \multicolumn{1}{c}{2}	& \multicolumn{1}{c}{--} 		& \multicolumn{1}{|c}{4} \\
View Coupled To View						& \multicolumn{1}{c}{--}		& \multicolumn{1}{c}{2}		& \multicolumn{1}{c}{4}	& \multicolumn{1}{c}{6}	& \multicolumn{1}{c}{--}	& \multicolumn{1}{c}{3}	& \multicolumn{1}{c}{1}	& \multicolumn{1}{c}{2}	& \multicolumn{1}{c}{--}	& \multicolumn{1}{c}{--}	& \multicolumn{1}{c}{--}	& \multicolumn{1}{c}{--}	& \multicolumn{1}{c}{--}	& \multicolumn{1}{c}{--}	& \multicolumn{1}{c}{--}	& \multicolumn{1}{c}{--}	& \multicolumn{1}{c}{--}	& \multicolumn{1}{c}{--} 		& \multicolumn{1}{|c}{6} \\
Life Cycle									& \multicolumn{1}{c}{4}		& \multicolumn{1}{c}{3}		& \multicolumn{1}{c}{3}	& \multicolumn{1}{c}{5}	& \multicolumn{1}{c}{--}	& \multicolumn{1}{c}{--}	& \multicolumn{1}{c}{1}	& \multicolumn{1}{c}{--}	& \multicolumn{1}{c}{--}	& \multicolumn{1}{c}{--}	& \multicolumn{1}{c}{--}	& \multicolumn{1}{c}{--}	& \multicolumn{1}{c}{--}	& \multicolumn{1}{c}{--}	& \multicolumn{1}{c}{--}	& \multicolumn{1}{c}{--}	& \multicolumn{1}{c}{--}	& \multicolumn{1}{c}{--} 		& \multicolumn{1}{|c}{5} \\
Use Include									& \multicolumn{1}{c}{--}		& \multicolumn{1}{c}{--}		& \multicolumn{1}{c}{--}	& \multicolumn{1}{c}{--}	& \multicolumn{1}{c}{--}	& \multicolumn{1}{c}{--}	& \multicolumn{1}{c}{--}	& \multicolumn{1}{c}{--}	& \multicolumn{1}{c}{12}	& \multicolumn{1}{c}{2}	& \multicolumn{1}{c}{--}	& \multicolumn{1}{c}{--}	& \multicolumn{1}{c}{--}	& \multicolumn{1}{c}{--}	& \multicolumn{1}{c}{--}	& \multicolumn{1}{c}{--}	& \multicolumn{1}{c}{1}	& \multicolumn{1}{c}{--} 	& \multicolumn{1}{|c}{3} \\
Use View Holder Pattern						& \multicolumn{1}{c}{--}		& \multicolumn{1}{c}{--}		& \multicolumn{1}{c}{--}	& \multicolumn{1}{c}{--}	& \multicolumn{1}{c}{12}	& \multicolumn{1}{c}{2}	& \multicolumn{1}{c}{--}	& \multicolumn{1}{c}{--}	& \multicolumn{1}{c}{--}	& \multicolumn{1}{c}{--}	& \multicolumn{1}{c}{--}	& \multicolumn{1}{c}{--}	& \multicolumn{1}{c}{--}	& \multicolumn{1}{c}{--}	& \multicolumn{1}{c}{--}	& \multicolumn{1}{c}{--}	& \multicolumn{1}{c}{--}	& \multicolumn{1}{c}{--} 	& \multicolumn{1}{|c}{2} \\
Drawable Size Matters						& \multicolumn{1}{c}{--}		& \multicolumn{1}{c}{--}		& \multicolumn{1}{c}{--}	& \multicolumn{1}{c}{--}	& \multicolumn{1}{c}{--}	& \multicolumn{1}{c}{--}	& \multicolumn{1}{c}{--}	& \multicolumn{1}{c}{--}	& \multicolumn{1}{c}{1}	& \multicolumn{1}{c}{1}	& \multicolumn{1}{c}{--}	& \multicolumn{1}{c}{--}	& \multicolumn{1}{c}{--}	& \multicolumn{1}{c}{--}	& \multicolumn{1}{c}{4}	& \multicolumn{1}{c}{6}	& \multicolumn{1}{c}{--}	& \multicolumn{1}{c}{--} 		& \multicolumn{1}{|c}{4} \\
Suspicious Extra Knowledge About Behavior	& \multicolumn{1}{c}{1}		& \multicolumn{1}{c}{--}		& \multicolumn{1}{c}{--}	& \multicolumn{1}{c}{--}	& \multicolumn{1}{c}{1}	& \multicolumn{1}{c}{3}	& \multicolumn{1}{c}{5}	& \multicolumn{1}{c}{2}	& \multicolumn{1}{c}{--}	& \multicolumn{1}{c}{--}	& \multicolumn{1}{c}{--}	& \multicolumn{1}{c}{--}	& \multicolumn{1}{c}{--}	& \multicolumn{1}{c}{--}	& \multicolumn{1}{c}{--}	& \multicolumn{1}{c}{--}	& \multicolumn{1}{c}{--}	& \multicolumn{1}{c}{--} 		& \multicolumn{1}{|c}{5} \\
God/Large Class *							& \multicolumn{1}{c}{1}		& \multicolumn{1}{c}{4}		& \multicolumn{1}{c}{1}	& \multicolumn{1}{c}{2}	& \multicolumn{1}{c}{--}	& \multicolumn{1}{c}{1}	& \multicolumn{1}{c}{--}	& \multicolumn{1}{c}{1}	& \multicolumn{1}{c}{--}	& \multicolumn{1}{c}{--}	& \multicolumn{1}{c}{--}	& \multicolumn{1}{c}{1}	& \multicolumn{1}{c}{--}	& \multicolumn{1}{c}{--}	& \multicolumn{1}{c}{--}	& \multicolumn{1}{c}{--}	& \multicolumn{1}{c}{--}	& \multicolumn{1}{c}{--} 		& \multicolumn{1}{|c}{7} \\
Use Fragment								& \multicolumn{1}{c}{3}		& \multicolumn{1}{c}{2}		& \multicolumn{1}{c}{5}	& \multicolumn{1}{c}{1}	& \multicolumn{1}{c}{--}	& \multicolumn{1}{c}{--}	& \multicolumn{1}{c}{--}	& \multicolumn{1}{c}{--}	& \multicolumn{1}{c}{--}	& \multicolumn{1}{c}{--}	& \multicolumn{1}{c}{--}	& \multicolumn{1}{c}{--}	& \multicolumn{1}{c}{--}	& \multicolumn{1}{c}{--}	& \multicolumn{1}{c}{--}	& \multicolumn{1}{c}{--}	& \multicolumn{1}{c}{--}	& \multicolumn{1}{c}{--} 		& \multicolumn{1}{|c}{4} \\
Use Vector Drawables						& \multicolumn{1}{c}{--}		& \multicolumn{1}{c}{--}		& \multicolumn{1}{c}{--}	& \multicolumn{1}{c}{--}	& \multicolumn{1}{c}{--}	& \multicolumn{1}{c}{--}	& \multicolumn{1}{c}{--}	& \multicolumn{1}{c}{--}	& \multicolumn{1}{c}{--}	& \multicolumn{1}{c}{--}	& \multicolumn{1}{c}{--}	& \multicolumn{1}{c}{--}	& \multicolumn{1}{c}{--}	& \multicolumn{1}{c}{--}	& \multicolumn{1}{c}{11}	& \multicolumn{1}{c}{--}	& \multicolumn{1}{c}{--}	& \multicolumn{1}{c}{--} 	& \multicolumn{1}{|c}{1} \\
Use Well Know Architectures					& \multicolumn{1}{c}{--}		& \multicolumn{1}{c}{--}		& \multicolumn{1}{c}{2}	& \multicolumn{1}{c}{--}	& \multicolumn{1}{c}{2}	& \multicolumn{1}{c}{--}	& \multicolumn{1}{c}{--}	& \multicolumn{1}{c}{--}	& \multicolumn{1}{c}{--}	& \multicolumn{1}{c}{--}	& \multicolumn{1}{c}{--}	& \multicolumn{1}{c}{--}	& \multicolumn{1}{c}{--}	& \multicolumn{1}{c}{--}	& \multicolumn{1}{c}{--}	& \multicolumn{1}{c}{--}	& \multicolumn{1}{c}{5}	& \multicolumn{1}{c}{1} 		& \multicolumn{1}{|c}{4} \\
Use Fragment Just If Needed					& \multicolumn{1}{c}{--}		& \multicolumn{1}{c}{--}		& \multicolumn{1}{c}{8}	& \multicolumn{1}{c}{2}	& \multicolumn{1}{c}{--}	& \multicolumn{1}{c}{--}	& \multicolumn{1}{c}{--}	& \multicolumn{1}{c}{--}	& \multicolumn{1}{c}{--}	& \multicolumn{1}{c}{--}	& \multicolumn{1}{c}{--}	& \multicolumn{1}{c}{--}	& \multicolumn{1}{c}{--}	& \multicolumn{1}{c}{--}	& \multicolumn{1}{c}{--}	& \multicolumn{1}{c}{--}	& \multicolumn{1}{c}{--}	& \multicolumn{1}{c}{--} 		& \multicolumn{1}{|c}{2} \\
God Style Resource							& \multicolumn{1}{c}{--}		& \multicolumn{1}{c}{--}		& \multicolumn{1}{c}{--}	& \multicolumn{1}{c}{--}	& \multicolumn{1}{c}{--}	& \multicolumn{1}{c}{--}	& \multicolumn{1}{c}{--}	& \multicolumn{1}{c}{--}	& \multicolumn{1}{c}{--}	& \multicolumn{1}{c}{--}	& \multicolumn{1}{c}{5}	& \multicolumn{1}{c}{3}	& \multicolumn{1}{c}{--}	& \multicolumn{1}{c}{--}	& \multicolumn{1}{c}{--}	& \multicolumn{1}{c}{--}	& \multicolumn{1}{c}{--}	& \multicolumn{1}{c}{--} 		& \multicolumn{1}{|c}{2} \\
Messy String Resources						& \multicolumn{1}{c}{--}		& \multicolumn{1}{c}{--}		& \multicolumn{1}{c}{--}	& \multicolumn{1}{c}{--}	& \multicolumn{1}{c}{--}	& \multicolumn{1}{c}{--}	& \multicolumn{1}{c}{--}	& \multicolumn{1}{c}{--}	& \multicolumn{1}{c}{--}	& \multicolumn{1}{c}{--}	& \multicolumn{1}{c}{--}	& \multicolumn{1}{c}{--}	& \multicolumn{1}{c}{4}	& \multicolumn{1}{c}{4}	& \multicolumn{1}{c}{--}	& \multicolumn{1}{c}{--}	& \multicolumn{1}{c}{--}	& \multicolumn{1}{c}{--} 		& \multicolumn{1}{|c}{2} \\
Avoid Images								& \multicolumn{1}{c}{--}		& \multicolumn{1}{c}{--}		& \multicolumn{1}{c}{--}	& \multicolumn{1}{c}{--}	& \multicolumn{1}{c}{--}	& \multicolumn{1}{c}{--}	& \multicolumn{1}{c}{--}	& \multicolumn{1}{c}{--}	& \multicolumn{1}{c}{1}	& \multicolumn{1}{c}{--}	& \multicolumn{1}{c}{--}	& \multicolumn{1}{c}{--}	& \multicolumn{1}{c}{--}	& \multicolumn{1}{c}{--}	& \multicolumn{1}{c}{4}	& \multicolumn{1}{c}{2}	& \multicolumn{1}{c}{--}	& \multicolumn{1}{c}{--} 		& \multicolumn{1}{|c}{3} \\
Duplicated Styles Attributes				& \multicolumn{1}{c}{--}		& \multicolumn{1}{c}{--}		& \multicolumn{1}{c}{--}	& \multicolumn{1}{c}{--}	& \multicolumn{1}{c}{--}	& \multicolumn{1}{c}{--}	& \multicolumn{1}{c}{--}	& \multicolumn{1}{c}{--}	& \multicolumn{1}{c}{1}	& \multicolumn{1}{c}{2}	& \multicolumn{1}{c}{2}	& \multicolumn{1}{c}{1}	& \multicolumn{1}{c}{--}	& \multicolumn{1}{c}{--}	& \multicolumn{1}{c}{--}	& \multicolumn{1}{c}{--}	& \multicolumn{1}{c}{1}	& \multicolumn{1}{c}{--} 		& \multicolumn{1}{|c}{5} \\
Zumbi Referended Activity					& \multicolumn{1}{c}{2}		& \multicolumn{1}{c}{4}		& \multicolumn{1}{c}{--}	& \multicolumn{1}{c}{--}	& \multicolumn{1}{c}{--}	& \multicolumn{1}{c}{--}	& \multicolumn{1}{c}{--}	& \multicolumn{1}{c}{1}	& \multicolumn{1}{c}{--}	& \multicolumn{1}{c}{--}	& \multicolumn{1}{c}{--}	& \multicolumn{1}{c}{--}	& \multicolumn{1}{c}{--}	& \multicolumn{1}{c}{--}	& \multicolumn{1}{c}{--}	& \multicolumn{1}{c}{--}	& \multicolumn{1}{c}{--}	& \multicolumn{1}{c}{--} 		& \multicolumn{1}{|c}{3} \\
DI											& \multicolumn{1}{c}{1}		& \multicolumn{1}{c}{--}		& \multicolumn{1}{c}{--}	& \multicolumn{1}{c}{--}	& \multicolumn{1}{c}{--}	& \multicolumn{1}{c}{--}	& \multicolumn{1}{c}{4}	& \multicolumn{1}{c}{--}	& \multicolumn{1}{c}{--}	& \multicolumn{1}{c}{--}	& \multicolumn{1}{c}{--}	& \multicolumn{1}{c}{--}	& \multicolumn{1}{c}{--}	& \multicolumn{1}{c}{--}	& \multicolumn{1}{c}{--}	& \multicolumn{1}{c}{--}	& \multicolumn{1}{c}{1}	& \multicolumn{1}{c}{--} 		& \multicolumn{1}{|c}{3} \\
Reuse String Resources Too Much				& \multicolumn{1}{c}{--}		& \multicolumn{1}{c}{--}		& \multicolumn{1}{c}{--}	& \multicolumn{1}{c}{--}	& \multicolumn{1}{c}{--}	& \multicolumn{1}{c}{--}	& \multicolumn{1}{c}{--}	& \multicolumn{1}{c}{--}	& \multicolumn{1}{c}{--}	& \multicolumn{1}{c}{--}	& \multicolumn{1}{c}{--}	& \multicolumn{1}{c}{--}	& \multicolumn{1}{c}{2}	& \multicolumn{1}{c}{4}	& \multicolumn{1}{c}{--}	& \multicolumn{1}{c}{--}	& \multicolumn{1}{c}{--}	& \multicolumn{1}{c}{--} 		& \multicolumn{1}{|c}{2} \\
Flex Adapter								& \multicolumn{1}{c}{--}		& \multicolumn{1}{c}{--}		& \multicolumn{1}{c}{--}	& \multicolumn{1}{c}{--}	& \multicolumn{1}{c}{3}	& \multicolumn{1}{c}{2}	& \multicolumn{1}{c}{--}	& \multicolumn{1}{c}{--}	& \multicolumn{1}{c}{--}	& \multicolumn{1}{c}{1}	& \multicolumn{1}{c}{--}	& \multicolumn{1}{c}{--}	& \multicolumn{1}{c}{--}	& \multicolumn{1}{c}{--}	& \multicolumn{1}{c}{--}	& \multicolumn{1}{c}{--}	& \multicolumn{1}{c}{--}	& \multicolumn{1}{c}{--} 		& \multicolumn{1}{|c}{3} \\
Inheritance **								& \multicolumn{1}{c}{2}		& \multicolumn{1}{c}{--}	& \multicolumn{1}{c}{2}	& \multicolumn{1}{c}{--}	& \multicolumn{1}{c}{1}	& \multicolumn{1}{c}{--}	& \multicolumn{1}{c}{--}	& \multicolumn{1}{c}{--}	& \multicolumn{1}{c}{--}	& \multicolumn{1}{c}{--}	& \multicolumn{1}{c}{--}	& \multicolumn{1}{c}{--}	& \multicolumn{1}{c}{--}	& \multicolumn{1}{c}{--}	& \multicolumn{1}{c}{--}	& \multicolumn{1}{c}{--}	& \multicolumn{1}{c}{--}	& \multicolumn{1}{c}{--} 		& \multicolumn{1}{|c}{3} \\
Hided Listener								& \multicolumn{1}{c}{--}		& \multicolumn{1}{c}{--}		& \multicolumn{1}{c}{--}	& \multicolumn{1}{c}{--}	& \multicolumn{1}{c}{--}	& \multicolumn{1}{c}{--}	& \multicolumn{1}{c}{--}	& \multicolumn{1}{c}{3}	& \multicolumn{1}{c}{--}	& \multicolumn{1}{c}{--}	& \multicolumn{1}{c}{--}	& \multicolumn{1}{c}{--}	& \multicolumn{1}{c}{--}	& \multicolumn{1}{c}{--}	& \multicolumn{1}{c}{--}	& \multicolumn{1}{c}{--}	& \multicolumn{1}{c}{--}	& \multicolumn{1}{c}{--} 		& \multicolumn{1}{|c}{1} \\
Opened Activity								& \multicolumn{1}{c}{--}		& \multicolumn{1}{c}{2}		& \multicolumn{1}{c}{--}	& \multicolumn{1}{c}{--}	& \multicolumn{1}{c}{--}	& \multicolumn{1}{c}{--}	& \multicolumn{1}{c}{--}	& \multicolumn{1}{c}{1}	& \multicolumn{1}{c}{--}	& \multicolumn{1}{c}{--}	& \multicolumn{1}{c}{--}	& \multicolumn{1}{c}{--}	& \multicolumn{1}{c}{--}	& \multicolumn{1}{c}{--}	& \multicolumn{1}{c}{--}	& \multicolumn{1}{c}{--}	& \multicolumn{1}{c}{--}	& \multicolumn{1}{c}{--} 		& \multicolumn{1}{|c}{2} \\
\hline
\multicolumn{1}{c}{\textbf{\#Categorias}}	& \multicolumn{1}{|c}{10} & \multicolumn{1}{c}{8} & \multicolumn{1}{c}{9} & \multicolumn{1}{c}{6} & \multicolumn{1}{c}{7} & \multicolumn{1}{c}{6} & \multicolumn{1}{c}{4} & \multicolumn{1}{c}{6} & \multicolumn{1}{c}{7} & \multicolumn{1}{c}{7} & \multicolumn{1}{c}{4} & \multicolumn{1}{c}{5} & \multicolumn{1}{c}{4} & \multicolumn{1}{c}{4} & \multicolumn{1}{c}{4} & \multicolumn{1}{c}{2} & \multicolumn{1}{c}{6} & \multicolumn{1}{c|}{2} \\
\hline
\multicolumn{20}{@{}l}{* God Classe \cite{Riel} e Large Class \cite{RefactoringFowler1999} s\~ao \textit{code smells} tradicionais previamente definidos em literaturas.} \\
\multicolumn{20}{@{}l}{** \textit{Inheritance}, ou heran\c{c}a, \'e um conceito da Programa\c{c}\~ao Orientada a Objetos \cite{WikipediaInhiritance}.} \\
\toprule
\end{tabular}
\label{tab:Categories}
\end{table*}

Esta se\c{c}\~ao est\'a organizada em 6 subse\c{c}\~oes. As primeiras 5 subse\c{c}\~oes abordam as categorias apenas de alta, m\'edia e baixa recorr\^encia dos temas: Conceitos Tradicionais, Problemas Arquiteturais, Problemas Pontuais, Ferramentas Utilit\'arias e Conhecimento (n\~ao h\'a categorias com essas recorr\^encias no grupo Estrutura de Arquivos). Por \'ultimo temos uma subse\c{c}\~ao para falar apenas das categorias de baix\'issima recorr\^encia.

\subsection{Conceitos Tradicionais}

\subsection{Problemas Arquiteturais}

\subsubsection{Use Fragments Just If Needed} 
Esta categoria agrupa 10 respostas que indicam que se deve evitar a utiliza\c{c}\~ao de \textsc{Fragment}s e s\'o us\'a-los se for realmente necess\'ario. Por exemplo, P4 indica como boa pr\'atica sobre \textsc{Fragments} que ``Tente substituir Fragments o m\'aximo poss\'ivel por ViewGroups. Apenas use-os se tiver uma boa raz\~ao para isso'' (tradu\c{c}\~ao livre), P28 diz ``Se poss\'ivel, evitar a todo custo''. 

\subsubsection{Use Fragments} 
Esta categoria agrupa 11 respostas onde os participantes recomendam o uso de \textsc{Fragments}. Por exemplo, P14 diz ``gosto de sempre usar fragmento, mesmo que seja uma tela simples com somente um fragmento'', j\'a P19 diz ``[...] utilizar Fragments sempre que poss\'ivel''. 

\subsubsection{Nested Layout} 
Esta categoria agrupa 21 respostas onde os participantes indicam \textsc{layout resources} com aninhamento de elementos muito profundos como uma m\'a pr\'atica. De fato, esta categoria poderia ser considerada um android code smell, mas j\'a existe uma ferramenta oficial do Android [] que valida este sintoma, logo, j\'a \'e um problema endere\c{c}ado.

\subsubsection{No Logic In View}

% \begin{table}[h]
% \centering
% \caption{Smells identificados vs. recorr\^encia percebida pelos participantes do question\'ario}
% \small
% \begin{tabular}{p{3cm}|p{4cm}}
% \toprule
% \textbf{Recorr\^encia} & \textbf{Quantidade de Smells} \\
% \hline
% Alta  			& 5 \\
% M\'edia 		& 17 (1 \textit{smell} tradicional) \\
% Baixa			& 5 \\
% Irrelevante 	& 20 \\
% \toprule
% \end{tabular}
% \label{tab:DadosDemograficos2}
% \end{table}

% Os irrelevantes foram desconsiderados nesta etapa de defini\c{c}\~ao. Os demais foram definidos com a ajuda das respostas dos participantes. Para cada \textit{smell} definimos os seguintes t\'opicos: 

% \begin{itemize} 
% 	\item[$\textasteriskcentered$] \textbf{Quando ocorre}. Indicamos os motivos pelo qual foi considerado uma m\'a pr\'atica.   
% 	\item[$\textasteriskcentered$] \textbf{Contexto/exemplo}. Indicamos algum exemplo ou contexto pr\'atico. 
% 	\item[$\textasteriskcentered$] \textbf{Elementos afetados}. Eventualmente algum \textit{smell} afeta mais de um elemento, nesse t\'opico abordamos em quais os elementos este \textit{smell} pode estar presente. 
% 	\item[$\textasteriskcentered$] \textbf{Solu\c{c}\~ao}. Indicamos poss\'iveis refatora\c{c}\~oes para reduzir ou eliminar o \textit{smell}. \\
% \end{itemize}

% \'E importante ressaltar que todas as defini\c{c}\~oes dos \textit{smells} foram apoiadas nas respostas dos participantes. Se os participantes n\~ao indicaram algum t\'opico para algum \textit{smell}, o mesmo n\~ao foi definido. \\


% \subsection{Android Code Smells}
% \label{sub:Android-code-smells}

\subsubsection{Duplicated Styles Attributes (DSA)}
Este smell foi identificado em 7 respostas. Segundo os participantes, ccorre quando mais de um view de um ou mais layouts usam o mesmo conjunto de atributos para definir sua apar\^encia. Por exemplo, segundo P34 \textit{``Sempre que eu noto que tenho mais de um recurso usando o mesmo estilo, eu tento mov\^e-lo para o meu \textit{style resource}.''} (tradu\c{c}\~ao livre). Passando a mesma ideia, P32 diz \textit{``Utilizar muitas propriedades em um \'unico componente. Se tiver que usar muitas, prefiro colocar no arquivo de styles.''}. Ou seja, os elementos afetados por este smell s\~ao layout ou style resources. E as solu\c{c}\~oes indicadas s\~ao extrair um novo estilo que agrupe o conjunto de atributos duplicados e passar a usar o estilo criado.

Por exemplo, considere o seguinte \textbf{contexto} onde o \texttt{TextView} abaixo \'e usado em dois momentos no mesmo layout (exemplo extra\'ido da documenta\c{c}\~ao do Android\footnote{https://developer.Android.com/guide/topics/ui/themes.html}):

\begin{lstlisting}[
	language=XML, 
	caption={\texttt{TextView} usado em mais de um layout e com estilo definido por atributos.}, 
	label={lst:AndroidManifest},
]
<TextView
  android:layout-width="match-parent"
  android:layout-height="wrap-content"
  android:textColor="#00FF00"
  android:typeface="monospace"
  android:text="@string/hello" />
\end{lstlisting}

Visto que o estilo de cor e formata\c{c}\~ao do texto ser\'a replicado, uma op\c{c}\~ao \'e extrair esse estilo para um style resource e utiliz\'a-lo no layout. A seguir, criamos o estilo \textit{CodeFont} e o aplicamos no \textsc{TextView}.

\begin{lstlisting}[
	language=XML, 
	label={lst:AndroidManifest},
	caption={main\_activity.xml},
]
<!-- ... -->
<TextView
  android:layout-width="match-parent"
  android:layout-height="wrap-content"
  android:textAppearance="@style/CodeFont"
  android:text="@string/hello" />
<!-- ... -->
\end{lstlisting}

Arquivo de estilo com o estilo estilo  criado:

\begin{lstlisting}[
	language=XML, 
	caption={styles.xml}, 
	label={lst:AndroidManifest},
]
<?xml version="1.0" encoding="utf-8"?>
<resources>
  <style name="CodeFont" ...>
    <item name="Android:textColor">
      #00FF00
    </item>
    <item name="Android:typeface">
      monospace
    </item>
  </style>
</resources>
\end{lstlisting}


\subsubsection{Suspicious Extra Knowledge About Behavior (SEKAB)}

\textbf{Ocorre quando} uma view possui um conhecimento maior do que o considerado aceit\'avel sobre os detalhes de seu comportamento. Para este smell identificamos por meio das respostas que tr\^es elementos da view \textbf{podem ser afetados} por ele, s\~ao: \textsc{activity}, \textsc{fragment} e \textsc{adapter}. E a depender de qual view est\'a se tratando, o que \'e considerado aceit\'avel pode mudar. Sendo assim, chegamos a seguinte escala de conhecimento aceit\'avel:

\begin{itemize} 
	\item[$\circ$] n\'ivel 1 - implementa\c{c}\~ao de classe concreta;
	\item[$\circ$] n\'ivel 2 - atrav\'es de \textit{implements} (polimorfismo);
	\item[$\circ$] n\'ivel 3 - atrav\'es de classe an\^onima;
	\item[$\circ$] n\'ivel 4 - atrav\'es de \textit{inner class}.
\end{itemize}

Para \textsc{Adapter}s, apenas o n\'ivel 1 foi considerado como aceit\'avel. Para \textsc{Activity}s e \textsc{Fragment}s, at\'e o n\'ivel 3 \'e considerado aceit\'avel, sendo que houveram indica\c{c}\~oes positivas e nenhuma negativa sobre o uso do n\'ivel 1, houveram indica\c{c}\~oes positivas e negativas sobre o n\'ivel 2 e houveram apenas indica\c{c}\~oes negativas sobre o n\'ivel 3 e 4. 

Como \textbf{solu\c{c}\~ao}, segundo as respostas R666 e R672, \textit{"adapters devem ser o mais est\'upido poss\'ivel"} (tradu\c{c}\~ao livre). Podemos dizer ent\~ao que se for necess\'ario atribuir algum comportamento para alguma view que est\'a sendo populada, isso dever\'a ser feito usando a m\'etodo descrito para o n\'ivel 1. J\'a para \textsc{Activity}s e \textsc{Fragment}s a toler\^ancia de conhecimento aumenta at\'e o n\'ivel 3, sendo 1 e 2 os mais indicados e 3 toler\'avel \'as vezes (aqui talvez existam dois thresholds como n\'umero de linhas do m\'etodo do evento ou mesmo quantidade de classes an\^onimas na classe). O n\'ivel 4 \'e inaceit\'avel para todos os elementos afetados por este smell.

Os \textbf{elementos afetados} s\~ao: \textsc{activity}, \textsc{fragment} e \textsc{adapters}. As \textbf{heur\'isticas} de detec\c{c}\~ao s\~ao:

\begin{itemize} 
	\item[$\circ$] [n\'ivel 4] se existe uma ou mais inner class de listener
	\item[$\circ$] [n\'ivel 3] se existe um certo n\'umero de classes an\^onimas
	\item[$\circ$] [n\'ivel 2 e 3] se LOC dentro do m\'etodo do listener - an\^onimo ou sobreescrito \'e maior que 1
\end{itemize}


\subsubsection{Flex Adapter (FA)}

\textbf{Ocorre quando} um adapter precisa de condicionais para decidir se determinada view deve estar visivel ou n\~ao ou quando um adapter recebe mais de um tipo de objeto para ser colocado na view.

Como \textbf{solu\c{c}\~ao}, R578 e R647 sugerem que, caso os condicionais sejam para decidir sobre a exibi\c{c}\~ao ou n\~ao de views, separa em duas ou mais views. Sobre adaptar mais de um tipo de objeto, a resposta R577 sugere a delega\c{c}\~ao para outros adapters especialistas em lidar com cada objeto.


\subsubsection{Magic Resource (MR)}

\textbf{Ocorre quando} se encontra uma string, integer, color ou dimension em arquivos que n\~ao sejam os seus respectivos onde \'e poss\'ivel atribuir um nome e reutiliz\'a-los. 


\subsubsection{Resource Name Pattern (RNP)}

\textbf{Ocorre quando} seus recursos n\~ao seguem um padr\~ao de nomenclatura. Tornando dif\'icil encontr\'a-los e reaproveit\'a-los. A seguir apresentamos uma pasta res desorganizada e despadronizada:

\begin{lstlisting}[
	language=, 
	caption={Recursos de aplica\c{c}\~ao sem padr\~ao de nomenclatura}, 
	label={lst:AndroidManifest}
]
- res
  - layout
    - tela-de-login.xml
    - cadastro-act.xml
    - propaganda-frag.xml
    - item1.xml
    - item2.xml
  - values
    - strings.xml 
    - styles.xml
    - colors.xml
  - drawables
    - ic-app.jpg
    - tela-cadastro.png
    - bg.xml
\end{lstlisting}

Agora, uma pasta res seguindo as sugest\~oes de solu\c{c}\~ao:

\begin{lstlisting}[
	language=, 
	caption={Recursos de aplica\c{c}\~ao com padr\~ao de nomenclatura}, 
	label={lst:AndroidManifest}
]
- res 
  - layout 
    - activity-login.xml 
    - activity-cadastro.xml 
    - fragment-propaganda.xml 
    - item-lista-propaganda-tipo1.xml 
    - item-lista-propaganda-tipo2.xml 
  - values 
    - strings-login.xml  
    - strings-cadastro.xml  
    - strings-propaganda.xml  
    - styles-login.xml 
    - colors.xml 
  - drawables 
    - icone-app.jpg 
    - icone-cadastro.png 
    - background.xml 
\end{lstlisting}

Em se falando de strings resources, segundo P27 "Iniciar o nome de uma string com o nome da tela onde vai ser usada", e P2 refor\c{c}a essa ideia sugerindo <screen>-<type>-<text>. Eg.: welcome-message-title, registration-field-name, registration-hint-edit-name. P34 e P4 sugere que o nome do elemento usando a string como parte do nome da string: dialog.STRING-NAME or hint.STRING-NAME

Em se falando de layout resources, P11 sugere sempre come\c{c}ar com "activity-", "fragment-", "ui-" (para UI customizadas) j\'a P12 sugere o uso de sufixos, como por exemplo "xxx-activity".

Sobre drawables e estilos n\~ao foi sugerida uma maneira espec\'ifica de nomenclatura, apenas enfatizaram para definir um padr\~ao e us\'a-lo em toda a aplica\c{c}\~ao.


\subsubsection{Messy String Resources (MStringR)} 

\textbf{Ocorre quando} se tem um \'unico arquivo strings.xml para guardar todas as strings da aplica\c{c}\~ao ou quando os arquivos string encontram-se desorganizados, sem seguir nenhum padr\~ao espec\'ifico. Como \textbf{solu\c{c}\~ao}, P28 e P42 sugerem separar as strings por arquivos que indiquem a qual tela aquelas strings pertencem. P42 tamb\'em sugere ter um arquivo de strings comuns a toda aplica\c{c}\~ao. P41 sugere ainda separar os arquivos de acordo com contextos como: urls, mensagens de erro. P32 sugere usar um \'unico arquivo por\'em separando-o em blocos, onde casa bloco indica uma tela.



\subsubsection{Messy Style Resources (MStyleR)} 

\textbf{Ocorre quando} se tem um \'unico arquivo styles.xml para guardar todos os estilos e temas da aplica\c{c}\~ao ou quando os arquivos styles encontram-se desorganizados, sem seguir nenhum padr\~ao espec\'ifico. Como \textbf{solu\c{c}\~ao}, P7, P42 e P40 sugerem criar um styles para estilos de componentes e outro para temas. P8 j\'a sugere quebrar em v\'arios arquivos estilos e um para o tema.



\subsection{Problemas Pontuais}

\subsection{Ferramentas Utilit\'arias}

\subsection{Conhecimento}

\subsection{Categorias de Baix\'issima Recorr\^encia}


\subsection{Interpreta\c{c}\~ao dos Dados}

As vezes o que era indicado como boa pr\'atica para um elemento era um smell percebido em outro elemento, por exemplo, a R1 diz que \textit{"Sempre que noto ter mais de um [layout] resource usando o mesmo estilo eu tento mov\^e-lo [o estilo] para meu recurso de estilo"} (tradu\c{c}\~ao livre) ao responder sobre boa pr\'atica para o style resource, por\'em esta resposta foi considerada para definir o smell Duplicated Styles Attributes que \'e percebido em recursos de layout ou styles.


% \subsection{Afirma\c{c}\~ao sobre o \textit{front-end} Android}

% Uma opini\~ao que foi un\^anime em muitas respostas foi que de fato, desenvolvedores tratam \textsc{Activitys}, \textsc{Fragments} e \textsc{Adapters} como elementos do front-end Android, conforme constatamos na se\c{c}\~ao 3.1. Isso pode ser observado diversas vezes com respostas por exemplo, P25 indicou como boa pr\'atica na Activity "Nenhuma l\'ogica [de neg\'ocio] aqui" (tradu\c{c}\~ao livre), o P40 afirma sobre m\'a pr\'atica em adapter \'e \textit{"L\'ogica de neg\'ocio em adapters \'e n\~ao-n\~ao"} (tradu\c{c}\~ao livre), ao falarem sobre fragments, muitos indicaram \textit{"O mesmo da Activity"}. Ou seja, primeiramente estas respostas refor\c{c}am nossa defini\c{c}\~ao inicial sobre elementos que compo\^em o front-end Android, e por outro lado, vimos que muitas vezes fragments s\~ao tratados como Activitys, ao se falar de boas e m\'as pr\'aticas de c\'odigo.
