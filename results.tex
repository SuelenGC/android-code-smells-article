% -*- root: article.tex -*-
Durante o processo de codifica\c{c}\~ao das 18 perguntas sobre boas e m\'as pr\'aticas, 56 categorias emergiram. Agrupamos essas categorias em 6 temas: Conceitos Tradicionais, Problemas Arquiteturais, Problemas Pontuais, Ferramentas Utilit\'arias, Conhecimento e Estrutura de Arquivos. 

Classificamos as categorias de acordo com sua recorr\^encia, ou seja, a quantidade de respostas a qual ela foi atribu\'ida. Utilizamos a seguinte escala: baix\'issima recorr\^encia significa menos de 3 respostas, baixa recorr\^encia significa de 3 a 7 respostas, m\'edia recorr\^encia significa de 8 a 20 respostas e alta recorr\^encia significa acima de 20 respostas.

A Tabela \ref{tab:Categories} apresenta o n\'umero de recorr\^encias, em cada quest\~ao sobre boas e m\'as pr\'aticas, das categorias de alta, m\'edia e baixa recorr\^encia agrupadas por temas. A \'ultima linha da Tabela, \#Categorias, apresenta quantas categorias emergiram de cada quest\~ao, ou seja, \textbf{com base em um elemento Android}, quais s\~ao os pontos de aten\c{c}\~ao (boas e m\'as pr\'aticas) que devemos avaliar pensando em qualidade de c\'odigo. J\'a a \'ultima coluna da Tabela, \#Quest\~oes, apresenta em quantas quest\~oes cada categoria surgiu, ou seja, \textbf{com base na categoria} (que agrupa observa\c{c}\~oes sobre boas e m\'as pr\'aticas), quais elementos devem ser investigados pensando em qualdiade de c\'odigo. 

% Quando se criam categoriza\c{c}\~oes para auxiliar desenvolvedores a identificar pontos de aten\c{c}\~ao a serem avaliados e onde este pontos devem ser investigados especificamente no c\'odigo, pensando em qualidade de c\'odigo d\'a-se o nome de \textit{smells}. Portanto, nesta se\c{c}\~ao iremos compilar este conjunto de Android \textit{code smells} que identificamos atrav\'es das categoriza\c{c}\~oes.

% A partir das categorias foi poss\'ivel criar defini\c{c}\~oes textuais de alguns code smells Android pois as respostas indicavam os sintomas e poss\'iveis solu\c{c}\~oes, entretando, outras categorias n\~ao foi possivel extrair um smells, pois apareceram nas respostas mais como uma discuss\~ao.

\begin{table*}[t]
\centering
\caption{Lista de categorias de alta, m\'edia e baixa recorr\^encia.}
\footnotesize
\begin{tabular}{@{}p{3cm}p{0.7cm}p{.2cm}p{.2cm}p{.2cm}p{.2cm}p{.2cm}p{.2cm}p{.2cm}p{.2cm}p{.2cm}p{.4cm}p{.4cm}p{.4cm}p{.4cm}p{.4cm}p{.4cm}p{.4cm}p{.4cm}p{.4cm}p{1.3cm}@{}}
\toprule
\textbf{Categoria} & \textbf{\#Total} & Q1 & Q2 & Q3 & Q4 & Q5 & Q6 & Q7 & Q8 & Q9 & Q10 & Q11 & Q12 & Q13 & Q14 & Q15 & Q16 & Q17 & Q18 &  \textbf{\#Quest\~oes} \\
\hline
\multicolumn{1}{c}{\scriptsize{\textbf{ALTA RECORR\^ENCIA}}} \\
Lógica em Views								& \multicolumn{1}{c}{53} 	& \multicolumn{1}{c}{12} 	& \multicolumn{1}{c}{15} 	& \multicolumn{1}{c}{6} 	& \multicolumn{1}{c}{8} 	& \multicolumn{1}{c}{3} 	& \multicolumn{1}{c}{8} 	& \multicolumn{1}{c}{--} 	& \multicolumn{1}{c}{1} 	& \multicolumn{1}{c}{--} 	& \multicolumn{1}{c}{--} 	& \multicolumn{1}{c}{--} 	& \multicolumn{1}{c}{--} 	& \multicolumn{1}{c}{--} 	& \multicolumn{1}{c}{--} 	& \multicolumn{1}{c}{--} 	& \multicolumn{1}{c}{--} 	& \multicolumn{1}{c}{--} 	& \multicolumn{1}{c}{--} 	& \multicolumn{1}{c}{7} \\	
Padroniza\c{c}\~ao de Nome					& \multicolumn{1}{c}{24} 	& \multicolumn{1}{c}{1} 	& \multicolumn{1}{c}{--} 	& \multicolumn{1}{c}{--} 	& \multicolumn{1}{c}{--} 	& \multicolumn{1}{c}{--} 	& \multicolumn{1}{c}{--} 	& \multicolumn{1}{c}{--} 	& \multicolumn{1}{c}{--} 	& \multicolumn{1}{c}{3} 	& \multicolumn{1}{c}{2} 	& \multicolumn{1}{c}{3} 	& \multicolumn{1}{c}{2} 	& \multicolumn{1}{c}{8} 	& \multicolumn{1}{c}{2} 	& \multicolumn{1}{c}{3} 	& \multicolumn{1}{c}{--} 	& \multicolumn{1}{c}{--} 	& \multicolumn{1}{c}{--} 	& \multicolumn{1}{c}{8} \\	
Recurso Mágico								& \multicolumn{1}{c}{23} 	& \multicolumn{1}{c}{--} 	& \multicolumn{1}{c}{--} 	& \multicolumn{1}{c}{--} 	& \multicolumn{1}{c}{--} 	& \multicolumn{1}{c}{--} 	& \multicolumn{1}{c}{--} 	& \multicolumn{1}{c}{--} 	& \multicolumn{1}{c}{--} 	& \multicolumn{1}{c}{4} 	& \multicolumn{1}{c}{2} 	& \multicolumn{1}{c}{1} 	& \multicolumn{1}{c}{1} 	& \multicolumn{1}{c}{9} 	& \multicolumn{1}{c}{6} 	& \multicolumn{1}{c}{--} 	& \multicolumn{1}{c}{--} 	& \multicolumn{1}{c}{--} 	& \multicolumn{1}{c}{--} 	& \multicolumn{1}{c}{6} \\	
Views Aninhados								& \multicolumn{1}{c}{21} 	& \multicolumn{1}{c}{--} 	& \multicolumn{1}{c}{--} 	& \multicolumn{1}{c}{--} 	& \multicolumn{1}{c}{--} 	& \multicolumn{1}{c}{1} 	& \multicolumn{1}{c}{--} 	& \multicolumn{1}{c}{--} 	& \multicolumn{1}{c}{--} 	& \multicolumn{1}{c}{9} 	& \multicolumn{1}{c}{9} 	& \multicolumn{1}{c}{--} 	& \multicolumn{1}{c}{--} 	& \multicolumn{1}{c}{--} 	& \multicolumn{1}{c}{--} 	& \multicolumn{1}{c}{--} 	& \multicolumn{1}{c}{--} 	& \multicolumn{1}{c}{1} 	& \multicolumn{1}{c}{1} 	& \multicolumn{1}{c}{5} \\	

\vspace{1sp} \\
\multicolumn{1}{@{}c}{\scriptsize{\textbf{M\'EDIA RECORR\^ENCIA}}} \\ 
Padr\~ao MVP								& \multicolumn{1}{c}{17} 	& \multicolumn{1}{c}{8} 	& \multicolumn{1}{c}{2} 	& \multicolumn{1}{c}{5} 	& \multicolumn{1}{c}{--} 	& \multicolumn{1}{c}{--} 	& \multicolumn{1}{c}{--} 	& \multicolumn{1}{c}{--} 	& \multicolumn{1}{c}{--} 	& \multicolumn{1}{c}{--} 	& \multicolumn{1}{c}{--} 	& \multicolumn{1}{c}{--} 	& \multicolumn{1}{c}{--} 	& \multicolumn{1}{c}{--} 	& \multicolumn{1}{c}{--} 	& \multicolumn{1}{c}{--} 	& \multicolumn{1}{c}{--} 	& \multicolumn{1}{c}{2} 	& \multicolumn{1}{c}{--} 	& \multicolumn{1}{c}{4} \\
View Acoplada com View						& \multicolumn{1}{c}{18} 	& \multicolumn{1}{c}{--} 	& \multicolumn{1}{c}{2} 	& \multicolumn{1}{c}{4} 	& \multicolumn{1}{c}{6} 	& \multicolumn{1}{c}{--} 	& \multicolumn{1}{c}{3} 	& \multicolumn{1}{c}{1} 	& \multicolumn{1}{c}{2} 	& \multicolumn{1}{c}{--} 	& \multicolumn{1}{c}{--} 	& \multicolumn{1}{c}{--} 	& \multicolumn{1}{c}{--} 	& \multicolumn{1}{c}{--} 	& \multicolumn{1}{c}{--} 	& \multicolumn{1}{c}{--} 	& \multicolumn{1}{c}{--} 	& \multicolumn{1}{c}{--} 	& \multicolumn{1}{c}{--} 	& \multicolumn{1}{c}{6} \\
Ciclo de Vida								& \multicolumn{1}{c}{16} 	& \multicolumn{1}{c}{4} 	& \multicolumn{1}{c}{3} 	& \multicolumn{1}{c}{3} 	& \multicolumn{1}{c}{5} 	& \multicolumn{1}{c}{--} 	& \multicolumn{1}{c}{--} 	& \multicolumn{1}{c}{1} 	& \multicolumn{1}{c}{--} 	& \multicolumn{1}{c}{--} 	& \multicolumn{1}{c}{--} 	& \multicolumn{1}{c}{--} 	& \multicolumn{1}{c}{--} 	& \multicolumn{1}{c}{--} 	& \multicolumn{1}{c}{--} 	& \multicolumn{1}{c}{--} 	& \multicolumn{1}{c}{--} 	& \multicolumn{1}{c}{--} 	& \multicolumn{1}{c}{--} 	& \multicolumn{1}{c}{5} \\
Usar Include								& \multicolumn{1}{c}{15} 	& \multicolumn{1}{c}{--} 	& \multicolumn{1}{c}{--} 	& \multicolumn{1}{c}{--} 	& \multicolumn{1}{c}{--} 	& \multicolumn{1}{c}{--} 	& \multicolumn{1}{c}{--} 	& \multicolumn{1}{c}{--} 	& \multicolumn{1}{c}{--} 	& \multicolumn{1}{c}{12} 	& \multicolumn{1}{c}{2} 	& \multicolumn{1}{c}{--} 	& \multicolumn{1}{c}{--} 	& \multicolumn{1}{c}{--} 	& \multicolumn{1}{c}{--} 	& \multicolumn{1}{c}{--} 	& \multicolumn{1}{c}{--} 	& \multicolumn{1}{c}{1} 	& \multicolumn{1}{c}{--} 	& \multicolumn{1}{c}{3} \\
Usar Padr\~ao View Holder					& \multicolumn{1}{c}{14} 	& \multicolumn{1}{c}{--} 	& \multicolumn{1}{c}{--} 	& \multicolumn{1}{c}{--} 	& \multicolumn{1}{c}{--} 	& \multicolumn{1}{c}{12} 	& \multicolumn{1}{c}{2} 	& \multicolumn{1}{c}{--} 	& \multicolumn{1}{c}{--} 	& \multicolumn{1}{c}{--} 	& \multicolumn{1}{c}{--} 	& \multicolumn{1}{c}{--} 	& \multicolumn{1}{c}{--} 	& \multicolumn{1}{c}{--} 	& \multicolumn{1}{c}{--} 	& \multicolumn{1}{c}{--} 	& \multicolumn{1}{c}{--} 	& \multicolumn{1}{c}{--} 	& \multicolumn{1}{c}{--} 	& \multicolumn{1}{c}{2} \\
Tamanho de Imagens Importam					& \multicolumn{1}{c}{12} 	& \multicolumn{1}{c}{--} 	& \multicolumn{1}{c}{--} 	& \multicolumn{1}{c}{--} 	& \multicolumn{1}{c}{--} 	& \multicolumn{1}{c}{--} 	& \multicolumn{1}{c}{--} 	& \multicolumn{1}{c}{--} 	& \multicolumn{1}{c}{--} 	& \multicolumn{1}{c}{1} 	& \multicolumn{1}{c}{1} 	& \multicolumn{1}{c}{--} 	& \multicolumn{1}{c}{--} 	& \multicolumn{1}{c}{--} 	& \multicolumn{1}{c}{--} 	& \multicolumn{1}{c}{4} 	& \multicolumn{1}{c}{6} 	& \multicolumn{1}{c}{--} 	& \multicolumn{1}{c}{--} 	& \multicolumn{1}{c}{4} \\
Comportamento Suspeito						& \multicolumn{1}{c}{14} 	& \multicolumn{1}{c}{1} 	& \multicolumn{1}{c}{1} 	& \multicolumn{1}{c}{--} 	& \multicolumn{1}{c}{--} 	& \multicolumn{1}{c}{1} 	& \multicolumn{1}{c}{3} 	& \multicolumn{1}{c}{5} 	& \multicolumn{1}{c}{3} 	& \multicolumn{1}{c}{--} 	& \multicolumn{1}{c}{--} 	& \multicolumn{1}{c}{--} 	& \multicolumn{1}{c}{--} 	& \multicolumn{1}{c}{--} 	& \multicolumn{1}{c}{--} 	& \multicolumn{1}{c}{--} 	& \multicolumn{1}{c}{--} 	& \multicolumn{1}{c}{--} 	& \multicolumn{1}{c}{--} 	& \multicolumn{1}{c}{6} \\
Classe Deus/Longa							& \multicolumn{1}{c}{11} 	& \multicolumn{1}{c}{1} 	& \multicolumn{1}{c}{4} 	& \multicolumn{1}{c}{1} 	& \multicolumn{1}{c}{2} 	& \multicolumn{1}{c}{--} 	& \multicolumn{1}{c}{1} 	& \multicolumn{1}{c}{--} 	& \multicolumn{1}{c}{1} 	& \multicolumn{1}{c}{--} 	& \multicolumn{1}{c}{--} 	& \multicolumn{1}{c}{--} 	& \multicolumn{1}{c}{1} 	& \multicolumn{1}{c}{--} 	& \multicolumn{1}{c}{--} 	& \multicolumn{1}{c}{--} 	& \multicolumn{1}{c}{--} 	& \multicolumn{1}{c}{--} 	& \multicolumn{1}{c}{--} 	& \multicolumn{1}{c}{7} \\
Use Fragment								& \multicolumn{1}{c}{11} 	& \multicolumn{1}{c}{3} 	& \multicolumn{1}{c}{2} 	& \multicolumn{1}{c}{5} 	& \multicolumn{1}{c}{1} 	& \multicolumn{1}{c}{--} 	& \multicolumn{1}{c}{--} 	& \multicolumn{1}{c}{--} 	& \multicolumn{1}{c}{--} 	& \multicolumn{1}{c}{--} 	& \multicolumn{1}{c}{--} 	& \multicolumn{1}{c}{--} 	& \multicolumn{1}{c}{--} 	& \multicolumn{1}{c}{--} 	& \multicolumn{1}{c}{--} 	& \multicolumn{1}{c}{--} 	& \multicolumn{1}{c}{--} 	& \multicolumn{1}{c}{--} 	& \multicolumn{1}{c}{--} 	& \multicolumn{1}{c}{4} \\
Use Imagens Vetoriais						& \multicolumn{1}{c}{11} 	& \multicolumn{1}{c}{--} 	& \multicolumn{1}{c}{--} 	& \multicolumn{1}{c}{--} 	& \multicolumn{1}{c}{--} 	& \multicolumn{1}{c}{--} 	& \multicolumn{1}{c}{--} 	& \multicolumn{1}{c}{--} 	& \multicolumn{1}{c}{--} 	& \multicolumn{1}{c}{--} 	& \multicolumn{1}{c}{--} 	& \multicolumn{1}{c}{--} 	& \multicolumn{1}{c}{--} 	& \multicolumn{1}{c}{--} 	& \multicolumn{1}{c}{--} 	& \multicolumn{1}{c}{11} 	& \multicolumn{1}{c}{--} 	& \multicolumn{1}{c}{--} 	& \multicolumn{1}{c}{--} 	& \multicolumn{1}{c}{1} \\
Use Arquiteturas Conhecidas					& \multicolumn{1}{c}{9} 	& \multicolumn{1}{c}{--} 	& \multicolumn{1}{c}{--} 	& \multicolumn{1}{c}{1} 	& \multicolumn{1}{c}{--} 	& \multicolumn{1}{c}{2} 	& \multicolumn{1}{c}{--} 	& \multicolumn{1}{c}{--} 	& \multicolumn{1}{c}{--} 	& \multicolumn{1}{c}{--} 	& \multicolumn{1}{c}{--} 	& \multicolumn{1}{c}{--} 	& \multicolumn{1}{c}{--} 	& \multicolumn{1}{c}{--} 	& \multicolumn{1}{c}{--} 	& \multicolumn{1}{c}{--} 	& \multicolumn{1}{c}{--} 	& \multicolumn{1}{c}{5} 	& \multicolumn{1}{c}{1} 	& \multicolumn{1}{c}{4} \\
Fragment Apenas Se Necessário				& \multicolumn{1}{c}{10} 	& \multicolumn{1}{c}{--} 	& \multicolumn{1}{c}{--} 	& \multicolumn{1}{c}{8} 	& \multicolumn{1}{c}{2} 	& \multicolumn{1}{c}{--} 	& \multicolumn{1}{c}{--} 	& \multicolumn{1}{c}{--} 	& \multicolumn{1}{c}{--} 	& \multicolumn{1}{c}{--} 	& \multicolumn{1}{c}{--} 	& \multicolumn{1}{c}{--} 	& \multicolumn{1}{c}{--} 	& \multicolumn{1}{c}{--} 	& \multicolumn{1}{c}{--} 	& \multicolumn{1}{c}{--} 	& \multicolumn{1}{c}{--} 	& \multicolumn{1}{c}{--} 	& \multicolumn{1}{c}{--} 	& \multicolumn{1}{c}{2} \\
Recurso de Estilo Deus						& \multicolumn{1}{c}{8} 	& \multicolumn{1}{c}{--} 	& \multicolumn{1}{c}{--} 	& \multicolumn{1}{c}{--} 	& \multicolumn{1}{c}{--} 	& \multicolumn{1}{c}{--} 	& \multicolumn{1}{c}{--} 	& \multicolumn{1}{c}{--} 	& \multicolumn{1}{c}{--} 	& \multicolumn{1}{c}{--} 	& \multicolumn{1}{c}{--} 	& \multicolumn{1}{c}{5} 	& \multicolumn{1}{c}{3} 	& \multicolumn{1}{c}{--} 	& \multicolumn{1}{c}{--} 	& \multicolumn{1}{c}{--} 	& \multicolumn{1}{c}{--} 	& \multicolumn{1}{c}{--} 	& \multicolumn{1}{c}{--} 	& \multicolumn{1}{c}{2} \\
Recurso de Strings Bagun\c{c}ado			& \multicolumn{1}{c}{8} 	& \multicolumn{1}{c}{--} 	& \multicolumn{1}{c}{--} 	& \multicolumn{1}{c}{--} 	& \multicolumn{1}{c}{--} 	& \multicolumn{1}{c}{--} 	& \multicolumn{1}{c}{--} 	& \multicolumn{1}{c}{--} 	& \multicolumn{1}{c}{--} 	& \multicolumn{1}{c}{--} 	& \multicolumn{1}{c}{--} 	& \multicolumn{1}{c}{--} 	& \multicolumn{1}{c}{--} 	& \multicolumn{1}{c}{4} 	& \multicolumn{1}{c}{4} 	& \multicolumn{1}{c}{--} 	& \multicolumn{1}{c}{--} 	& \multicolumn{1}{c}{--} 	& \multicolumn{1}{c}{--} 	& \multicolumn{1}{c}{2} \\
Evite Imagens								& \multicolumn{1}{c}{7} 	& \multicolumn{1}{c}{--} 	& \multicolumn{1}{c}{--} 	& \multicolumn{1}{c}{--} 	& \multicolumn{1}{c}{--} 	& \multicolumn{1}{c}{--} 	& \multicolumn{1}{c}{--} 	& \multicolumn{1}{c}{--} 	& \multicolumn{1}{c}{--} 	& \multicolumn{1}{c}{1} 	& \multicolumn{1}{c}{--} 	& \multicolumn{1}{c}{--} 	& \multicolumn{1}{c}{--} 	& \multicolumn{1}{c}{--} 	& \multicolumn{1}{c}{--} 	& \multicolumn{1}{c}{4} 	& \multicolumn{1}{c}{2} 	& \multicolumn{1}{c}{--} 	& \multicolumn{1}{c}{--} 	& \multicolumn{1}{c}{3} \\
Atributos de Estilos Duplicados				& \multicolumn{1}{c}{8} 	& \multicolumn{1}{c}{--} 	& \multicolumn{1}{c}{--} 	& \multicolumn{1}{c}{--} 	& \multicolumn{1}{c}{--} 	& \multicolumn{1}{c}{--} 	& \multicolumn{1}{c}{--} 	& \multicolumn{1}{c}{--} 	& \multicolumn{1}{c}{--} 	& \multicolumn{1}{c}{1} 	& \multicolumn{1}{c}{2} 	& \multicolumn{1}{c}{2} 	& \multicolumn{1}{c}{2} 	& \multicolumn{1}{c}{--} 	& \multicolumn{1}{c}{--} 	& \multicolumn{1}{c}{--} 	& \multicolumn{1}{c}{--} 	& \multicolumn{1}{c}{1} 	& \multicolumn{1}{c}{--} 	& \multicolumn{1}{c}{5} \\

\vspace*{1sp} \\
\multicolumn{1}{@{}c}{\scriptsize{\textbf{BAIXA RECORR\^ENCIA}}} \\
Activity Destru\'ida  						& \multicolumn{1}{c}{7} 	& \multicolumn{1}{c}{2} 	& \multicolumn{1}{c}{4} 	& \multicolumn{1}{c}{--} 	& \multicolumn{1}{c}{--} 	& \multicolumn{1}{c}{--} 	& \multicolumn{1}{c}{--} 	& \multicolumn{1}{c}{--} 	& \multicolumn{1}{c}{1} 	& \multicolumn{1}{c}{--} 	& \multicolumn{1}{c}{--} 	& \multicolumn{1}{c}{--} 	& \multicolumn{1}{c}{--} 	& \multicolumn{1}{c}{--} 	& \multicolumn{1}{c}{--} 	& \multicolumn{1}{c}{--} 	& \multicolumn{1}{c}{--} 	& \multicolumn{1}{c}{--} 	& \multicolumn{1}{c}{--} 	& \multicolumn{1}{c}{3} \\
Ferramenta de DI							& \multicolumn{1}{c}{7} 	& \multicolumn{1}{c}{1} 	& \multicolumn{1}{c}{1} 	& \multicolumn{1}{c}{--} 	& \multicolumn{1}{c}{--} 	& \multicolumn{1}{c}{--} 	& \multicolumn{1}{c}{--} 	& \multicolumn{1}{c}{4} 	& \multicolumn{1}{c}{--} 	& \multicolumn{1}{c}{--} 	& \multicolumn{1}{c}{--} 	& \multicolumn{1}{c}{--} 	& \multicolumn{1}{c}{--} 	& \multicolumn{1}{c}{--} 	& \multicolumn{1}{c}{--} 	& \multicolumn{1}{c}{--} 	& \multicolumn{1}{c}{--} 	& \multicolumn{1}{c}{1} 	& \multicolumn{1}{c}{--} 	& \multicolumn{1}{c}{4} \\
Reuso Excessivo de Strings					& \multicolumn{1}{c}{6} 	& \multicolumn{1}{c}{--} 	& \multicolumn{1}{c}{--} 	& \multicolumn{1}{c}{--} 	& \multicolumn{1}{c}{--} 	& \multicolumn{1}{c}{--} 	& \multicolumn{1}{c}{--} 	& \multicolumn{1}{c}{--} 	& \multicolumn{1}{c}{--} 	& \multicolumn{1}{c}{--} 	& \multicolumn{1}{c}{--} 	& \multicolumn{1}{c}{--} 	& \multicolumn{1}{c}{--} 	& \multicolumn{1}{c}{2} 	& \multicolumn{1}{c}{4} 	& \multicolumn{1}{c}{--} 	& \multicolumn{1}{c}{--} 	& \multicolumn{1}{c}{--} 	& \multicolumn{1}{c}{--} 	& \multicolumn{1}{c}{2} \\
Adapter Flex\'ivel							& \multicolumn{1}{c}{6} 	& \multicolumn{1}{c}{--} 	& \multicolumn{1}{c}{--} 	& \multicolumn{1}{c}{--} 	& \multicolumn{1}{c}{--} 	& \multicolumn{1}{c}{3} 	& \multicolumn{1}{c}{2} 	& \multicolumn{1}{c}{--} 	& \multicolumn{1}{c}{--} 	& \multicolumn{1}{c}{--} 	& \multicolumn{1}{c}{1} 	& \multicolumn{1}{c}{--} 	& \multicolumn{1}{c}{--} 	& \multicolumn{1}{c}{--} 	& \multicolumn{1}{c}{--} 	& \multicolumn{1}{c}{--} 	& \multicolumn{1}{c}{--} 	& \multicolumn{1}{c}{--} 	& \multicolumn{1}{c}{--} 	& \multicolumn{1}{c}{3} \\
Heran\c{c}a									& \multicolumn{1}{c}{5} 	& \multicolumn{1}{c}{2} 	& \multicolumn{1}{c}{--} 	& \multicolumn{1}{c}{2} 	& \multicolumn{1}{c}{--} 	& \multicolumn{1}{c}{1} 	& \multicolumn{1}{c}{--} 	& \multicolumn{1}{c}{--} 	& \multicolumn{1}{c}{--} 	& \multicolumn{1}{c}{--} 	& \multicolumn{1}{c}{--} 	& \multicolumn{1}{c}{--} 	& \multicolumn{1}{c}{--} 	& \multicolumn{1}{c}{--} 	& \multicolumn{1}{c}{--} 	& \multicolumn{1}{c}{--} 	& \multicolumn{1}{c}{--} 	& \multicolumn{1}{c}{--} 	& \multicolumn{1}{c}{--} 	& \multicolumn{1}{c}{3} \\
Listener Escondido							& \multicolumn{1}{c}{3} 	& \multicolumn{1}{c}{--} 	& \multicolumn{1}{c}{--} 	& \multicolumn{1}{c}{--} 	& \multicolumn{1}{c}{--} 	& \multicolumn{1}{c}{--} 	& \multicolumn{1}{c}{--} 	& \multicolumn{1}{c}{--} 	& \multicolumn{1}{c}{3} 	& \multicolumn{1}{c}{--} 	& \multicolumn{1}{c}{--} 	& \multicolumn{1}{c}{--} 	& \multicolumn{1}{c}{--} 	& \multicolumn{1}{c}{--} 	& \multicolumn{1}{c}{--} 	& \multicolumn{1}{c}{--} 	& \multicolumn{1}{c}{--} 	& \multicolumn{1}{c}{--} 	& \multicolumn{1}{c}{--} 	& \multicolumn{1}{c}{1} \\
Opera\c{c}ões de IO							& \multicolumn{1}{c}{6} 	& \multicolumn{1}{c}{--} 	& \multicolumn{1}{c}{3} 	& \multicolumn{1}{c}{--} 	& \multicolumn{1}{c}{2} 	& \multicolumn{1}{c}{--} 	& \multicolumn{1}{c}{1} 	& \multicolumn{1}{c}{--} 	& \multicolumn{1}{c}{--} 	& \multicolumn{1}{c}{--} 	& \multicolumn{1}{c}{--} 	& \multicolumn{1}{c}{--} 	& \multicolumn{1}{c}{--} 	& \multicolumn{1}{c}{--} 	& \multicolumn{1}{c}{--} 	& \multicolumn{1}{c}{--} 	& \multicolumn{1}{c}{--} 	& \multicolumn{1}{c}{--} 	& \multicolumn{1}{c}{--} 	& \multicolumn{1}{c}{3} \\
\hline
\multicolumn{2}{r}{\textbf{\#Categorias}}	& \multicolumn{1}{c}{10} 	& \multicolumn{1}{c}{10} 	& \multicolumn{1}{c}{9} 	& \multicolumn{1}{c}{7} 	& \multicolumn{1}{c}{7} 	& \multicolumn{1}{c}{7} 	& \multicolumn{1}{c}{4} 	& \multicolumn{1}{c}{6} 	& \multicolumn{1}{c}{7} 	& \multicolumn{1}{c}{7} 	& \multicolumn{1}{c}{4} 	& \multicolumn{1}{c}{5} 	& \multicolumn{1}{c}{4} 	& \multicolumn{1}{c}{4} 	& \multicolumn{1}{c}{4} 	& \multicolumn{1}{c}{2} 	& \multicolumn{1}{c}{6} 	& \multicolumn{1}{c}{2} \\
\hline
\multicolumn{20}{@{}l}{* \textit{God Classe} \cite{Riel} e \textit{Large Class} \cite{RefactoringFowler1999} s\~ao \textit{code smells} tradicionais previamente definidos em literaturas.} \\
\multicolumn{20}{@{}l}{** \textit{Inheritance}, ou heran\c{c}a, \'e um conceito da Programa\c{c}\~ao Orientada a Objetos \cite{WikipediaInhiritance}.} \\
\multicolumn{20}{@{}l}{A linha \textbf{\#Categorias} indica quantas categorias diferentes surgiram a partir de uma quest\~ao.} \\
\multicolumn{20}{@{}l}{A coluna \textbf{\#Quest\~oes} indica quantas quest\~oes diferentes cont\'em uma categoria.} \\
\toprule
\end{tabular}
\label{tab:Categories}
\end{table*}

Esta se\c{c}\~ao est\'a organizada em 6 subse\c{c}\~oes. As primeiras 5 subse\c{c}\~oes abordam as categorias apenas de alta, m\'edia e baixa recorr\^encia dos temas: Conceitos Tradicionais, Problemas Arquiteturais, Problemas Pontuais, Ferramentas Utilit\'arias e Conhecimento (n\~ao h\'a categorias com essas recorr\^encias no grupo Estrutura de Arquivos). Por \'ultimo temos uma subse\c{c}\~ao para falar apenas das categorias de baix\'issima recorr\^encia.

\subsection{Categorias de Alta Recorr\^encia}
Obtivemos 4 categorias consideradas de alta recorr\^encia: No Logic In View, Resource Name Pattern, Magic Resource e Nested Layout. A quantidade de respostas recebida por cada categoria \'e apresentada na Tabela \ref{tab:Categories}.

\begin{table*}[t]
\centering
\caption{Lista de categorias de alta recorr\^encia.}
\footnotesize
\begin{tabular}{@{}p{2.9cm}p{1cm}|p{0.3cm}p{0.3cm}p{0.3cm}p{0.3cm}p{0.3cm}p{0.3cm}p{0.3cm}p{0.3cm}p{0.4cm}p{0.4cm}p{0.4cm}p{0.4cm}p{0.4cm}p{0.4cm}p{0.4cm}p{0.4cm}|p{1.5cm}@{}}
\toprule
\textbf{Categoria}	& \textbf{\#Total} 	& Q1 	& Q2 	& Q3 	& Q4 	& Q5 	& Q6 	& Q8 	& Q9 	& Q10 	& Q11 	& Q12 	& Q13 	& Q14 	& Q15 	& Q17 	& Q18 	&  \textbf{\#Quest\~oes} \\
\hline
No Logic In View				& 	\multicolumn{1}{c|}{53} 	& \multicolumn{1}{c}{12} 	& \multicolumn{1}{c}{15} 	& \multicolumn{1}{c}{6} 	& \multicolumn{1}{c}{8} 	& \multicolumn{1}{c}{3} 	& \multicolumn{1}{c}{8} 	& \multicolumn{1}{c}{1} 	& \multicolumn{1}{c}{--} 	& \multicolumn{1}{c}{--} 	& \multicolumn{1}{c}{--} 	& \multicolumn{1}{c}{--} 	& \multicolumn{1}{c}{--} 	& \multicolumn{1}{c}{--} 	& \multicolumn{1}{c}{--} 	& \multicolumn{1}{c}{--} 	& \multicolumn{1}{c|}{0} 	& \multicolumn{1}{c}{7}	\\
Resource Name Pattern			& 	\multicolumn{1}{c|}{24} 	& \multicolumn{1}{c}{1} 	& \multicolumn{1}{c}{--} 	& \multicolumn{1}{c}{--} 	& \multicolumn{1}{c}{--} 	& \multicolumn{1}{c}{--} 	& \multicolumn{1}{c}{--} 	& \multicolumn{1}{c}{--} 	& \multicolumn{1}{c}{3} 	& \multicolumn{1}{c}{2} 	& \multicolumn{1}{c}{3} 	& \multicolumn{1}{c}{2} 	& \multicolumn{1}{c}{8} 	& \multicolumn{1}{c}{2} 	& \multicolumn{1}{c}{3} 	& \multicolumn{1}{c}{--} 	& \multicolumn{1}{c|}{0} 	& \multicolumn{1}{c}{8}	\\
Magic Resource					& 	\multicolumn{1}{c|}{23} 	& \multicolumn{1}{c}{--} 	& \multicolumn{1}{c}{--} 	& \multicolumn{1}{c}{--} 	& \multicolumn{1}{c}{--} 	& \multicolumn{1}{c}{--} 	& \multicolumn{1}{c}{--} 	& \multicolumn{1}{c}{--} 	& \multicolumn{1}{c}{4} 	& \multicolumn{1}{c}{2} 	& \multicolumn{1}{c}{1} 	& \multicolumn{1}{c}{1} 	& \multicolumn{1}{c}{9} 	& \multicolumn{1}{c}{6} 	& \multicolumn{1}{c}{--} 	& \multicolumn{1}{c}{--} 	& \multicolumn{1}{c|}{0} 	& \multicolumn{1}{c}{6}	\\
Nested Layout					& 	\multicolumn{1}{c|}{21} 	& \multicolumn{1}{c}{--} 	& \multicolumn{1}{c}{--} 	& \multicolumn{1}{c}{--} 	& \multicolumn{1}{c}{--} 	& \multicolumn{1}{c}{1} 	& \multicolumn{1}{c}{--} 	& \multicolumn{1}{c}{--} 	& \multicolumn{1}{c}{9} 	& \multicolumn{1}{c}{9} 	& \multicolumn{1}{c}{--} 	& \multicolumn{1}{c}{--} 	& \multicolumn{1}{c}{--} 	& \multicolumn{1}{c}{--} 	& \multicolumn{1}{c}{--} 	& \multicolumn{1}{c}{1} 	& \multicolumn{1}{c|}{1} 	& \multicolumn{1}{c}{5}	\\
% \hline
% \multicolumn{2}{r}{\textbf{\#Total}}										& \multicolumn{1}{c}{13} 	& \multicolumn{1}{c}{15} 	& \multicolumn{1}{c}{6} 	& \multicolumn{1}{c}{8} 	& \multicolumn{1}{c}{4} 	& \multicolumn{1}{c}{8} 	& \multicolumn{1}{c}{1} 	& \multicolumn{1}{c}{16} 	& \multicolumn{1}{c}{13} 	& \multicolumn{1}{c}{4} 	& \multicolumn{1}{c}{3} 	& \multicolumn{1}{c}{17} 	& \multicolumn{1}{c}{8} 	& \multicolumn{1}{c}{3} 	& \multicolumn{1}{c}{1} 	& \multicolumn{1}{c}{1} 	& \\
\hline
\multicolumn{19}{@{}l}{\scriptsize{* As categorias de alta recorr\^encia n\~ao n\~ao receberam respostas \`as quest\~oes Q7 e Q16.}} \\
\toprule
\end{tabular}
\label{tab:CategoriasAltaRecorrencia}
\end{table*}

\subsubsection{No Logic In View}
Esta categoria re\'une respostas que indicam como m\'a pr\'atica haver regra de neg\'ocio nos elementos Android afetados. De forma similar, respostas indicam como boas pr\'aticas n\~ao haver c\'odigo de regra de neg\'ocio. Exemplos de frases que indicaram m\'as pr\'aticas s\~ao: P16 sobre \textsc{Activitys} diz \textit{``Fazer l\'ogica de neg\'ocio''} (tradu\c{c}\~ao livre), P19 diz \textit{``Colocar regra de neg\'ocio no adapter''} e P11 diz \textit{``Manter l\'ogica de neg\'ocio em Fragments''} (tradu\c{c}\~ao livre). Exemplos de frases que indicaram boas pr\'atica s\~ao: P16 diz sobre \textsc{Activitys} \textit{``Elas representam uma \'unica tela e apenas interagem com a UI, qualquer l\'ogica deve ser delegada para outra classe''} (tradu\c{c}\~ao livre), P23 diz \textit{``Apenas c\'odigo relacionado \`a Interface de Usu\'ario nas Activities''}, P40 diz \textit{``Adapters devem apenas se preocupar sobre como mostrar os dados, sem trabalh\'a-los''}. Os elementos que entraram nessa categoria foram: \textsc{Activitys}, \textsc{Fragments}, \textsc{Listeners} e \textsc{Adapters}. 


\subsubsection{Resource Name Pattern}
Esta categoria re\'une respostas que indicam como m\'a pr\'atica o n\~ao uso de um padr\~ao de nomenclatura a ser usado nos recursos da aplica\c{c}\~ao. De forma similar, respostas indicam como boas pr\'aticas o uso de um padr\~ao de nomenclatura a ser usados nos recursos. Exemplos de frases que indicaram m\'as pr\'aticas s\~ao: P8 sobre \textsc{Style Resources} diz \textit{``[...] o nome das strings sem um contexto''} (tradu\c{c}\~ao livre), P37 tamb\'em sobre \textsc{Style Resources} diz \textit{``Nada al\'em de ter uma boa conven\c{c}\~ao de nomes''} (tradu\c{c}\~ao livre), ainda P37, por\'em sobre \textsc{Layout Resources} diz \textit{``Mantenha uma conven\c{c}\~ao de nomes da sua escolha [...]''} (tradu\c{c}\~ao livre). Exemplos de frases que indicaram boas pr\'atica s\~ao: P27 diz sobre \textsc{String Resources} \textit{``Iniciar o nome de uma string com o nome da tela onde vai ser usada''}, P43 sobre \textsc{Layout Resources} diz \textit{``Ter uma boa conven\c{c}\~ao de nomea\c{c}\~ao''} (tradu\c{c}\~ao livre), P11 diz sobre \textsc{Style Resources} \textit{``[...] colocar um bom nome [...]''} (tradu\c{c}\~ao livre). Os elementos que entraram nessa categoria foram: \textsc{Activitys}, \textsc{Layout Resources}, \textsc{String Resources}, \textsc{Style Resources} e \textsc{Drawable Resources}. 

Dentre as respostas, algumas indicaram padr\~oes de prefer\^encia. P11 indica usar prefixos nos \textsc{Layout Resources}: \texttt{activity\_}, \texttt{fragment\_}, \texttt{ui\_} (para UI customizadas). P12 sugeriu usar sufixos em \textsc{Activitys}: \texttt{\_Activity}. Os padr\~oes indicados para \textsc{String Resources} foram: P27 indicou \textit{``Iniciar o nome da string com o nome da tela onde vai ser usada''}, P6 sugeriu a conven\c{c}\~ao \texttt{[screen]\_[type]\_[text]} e citou como exemplo \texttt{welcome\_message\_title}. P34 indicou que deve-se usar como prefixo o recurso usando a string, por exemplo \texttt{dialog.STRING\_NAME} ou \texttt{hint.STRING\_NAME}. De forma similar por\'em sem sugerir um exemplo, P4 sugeriu basear o nome da string no nome do recurso que a esta usando. N\~ao foram sugeridos nenhum padr\~ao para \textsc{Styles Resources} e \textsc{Drawable Resources}.

\subsubsection{Magic Resource}
Esta categoria re\'une respostas que indicam como m\'a pr\'atica o uso direto de valores como, por exemplo, strings, n\'umeros e cores, sem a cria\c{c}\~ao um recurso. De forma similar, respostas indicam como boas pr\'aticas o uso de um padr\~ao de nomenclatura a ser usados nos recursos. O nome dessa foi inspirado no \textit{code smell} \textit{Magic Number} \cite{Martin:2008:CCH:1388398} que trata sobre n\'umeros usados diretamente no c\'odigo. Exemplos de frases que indicaram m\'as pr\'aticas s\~ao: P23 diz \textit{``Strings diretamente no c\'odigo''}, P31 e P35 falam respectivamente sobre n\~ao extrair as strings e sobre n\~ao extrair os valores dos arquivos de layout. Exemplos de frases que indicaram boas pr\'atica s\~ao: P7 diz \textit{``Sempre pegar valores de string ou dp de seus respectivos resources para facilitar''}, P36 diz para \textit{``sempre adicionar as strings em resources para traduzir em diversos idiomas [...]''}. Os elementos que entraram nessa categoria foram: \textsc{Layout Resources}, \textsc{String Resources} e \textsc{Style Resources}. 

\subsubsection{Nested Layout} 
Esta categoria re\'une respostas que indicam como m\'a pr\'atica o uso de profundos aninhamentos na constru\c{c}\~ao de layouts. De forma similar, respostas indicam como boas pr\'aticas evitar ao m\'aximo o aninhamento de \textit{views}. Exemplos de frases que indicaram m\'as pr\'aticas s\~ao: P26 diz \textit{``Hierarquia de views longas''} (tradu\c{c}\~ao livre), P4 aborda a mesma ideia ao dizer \textit{``Estruturas profundamente aninhadas''} (tradu\c{c}\~ao livre), P39 diz \textit{``Hierarquias desnecess\'arias''} e P45 diz \textit{``Criar muitos ViewGroups dentro de ViewGroups''}. Exemplos de frases que indicaram boas pr\'atica s\~ao: P4 diz \textit{``tento usar o m\'inimo de layout aninhado''}, P19 diz \textit{``Utilizar o m\'inimo de camadas poss\'ivel''}, P8 diz \textit{``[...] n\~ao fazer uma hierarquia profunda de ViewGroups [...]''}. Apenas o elemento \textsc{Layout Resources} recebeu esta categoria categoria. O site oficial do Android conta com informa\c{c}\~oes e ferramentas automatizadas para lidar com esse sintoma \cite{OptmizingViewHierarchies}.


\subsection{Categorias de M\'edia Recorr\^encia}
Obtivemos 15 categorias consideradas de m\'edia recorr\^encia: View Coupled To View, MVP, Life Cycle, Use Include, Use View Holder Pattern, Suspicious Extra Knowledge About Behavior, Drawable Size Matters, God/Large Class, Use Fragment, Use Vector Drawables, Use Well Know Architectures, Use Fragment Just If Needed, God Style Resource, Messy String Resources e Duplicated Styles Attributes.


\subsection{Categorias de Baixa Recorr\^encia}
Obtivemos 8 categorias consideradas de baixa recorr\^encia: Avoid Images, Zumbi Referended Activity, DI, Reuse String Resources Too Much, Flex Adapter, IO Operations, Inheritance e Hided Listener.


\subsection{Categorias de Baix\'issima Recorr\^encia}
Obtivemos 29 categorias consideradas de baix\'issima recorr\^encia: MVC, Clean Architecture, Don't Use Fragment, Nested Fragment, Mix String Resources With Business Logic, Use 9 Patch Files, Dealing With App Stack Manually, Styles Knows Too Much, Package Structure, MVVM, Activity Handle More Than One Layout, Bad Relative, Deprecated Attributes, Fat onCreate, HTML Into String File, Inherit From Support Library Always, Listener Has A Valid Context, Safe Adapter, Single Activity, Static Things On Adapter, Style Into String File, Use Constraint Layout, Dead Resources, DRY, SOLID, Reuse, Singleton, Opened Activity e Unnecessary ViewGroup








