% -*- root: article.tex -*-
Muitas pesquisas t\^em sido realizadas sobre a plataforma Android, muitas delas focam em vulnerabilidades \cite{Y, F, G, X, P, D, E}, autentica\c{c}\~ao \cite{T, Yamashita6405287, R} e testes \cite{J, M}. Diferentemente dessas pesquisas, nossa pesquisa tem foco na percep\c{c}\~ao dos desenvolvedores sobre boas e m\'as pr\'aicas de desenvolvimento na plataforma Android. 

A percep\c{c}\~ao desempenha um importante papel na defini\c{c}\~ao de code smells relacionados a uma tecnologia, visto que code smells possuem uma natureza subjetiva. Code smells desempenham um importante papel na busca por qualidade de c\'odigo, visto que, ap\'os mapeados, podemos chegar a heur\'isticas para identific\'a-los e com essas heur\'isticas, implementar ferramentas que automatizem o processo de identificar c\'odigos problem\'aticos.

Verloop \cite{MobileSmells:13} conduziu um estudo no qual avaliou por meio de 4 ferramentas de detec\c{c}\~ao automatizada de cheiros de c\'odigo (JDeodorant, Checkstyle, PMD e UCDetector) a presen\c{c}a de 5 cheiros de c\'odigo (Long Method, Large Class, Long Parameter List, Feature Envy e Dead Code) em 4 projetos Android. Nossa pesquisa se relaciona com a de Verloop no sentido de que tamb\'em estamos buscando por cheiros de c\'odigo, entretanto, em vez de de buscarmos por cheiros de c\'odigo j\'a definidos, realizamos uma abordagem inversa na qual, primeiro buscamos entender a percep\c{c}\~ao de desenvolvedores sobre boas e m\'as pr\'aticas em Android, e a partir dessa percep\c{c}\~ao, relacionamos com algum cheiro de c\'odigo pr\'e-existente ou derivamos algum novo.

Gottschalk et al \cite{EnergyAndroidSmells} conduziram um estudo sobre formas de detectar e refatorar cheiros de c\'odigo relacionados ao uso efici\^ente de energia. Os autores compilaram um cat\'alogo com 8 cheiros de c\'odigo e trabalharam sob um trecho de c\'odigo Android para exemplificar um deles, o "binding resource too early", quando algum recurso \'e alocado muito antes de precisar ser utilizado. Essa pesquisa \'e relacionada \`a nossa por ambas considerarem a tecnologia Android e se diferenciam pois focamos na busca por cheiros de c\'odigo relacionados a qualidade de c\'odigo, no sentido de legibilidade e manutenablidade.

Aplicativos Android s\~ao escritos na linguagem de programa\c{c}\~ao Java \cite{AndroidFundamentals}. Ent\~ao a primeira quest\~ao \'e: por que buscar por \textit{smells} Android sendo que j\'a existem tantos \textit{smells} Java? Pesquisas t\^em demonstrado que tecnologias diferentes podem apresentar \textit{code smells} espec\'ificos, como por exemplo Aniche et al. identificaram 6 \textit{code smells} espec\'ificos ao framework Spring MVC, um framework Java para desenvolvimento web. Outras pesquisas concluem que projetos Android possuem caracter\'isticas diferentes de projetos java \cite{Hecht2015, Mannan_Dig_Ahmed_Jensen_Abdullah_Almurshed, ReimannBrylski2013}, por exemplo, o \textit{front-end} \'e representado por arquivos XML e o ponto de entrada da aplica\c{c}\~ao \'e dado por \textit{event-handler} \cite{AndroidActivities2016} como o m\'etodo \textsc{onCreate}. Encontramos tamb\'em diversas pesquisas sobre \textit{code smells} sobre tecnologias usadas no desenvolvimento de \textit{front-end} web como CSS \cite{CSSCodeSmell} e JavaScript \cite{BB}. Essas pesquisas nos inspiraram a buscar entender se existem \textit{code smells} no \textit{front-end} Android. \\


\textbf{[se\c{c}\~ao n\~ao finalizada, \'a concluir.]}