% -*- root: article.tex -*-
Muitas pesquisas t\^em sido realizadas sobre a plataforma android, muitas delas focam em vulnerabilidades [25, 43, 27, 28, 81, 87, 88], autentica\c{c}\~ao [29, 30, 82, 85] e testes [17, 40]. Diferentemente destas pesquisas, nossa pesquisa tem foco na percep\c{c}\~ao dos desenvolvedores sobre boas e m\'as pr\'aicas de desenvolvimento na plataforma android. A percep\c{c}\~ao desempenha um importante papel na defini\c{c}\~ao de code smells relacionados a uma tecnologia, visto que code smells possuem uma natureza subjetiva. Code smells desempenham um importante papel na busca por qualidade de c\'odigo, visto que, ap\'os mapeados code smells, podemos chegar a heur\'isticas para identific\'a-los e com estas heur\'isticas, implementar ferramentas que automatizem o processo de identificar c\'odigos maus cheirosos.

Verloop \cite{verloop:13} conduziu um estudo onde avaliou por meio de 4 ferramentas de detec\c{c}\~ao automatizada de cheiros de c\'odigo (JDeodorant, Checkstyle, PMD e UCDetector) a presen\c{c}a de 5 cheiros de c\'odigo (Long Method, Large Class, Long Parameter List, Feature Envy e Dead Code) em 4 projetos android. Nossa pesquisa se relaciona com a de Verloop no sentido de que tamb\'em estamos buscando por cheiros de c\'odigo, entretanto, ao inv\'es de buscarmos por cheiros de c\'odigo j\'a definidos, realizamos uma abordagem inversa onde, primeiro buscamos entender a percep\c{c}\~ao de desenvolvedores sobre boas e m\'as pr\'aticas em android, e a partir dessa percep\c{c}\~ao, relacionamos com algum cheiro de c\'odigo pr\'e-existente ou derivamos algum novo.