% -*- root: article.tex -*-
Uma limitação deste artigo é que os dados foram coletados apenas a partir de questionários online e o processo de codificação foi realizado apenas por um dos autores. Alterativas a esses cenários seriam realizar a coleta de dados de outras formas como entrevistas ou consulta a especialistas, e que o processo de codificação fosse feito por mais de um autor de forma a reduzir possíveis enviezamentos.

Outra possível ameaça é com relação a seleção de códigos limpos. Selecionar códigos limpos é difícil. Sentimos uma dificuldade maior ao selecionar códigos de recursos do Android. Um alternativa seria investigar a existência de ferramentas que façam esta selecão, validar os códigos selecionados com um especialista ou mesmo extender o teste piloto. 

Nossa pesquisa tenta replicar o método utilizado por Aniche \cite{FinavaroAniche2016} ao investigar cheiros de código no framework Spring MVC. Entretanto, nos deparamos com situações diferentes, das quais, após a execução nos questionamos se aquele método seria o mais adequado para todos os contextos neste artigo. Por exemplo, nosso resultado com a má prática RM nos levou a conjecturar se desenvolvedores consideram problemas em códigos Java mais severos que problemas em recursos do aplicativo. O que nos levou a pensar sobre isso foi que, apesar do resultado, obtivemos muitas respostas que se aproximavam da deifnição da má prática RM. Desta forma, levantamos que de todos os recursos avaliados, 74\% receberam severidade igual ou inferior a 3, contra apenas 30\% com os mesmo níveis de severidade em código Java. Desta forma, uma alternativa é repensar a forma de avaliar a percepção dos desenvolvedores sobre más práticas que afetem recursos do aplicativo.
