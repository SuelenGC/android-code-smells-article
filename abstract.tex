% -*- root: article.tex -*-
Existem diversas práticas, ferramentas e padrões que auxiliam desenvolvedores a produzir código com qualidade. Dentre elas podemos citar catálogos de maus cheiros, que indicam possíveis problemas no código. Esses catálogos possibilitam a implementação de ferramentas de detecção automática de trechos de códigos problemáticos ou mesmo a inspeção manual. Apesar de já existirem diversos cheiros de código catalogados, pesquisas sugerem que tecnologias diferentes apresentaram cheiros de código específicos, e uma tecnologia que tem chamado a atenção de muitos pesquisadores é o Android. Neste artigo, nós investigamos a existência de cheiros de código em projetos Android. Como ponto de partida, por meio de um questionário online com 45 desenvolvedores, catalogamos 23 más práticas Android. Por meio de um experimento online, validamos a perceção de 20 desenvolvedores sobre quatro dessas más práticas, onde duas se confirmaram estatísticamente. Esperamos que nosso catálogo e metodologia de pesquisa, bem como sugestões de como mitigar as ameaças à validade possam colaborar com outros pesquisadores.


% Cheiros de código são aliados na busca pela qualidade de código durante o desenvolvimento de software pois possibilitam a implementação de ferramentas de detecção automática de trechos de códigos problemáticos ou mesmo a inspeção manual. Apesar de já existirem diversos cheiros de código catalogados, pesquisas sugerem que tecnologias diferentes apresentaram cheiros de código específicos, e uma tecnologia que tem chamado a atenção de muitos pesquisadores é o Android. Neste artigo, nós investigamos a existência de maus cheiros em projetos Android. Por meio de um \textit{survey} com 45 desenvolvedores, descobrimos que além de maus cheiros tradicionais, o uso de algumas estruturas específicas da plataforma são amplamente percebidas como más práticas, portanto, possíveis maus cheiros específicos. Identificamos 23 más práticas específicos ao \textit{front-end} Android e validamos a percepção sobre quatro delas com 20 desenvolvedores. Descobrimos que fato há boas e más práticas específicas ao \textit{front-end} Android. Elaboramos um catálogo com 23, das quais investigamos a percepcão de desenvovlvedores sobre quatro e validamos com sucesso a percepção sobre duas.