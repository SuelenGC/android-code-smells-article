% -*- root: article.tex -*-
Cheiros de código são aliados na busca pela qualidade de código durante o desenvolvimento de software pois possibilitam a implementação de ferramentas de detecção automática de trechos de códigos problemáticos ou mesmo a inspeção manual. Apesar de já existirem diversos cheiros de código catalogados, pesquisas sugerem que tecnologias diferentes apresentaram cheiros de código específicos, e uma tecnologia que tem chamado a atenção de muitos pesquisadores é o Android. Neste artigo, nós investigamos a existência de cheiros de código em projetos Android. Por meio de um \textit{survey} com 45 desenvolvedores, descobrimos que além de cheiros de código tradicionais, o uso de algumas estruturas específicas da plataforma são amplamente percebidas como más práticas, portanto, possíveis cheiros de código específicos. Identificamos 23 más práticas específicos ao \textit{front-end} Android e validamos a percepção sobre quatro delas com 20 desenvolvedores. Descobrimos que fato há boas e más práticas específicas ao \textit{front-end} Android. Elaboramos um catálogo com 23, das quais investigamos a percepcão de desenvovlvedores sobre quatro e validamos com sucesso a percepção sobre duas.