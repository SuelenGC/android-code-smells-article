Android is the most widely used mobile operating system with 83\% of the world market and more than 2 million applications available in the official store. Android applications have been made complex software projects that need to be quickly developed and regularly evolved to meet the requirements of users. This context can lead to bad code design choices, known as code smells, that can become anomalies that degrade project quality, making it difficult to maintain. Consequently, software developers need to identify problematic code snippets in order to have a code base that favors maintenance and evolution. For this, developers often make use of techniques to detect code smells. Although there are already several code smells cataloged, such as God Class and Long Method, they do not take into account the nature of the project. Researches demonstrated that different platforms, languages and frameworks present specific code quality metrics. Android projects have specific characteristics, such as a directory that stores all the resources used and a class that tends to accumulate various responsibilities. Research on Android are still few. In this dissertation we aim to identify, validate and document code smells related with presentation layer of Android, where the greatest differences are found when compared to traditional projects. In other works on Android, were identified code smells related to safety, intelligent features or somehow impact the experience or user expectation consumption. Unlike them, our proposal is to catalog code smells Android that influence the quality of the code. With this, developers will have another resource for producing quality code.

\noindent \textbf{Keywords:} android, code smells, code quality, code anomalies, code metrics.