% -*- root: article.tex -*-
Cheiros de c\'odigo s\~ao aliados na busca pela qualidade de c\'odigo durante o desenvolvimento de software pois possibilitam a implementa\c{c}\~ao de ferramentas de detec\c{c}\~ao autom\'atica de trechos de c\'odigos problem\'aticos ou mesmo a inspe\c{c}\~ao manual. Apesar de j\'a existirem v\'arios cheiros de c\'odigo catalogados, pesquisas sugerem que tecnologias diferentes podem apresentar cheiros de c\'odigo espec\'ificos, e uma tecnologia que tem chamado a aten\c{c}\~ao de muitos pesquisadores \'e o Android. Neste artigo, n\'os investigamos a exist\^encia de cheiros de c\'odigo em projetos Android. Por meio de um \textit{survey} com 45 desenvolvedores descobrimos que al\'em de cheiros de c\'odigo j\'a mapeados, o uso de algumas estruturas espec\'ificas da plataforma s\~ao amplamente percebidas como m\'as pr\'aticas, portanto, poss\'iveis cheiros de c\'odigo espec\'ificos. Desta percep\c{c}\~ao prop\^omos tr\^es cheiros de c\'odigo Android, validados com um especialista e em um experimento com 30 desenvolvedores. Ao final, discutimos os resultados encontrados bem como pontos de melhoria e trabalhos futuros.