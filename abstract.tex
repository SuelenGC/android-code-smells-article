% -*- root: article.tex -*-
Cat\'alogos de maus cheiros s\~ao fortes aliados na busca pela qualidade de c\'odigo durante o desenvolvimento de software. Esses cat\'alogos possibilitam a implementa\c{c}\~ao de ferramentas de detec\c{c}\~ao autom\'atica de trechos de c\'odigos problem\'aticos ou mesmo a inspe\c{c}\~ao manual. Apesar de j\'a existirem diversos cat\'alogos de maus cheiros, pesquisas sugerem que tecnologias diferentes podem apresentar maus cheiros espec\'ificos. Neste artigo n\'os investigamos a exist�ncia de maus cheiros em projetos Android. N\'os conduzimos um \textit{survey} com 45 desenvolvedores android e descobrimos que al\'em de maus cheiros j\'a conhecidos, algumas implementa\c{c}\~oes espec\'ificas s\~ao amplamente percebidas como m\'as pr\'aticas, portanto, poss\'iveis maus cheiros espec\'ificos. Desta percep\c{c}\~ao prop�mos um cat\'alogo de maus cheiros android validado com um especialista e em um experimento com 30 desenvolvedores. Ao final discutimos os resultados encontrados bem como pontos de melhoria e trabalhos futuros.


