\documentclass[sigconf]{acmart}

\usepackage{booktabs} % For formal tables
\usepackage{blindtext} % Package to generate dummy text throughout this template 
\usepackage{lettrine} % The lettrine is the first enlarged letter at the beginning of the text

% Copyright
%\setcopyright{none}
%\setcopyright{acmcopyright}
%\setcopyright{acmlicensed}
\setcopyright{rightsretained}
%\setcopyright{usgov}
%\setcopyright{usgovmixed}
%\setcopyright{cagov}
%\setcopyright{cagovmixed}


% DOI
\acmDOI{10.475/123_4}

% ISBN
\acmISBN{123-4567-24-567/08/06}

%Conference
\acmConference[SBES'17]{SBES 2017: 31st Brazilian Symposium on Software Engineering}{Setembro 2017}{Fortaleza, Cear\'a Brasil} 
\acmYear{2017}
\copyrightyear{2017}

\acmPrice{15.00}


\begin{document}
\title{Cheiros de C\'odigo em projetos Android. Um estudo qualitativo sobre a percep\c{c}\~ao de desenvolvedores}
% \titlenote{Produces the permission block, and copyright information}
% \subtitle{Extended Abstract}
% \subtitlenote{The full version of the author's guide is available as
%   \texttt{acmart.pdf} document}


\author{Suelen G. Carvalho}
% \authornote{Dr.~Trovato insisted his name be first.}
% \orcid{1234-5678-9012}
\affiliation{
  \institution{Universidade de S\~ao Paulo}
  \streetaddress{Rua do Mat\~ao, 1010}
  \city{São Paulo} 
  \state{SP} 
  \postcode{05508-090}
}
\email{suelengc@ime.usp.br}

\author{Marco Aur\'elio Gerosa}
\affiliation{
  \institution{Universidade de S\~ao Paulo}
  \streetaddress{Rua do Mat\~ao, 1010}
  \city{S\~ao Paulo} 
  \state{SP} 
  \postcode{05508-090}
}
\email{gerosa@ime.usp.br}

\author{Maur\'icio Aniche}
\affiliation{
  \institution{Delft University of Technology}
  \streetaddress{Mekelweg 2}
  \city{Delft} 
  \state{Netherland} 
  \postcode{2628}
}
\email{m.f.aniche@tudelft.nl}

% The default list of authors is too long for headers}
\renewcommand{\shortauthors}{B. Trovato et al.}


\begin{abstract}
% -*- root: article.tex -*-
Cheiros de c\'odigo s\~ao aliados na busca pela qualidade de c\'odigo durante o desenvolvimento de software pois possibilitam a implementa\c{c}\~ao de ferramentas de detec\c{c}\~ao autom\'atica de trechos de c\'odigos problem\'aticos ou mesmo a inspe\c{c}\~ao manual. Apesar de j\'a existirem v\'arios cheiros de c\'odigo catalogados, pesquisas sugerem que tecnologias diferentes podem apresentar cheiros de c\'odigo espec\'ificos, e uma tecnologia que tem chamado a aten\c{c}\~ao de muitos pesquisadores \'e o Android. Neste artigo, n\'os investigamos a exist\^encia de cheiros de c\'odigo em projetos Android. Por meio de um \textit{survey} com 45 desenvolvedores descobrimos que al\'em de cheiros de c\'odigo j\'a mapeados, o uso de algumas estruturas espec\'ificas da plataforma s\~ao amplamente percebidas como m\'as pr\'aticas, portanto, poss\'iveis cheiros de c\'odigo espec\'ificos. Desta percep\c{c}\~ao prop\^omos tr\^es cheiros de c\'odigo Android, validados com um especialista e em um experimento com 30 desenvolvedores. Ao final, discutimos os resultados encontrados bem como pontos de melhoria e trabalhos futuros.
\end{abstract}

%
% The code below should be generated by the tool at
% http://dl.acm.org/ccs.cfm
% Please copy and paste the code instead of the example below. 
%
\begin{CCSXML}
<ccs2012>
 <concept>
  <concept_id>10010520.10010553.10010562</concept_id>
  <concept_desc>Computer systems organization~Embedded systems</concept_desc>
  <concept_significance>500</concept_significance>
 </concept>
 <concept>
  <concept_id>10010520.10010575.10010755</concept_id>
  <concept_desc>Computer systems organization~Redundancy</concept_desc>
  <concept_significance>300</concept_significance>
 </concept>
 <concept>
  <concept_id>10010520.10010553.10010554</concept_id>
  <concept_desc>Computer systems organization~Robotics</concept_desc>
  <concept_significance>100</concept_significance>
 </concept>
 <concept>
  <concept_id>10003033.10003083.10003095</concept_id>
  <concept_desc>Networks~Network reliability</concept_desc>
  <concept_significance>100</concept_significance>
 </concept>
</ccs2012>  
\end{CCSXML}

% \ccsdesc[500]{Computer systems organization~Embedded systems}
% \ccsdesc[300]{Computer systems organization~Redundancy}
% \ccsdesc{Computer systems organization~Robotics}
% \ccsdesc[100]{Networks~Network reliability}

% We no longer use \terms command
%\terms{Theory}

\keywords{android, cheiros de c\'odigo, qualidade de c\'odigo}


\maketitle

\section{Introdu\c{c}\~ao}
% -*- root: article.tex -*-
% \lettrine[nindent=0em,lines=3]{L}orem ipsim ult as 
Escrever c\'odigo com qualidade tem se tornado cada vez mais importante com o aumento da complexidade da tecnologia. Existem diferentes t\'ecnicas que auxiliam os desenvolvedores a escrever c\'odigo com qualidade incluindo \textit{design patterns}\mau{cite gof} e cheiros de c\'odigo\mau{cite beck e fowler}. A falta de qualidade pode resultar em defeitos de software que podem custar a empresas quantias significativas, especialmente quando conduzem a falhas de software \cite{Nagappan:2005, briand1993modeling}. Evolu\c{c}\~ao e manuten\c{c}\~ao de software tamb\'em j\'a se provaram como os maiores gastos com aplica\c{c}\~oes \cite{RefactoringAndImprovements:10}.

Uma das formas de aumentar a qualidade de software \'e identificar trechos de c\'odigos ruins e refator\'a-los, ou seja, alterar o c\'odigo sem alterar o comportamento \cite{Refactoring:99}. Com base nisso, temos que cheiros de c\'odigo desempenham um importante papel na busca por qualidade de c\'odigo, pois eles s\~ao sintomas que podem indicar problemas mais profundos no software, mas n\~ao necessariamente s\~ao o problema em si \cite{CodeSmell:06}. Seu mapeamento possibilita a defini\c{c}\~ao de heur\'isticas que, por sua vez, possibilitam a implementa\c{c}\~ao de ferramentas que os identificam de modo autom\'atico no c\'odigo. PMD \cite{PMD2016}, Checkstyle e FindBugs s\~ao exemplos de ferramentas que identificam automaticamente alguns tipos de cheiros de c\'odigo em c\'odigos Java.

Determinar o que \'e ou n\~ao um cheiros de c\'odigo \'e subjetivo e pode variar de acordo com tecnologia, desenvolvedor, metodologia de desenvolvimento dentre outros aspectos \cite{WikiCodeSmell}. Em particular, Aniche et al.~\cite{MvcSmells:16,aniche2016satt} mostraram que a arquitetura do software é um fator importante e que deve ser levado em conta ao analisar a qualidade de um sistema.   Alguns estudos t\^em buscado por cheiros de c\'odigo tradicionais em projetos Android. Por exemplo, Verloop \cite{MobileSmells:13} analisou se classes derivadas do SDK Android s\~ao mais ou menos propensas a cheiros de c\'odigo tradicionais do que classes puramente Java. Linares et al. \cite{DomainMatters} usaram o m\'etodo DECOR para realizar a detec\c{c}\~ao de 18 \textit{anti-patterns} orientado a objetos em aplicativos m\'oveis. Outros estudos identificaram cheiros de c\'odigo espec\'ificos Android relacionados ao consumo inteligente de recursos do dispositivo, como bateria e mem\'oria, usabilidade, dentre outros \cite{EnergyAndroidSmells, ReimannBrylski2013}. 

Nossa pesquisa complementa as anteriores no sentido de que tamb\'em buscamos cheiros de c\'odigo Android, e se difere delas pois estamos buscando cheiros de c\'odigo relacionados \`a qualidade do c\'odigo Android, ou seja, qualidade relacionada a c\'odigos espec\'ificos dessa plataforma. Por exemplo \textsc{Activities}\mau{plural de activity é activies, arrumei aqui, arrume nos outros lugares}, \textsc{Fragments} e \textsc{Adapters} s\~ao classes usadas na constru\c{c}\~ao de telas e \textsc{listeners} s\~ao respons\'aveis pelas intera\c{c}\~oes com os usu\'arios. Buscamos entender, por exemplo, \textit{``quais s\~ao as \textbf{boas e m\'as} pr\'aticas ao lidar com \textsc{Activitys}, \textsc{Fragments}, \textsc{Adapters} e \textsc{listeners}''} ou \textit{``quais s\~ao as \textbf{boas e m\'as} pr\'aticas para a constru\c{c}\~ao da interface visual''}. 

Optamos por focar em cheiros de código relacionados ao \textit{front-end} Android pois encontramos pesquisas com abordagem similar, por\'em relacionadas ao \textit{front-end} de tecnologias web \cite{CSSCodeSmell, BB, FinavaroAniche2016}. E enquanto que cheiros de c\'odigo em projetos Java j\'a foram extensivamente estudados \cite{Riel, Refactoring:99, Martin:2008:CCH:1388398}, o \textit{front-end} Android ainda carece de estudo e possui peculiaridades n\~ao encontradas em c\'odigo Java tradicional \cite{Mannan_Dig_Ahmed_Jensen_Abdullah_Almurshed}. Alguns exemplos dessas peculiaridas s\~ao o ciclo de vida de \textsc{Activitys} e \textsc{Fragments} e a cria\c{c}\~ao da interface visual que \'e feita atrav\'es de arquivos XML chamados de \textsc{layout resources}.

Para definirmos quais elementos representam o \textit{front-end} Android, fizemos uma extensa revis\~ao da documenta\c{c}\~ao oficial \cite{AndroidDeveloperSite2016} e chegamos nos seguintes itens: \textsc{Activitys}, \textsc{Fragments}, \textsc{Listeners}, \textsc{Adapters} e os recursos do aplicativo, que s\~ao arquivos XML ou imagens utilizados na interface visual como por exemplo \textsc{Drawables}, \textsc{Layouts}, \textsc{Styles} e \textsc{Colors}. Como existem muitos tipos de recursos do aplicativo \cite{AndroidResourcesOverview}, com o objetivo de limitar o tamanho do question\'ario e foco da pesquisa, selecionamos quatro: \textsc{Layout}, \textsc{Styles}, \textsc{String} e \textsc{Drawable}. Optamos por esses recursos pois os mesmos est\~ao presentes no template padr\~ao do Android Studio \cite{FirstApp2017}, IDE oficial para desenvolvimento de projetos da plataforma Android \cite{AndroidStudio}. 

Para obter os dados iniciais, publicamos um question\'ario online com perguntas sobre boas e m\'as pr\'aticas no desenvolvimento do \textit{front-end} Android. Para possibilitar respostas mais completas foram usadas perguntas dissertativas. Com isso, pretendemos responder as seguintes quest\~oes de pesquisa: \\

\textbf{RQ1} O que desenvolvedores consideram boas e m\'as pr\'aticas no desenvolvimento Android? \\

\textbf{RQ2} C\'odigos afetados por estas m\'as pr\'aticas s\~ao percebidos pelos desenvolvedores como problem\'aticos? \\

\mau{pode manter esse parágrafo aqui sim, quando estiver pronto} \textcolor{red}{Com os resultados obtidos foi poss\'ivel compilar um cat\'alogo com 21 Android cheiros de c\'odigo classificados em alta, m\'edia e baixa recorr\^encia de percep\c{c}\~ao. Esperamos com este cat\'alogo contribuir com as ideias iniciais para a defini\c{c}\~ao de heur\'isticas para a detec\c{c}\~ao sistematizada desses \textit{smells}.}

As contribuições deste trabalho são\mau{melhorar aqui}:

\begin{enumerate}

	\item Relação de maus cheiros de código na camada
	de \textit{front-end} em aplicações Android, derivada após
	entrevistas com \textcolor{red}{XX} desenvolvedores.

	\item Experimento controlado com \textcolor{red}{XX} 
	desenvolvedores demonstrando \textcolor{bla bla bla}.

	\item Roteiro de entrevista para que outros pesquisadores
	possam replicar nosso estudo.
\end{enumerate}

As se\c{c}\~oes seguintes deste artigo est\~ao organizadas da seguinte forma: na Se\c{c}\~ao 2 discutimos trabalhos relacionados e o estado da arte sobre Android e cheiros de c\'odigo. Na Se\c{c}\~ao 3 falamos sobre a metodologia de pesquisa utilizada. A Se\c{c}\~ao 4 apresenta os resultados obtidos. Na Se\c{c}\~ao 5 discutimos os resultados. Na Se\c{c}\~ao 6 tratamos das amea\c{c}as \`a validade do nosso estudo e ent\~ao, na Se\c{c}\~ao 7 conclu\'imos.


\section{Trabalhos Relacionados}
Muitas pesquisas t\^em sido realizadas sobre a plataforma android, muitas delas focam em vulnerabilidades [25, 43, 27, 28, 81, 87, 88], autentica\c{c}\~ao [29, 30, 82, 85] e testes [17, 40]. Diferentemente destas pesquisas, nossa pesquisa tem foco na percep\c{c}\~ao dos desenvolvedores sobre boas e m\'as pr\'aicas de desenvolvimento na plataforma android. A percep\c{c}\~ao desempenha um importante papel na defini\c{c}\~ao de code smells relacionados a uma tecnologia, visto que code smells possuem uma natureza subjetiva. Code smells desempenham um importante papel na busca por qualidade de c\'odigo, visto que, ap\'os mapeados code smells, podemos chegar a heur\'isticas para identific\'a-los e com estas heur\'isticas, implementar ferramentas que automatizem o processo de identificar c\'odigos maus cheirosos.

\section{Metodologia}
% -*- root: article.tex -*-
Conduzimos um estudo qualitativo e explorat\'orio onde os dados foram coletados atrav\'es de um question\'ario online com desenvoveldores Android. Esta se\c{c}\~ao descreve de forma detalhada a estrutura do question\'ario, os participantes e o an\'alise realizada sob as respostas do question\'ario.

\subsection{Question\'ario}
\label{sub:questionario}

O question\'ario continha 25 quest\~oes subdivididas em tr\^es se\c{c}\~oes: a primeira se\c{c}\~ao continha 6 perguntas demogr\'aficas, a segunda se\c{c}\~ao continha 16 perguntas sobre boas e m\'as pr\'aticas relacionadas ao \textit{front-end} Android e a terceira se\c{c}\~ao continha 3 perguntas, 2 para obter \'ultimos pensamentos sobre boas e m\'as pr\'aticas e 1 \'ultima solicitando email caso o participante tivesse interesse em etapas futuras da pesquisa. O question\'ario foi escrito em ingl\^es por\'em informava o participante que respostas em ingl\^es ou portugu\^es eram aceitas. Antes da divulga\c{c}\~ao, realizamos um piloto com 3 desenvolvedores Android. Todos estes dados est\~ao dispon\'iveis no nosso pacote de replica\c{c}\~ao\footnote{https://github.com/SuelenGC/android-code-smells-article}.

A primeira se\c{c}\~ao continha 6 quest\~oes demogr\'aficas obrigat\'orias de multipla escolha. Abordavam sobre idade (18 ou menos, 19 a 24, 25 a 34 e assim por diante at\'e 55 ou mais), estado de resid\^encia (foi dada uma lista com estados do Brasil, Estados Unidos e Europa), anos de experi\^encia com desenvolvimento de software, (1 anos ou menos, 2 anos, 3 anos, e assim por diante at\'e 10 ou mais), anos de experi\^encia com desenvolvimento Android (mesma escala de anos da quest\~ao anterior), uma quest\~ao sobre linguagens que o participante se considerava proeficiente (Java, Python, Ruby, Android, dentre outras) e sobre o \'ultimo grau de escolaridade (estudante de bacharelado, bacharelado, mestrado, doutorado). As quest\~oes sobre idade, regi\~ao, linguagens e grau de escolaridade continham a op\c{c}\~ao ``outros'' onde possibilitava que o participante inserisse sua resposta de forma manual.

A segunda se\c{c}\~ao continha 16 quest\~oes opcionais e dissertativas sobre boas e m\'as pr\'aticas relacionadas ao \textit{front-end} Android. Para cada elemento do \textit{front-end} Android foram feitas duas perguntas, uma sobre boas e outra sobre m\'as pr\'aticas percebidas pelos participantes. Por exemplo, para a \textsc{Activity} as perguntas foram:

\begin{itemize} 
	\item[$\textasteriskcentered$] Do you have any good practices to deal with Activities?
	\item[$\textasteriskcentered$] Do you have any bad practices to deal with Activities? 
\end{itemize}

A terceira se\c{c}\~ao continha 3 perguntas opcionais e dissertativas, 2 para captar qualquer \'ultima ideia sobre boas e m\'as pr\'aticas n\~ao captadas nas quest\~oes anteriores e 1 opcional onde solicitamos o email do participante caso o mesmo tivesse interesse em participar de etapas futuras da pesquisa. As perguntas sobre boas e m\'as pr\'aticas foram as a seguir: 

\begin{itemize} 
	\item[$\textasteriskcentered$] Are there any other *GOOD* practices in Android Presentation Layer we did not asked you or you did not said yet?
	\item[$\textasteriskcentered$] Are there any other *BAD* practices in Android Presentation Layer we did not asked you or you did not said yet?
\end{itemize}

Antes da divulga\c{c}\~ao, realizamos um piloto com 3 desenvolvedores Android e com o feedback deles fizemos alguns ajustes relacionados a obrigatoriedade das perguntas da segunda se\c{c}\~ao do question\'ario, onde todas tornaram-se opcionais. As respostas dos participantes piloto foram desconsideradas para efeitos de vi\'es. 

\subsection{Participantes}
\label{sub:participantes}

O question\'ario foi divulgado em redes sociais como Facebook, Twitter e Linkedin, em grupos de discuss\~ao sobre Android como \href{https://groups.google.com/forum/#!forum/androidbrasil-dev}{\textit{Android Dev Brasil}}, \href{https://groups.google.com/forum/\#!forum/android-brasil-projetos}{\textit{Android Brasil Projetos}} e o grupo do \href{http://slack.androiddevbr.org/}{\textit{Slack Android Dev Br}}, maior grupo de desenvolvedores Android do Brasil com 2622 participantes at\'e o momento da escrita deste artigo. 

O question\'ario esteve aberto por aproximadamente 4 meses, de 9 de Outubro de 2016 at\'e 18 de Janeiro de 2017. Recebemos um total de 45 respostas sendo que 41 foram coletadas em Outubro de 2016, 3 no come\c{c}o de Novembro de 2016 e 1 em Janeiro de 2017. Acreditamos que poucas e nenhuma respostas foram coletadas respectivamente, nos meses de Novembro e Dezembro, devido as festas comemorativas de fim de ano. 7 participantes n\~ao responderam nenhumas das questões da segunda e terceira se\c{c}\~ao, tornando suas respostas n\~ao \'uteis a pesquisa. 53\% preencheram seus emails se disponibilizando para futuras etapas da pesquisa, acreditamos que este alto percentual pode indicar um interesse leg\'itimo da comunidade de desenvolvedores Android pelo tema, refor\c{c}ando a relev\^ancia do estudo. 

A primeira se\c{c}\~ao do question\'ario continham quest\~oes demogr\'aficas. Com a an\'alise dessas quest\^oes foi poss\'ivel notar que atingimos com sucesso \textit{desenvolvedores Android com variados n\'iveis de experi\^encia e de diversas regi\~oes}, pois 1) 100\% dos participantes indicaram possuir alguma experi\^encia com desenvolvimento Android, 2) menos de 30\% indicaram possuir 2 anos ou menos de experi\^encia com Android e mais de 70\% indicaram 3 anos ou mais (13\% 4 anos, 6,5\% 5 anos, 15,5\% 6 anos, 4,4\% 7 anos e 4,4\% 8 anos), vale considerar que a plarforma Android completa 10 anos em 2017, ou seja, 5 anos representam 50\% do tempo desta tecnolgia 4) 36 respostas foram do Brasil, 7 de pa\'ises europeus e 1 dos Estados Unidos (Calif\'ornia). Estes dados est\~ao sumarizados na Tabela \ref{tab:DadosDemograficos}.

\begin{table}[h]
\centering
\caption{Experi\^encia dos participantes com desenvolvimento Android.}
\begin{tabular}{c|c}
Experi\^encia com Android & Total de Participantes \\
\hline
at\'e 1 ano 	&	6 (13,33\%) \\
2 anos 			& 	 7 (15,56\%) \\
3 anos 			& 	 12 (26,67\%) \\
4 anos 			& 	 6 (13,33\%) \\
5 anos 			& 	 3 (6,67\%) \\
6 anos 			& 	 7 (15,56\%) \\
7 anos 			& 	 2 (4,44\%) \\
8 anos 			& 	 2 (4,44\%) \\

\end{tabular}
\label{tab:DadosDemograficos}
\end{table}

% \begin{tikzpicture}
% \begin{axis}[
%     ybar,
%     enlargelimits=0.15,
%     legend style={at={(0.5,-0.15)},
%       anchor=north,legend columns=-1},
%     ylabel={participantes},
%     symbolic x coords={1-2 anos,2-5 anos,6-10 anos},
%     xtick=data,
%     nodes near coords,
%     nodes near coords align={vertical},
%     ]
% \addplot coordinates {(1-2 anos,5) (2-5 anos,10) (6-10 anos,30)};
% \addplot coordinates {(1-2 anos,13) (2-5 anos,21) (6-10 anos,11)};
% \legend{experi\^encia com software,experi\^encia com Android}
% \end{axis}
% \end{tikzpicture}

\subsection{An\'alise dos Dados}
\label{sub:smells-definition}

O processo de análise das respostas do questionário se deu em 3 passos: verticaliza\c{c}\~ao, limpeza dos dados e codifica\c{c}\~ao t\'opica e divis\~ao. A an\'alise partiu da listagem das 45 respostas do question\'ario. A partir desta listagem realizamos o processo que denominamos como verticaliza\c{c}\~ao, ou seja, cada resposta de boa ou m\'a pr\'atica se tornou um registro individual a ser analisado, resultando em 810 respostas sobre boas ou m\'as pr\'aticas em algum elemento do \textit{front-end} Android. O n\'umero 810 refere-se as 18 perguntas sobre boas e m\'as pr\'aticas multiplicado pelos 45 participantes.

Nosso segundo foi realizar a limpeza dos dados. Essa etapa consistiu em remover respostas obviamente n\~ao \'uteis como respostas em branco, que continham frases como \textit{"N\~ao"}, \textit{"N\~ao que eu saiba"}, \textit{"Eu n\~ao me lembro"} e similares, as consideradas vagas como \textit{"Eu n\~ao tenho certeza se s\~ao boas praticas mas uso o que vejo por ai"}, as consideradas gen\'ericas como \textit{"Como todo c\'odigo java..."} e as que n\~ao eram relacionadas a boas pr\'aticas de c\'odigo. Das 810 boas e m\'as pr\'aticas, 352 foram consideradas e 458 desconsideradas. Das 352, 44,6\% foram apontadas como m\'as pr\'aticas e 55,4\% como boas pr\'aticas. 

Em seguida realizamos a codifica\c{c}\~ao sobre as boas e más práticas. Durante esse processo, categorias foram emergindo a partir das respostas. Durante a codifica\c{c}\~ao, houveram 30 respostas que n\~ao eram triviais de identificar uma categoria ou mesmo de dizer se essas respostas deveriam ser consideradas, sendo marcadas como \textit{"talvez"} durante o processo e reavaliadas ao final, onde 6 permaneceram e 24 foram desconsideradas. Uma situa\c{c}\~ao interessante \'e que diversas dessas respostas indicavam que n\~ao se deve usar \textsc{Fragments} por\'em n\~ao apresentavam nenhum argumento sobre o motivo, por exemplo: ``Eu tento evit\'a-los.' (tradu\c{c}\~ao livre). Essas respostas inicialmente seriam desconsideradas, mas pela quantidade de repeti\c{c}\~oes obtidas, 10 no caso, optamos por considerar. Tamb\'em durante a codifica\c{c}\~ao, 9 respostas incialmente consideradas, foram desconsideradas. Para toda resposta desconsiderada foi indicado um motivo. Ao final da categoriza\c{c}\~ao, 313 boas e m\'as pr\'aticas foram de fato consideradas.

% Em seguida, realizamos diversas itera\c{c}\~oes nas respostas sobre boas e m\'as pr\'aticas consideradas a fim de categoriz\'a-las em algum novo smell Android ou algum smell pr\'e-existente. Essas itera\c{c}\~oes consistiram em analisar resposta a resposta e atribuir uma ou mais categorias de algum poss\'ivel novo smell Android ou pr\'e-existente. Foram realizadas diversas itera\c{c}\~oes de categoriza\c{c}\~ao com o objetivo de normalizar as categorias, ou seja, evitar sin\^onimos e hom\^onimos. Um sin\^onimo \'e o mesmo conceito com dois nomes diferentes e hom\^onimos s\~ao dois conceitos diferentes com o mesmo nome. 

% Durante a categoriza\c{c}\~ao, uma mec\^anica que consideramos importante para a normaliza\c{c}\~ao das categorias, foi a cria\c{c}\~ao de uma lista de categorias, onde a cada nova categoria atribu\'ida, increment\'avamos a lista e preench\'iamos com descri\c{c}\~oes que indicavam que tipo de boa ou m\'a pr\'atica estava recebendo aquela categoria. Esta mec\^anica ajudou a evitar hom\^onimos e sin\^onimos e serviu como base para a defini\c{c}\~ao e avalia\c{c}\~ao da relev\^ancia dos \textit{smells} a serem trabalhados nos pr\'oximos passos.

Nosso último passo foi a etapa de divis\~ao, ou seja, as respostas que receberam mais de uma categoria foram divididas em duas ou mais respostas, de acordo com o n\'umero de categorias identificadas. Por exemplo, a resposta \textit{"N\~ao fazer Activities serem callbacks de execu\c{c}\~oes ass\'incronas. Herdar sempre das classes fornecidas pelas bibliotecas de suporte, nunca diretamente da plataforma"} indica na primeira ora\c{c}\~ao a categoria de smell que denominamos \textbf{Zumbi Referended Activity} e na segunda ora\c{c}\~ao, a categoria de smell \textbf{Inherit From Support Library Always}. Ao divid\'i-la mantivemos o texto da resposta apenas relativo a categoria, como se fossem duas respostas v\'alidas. Em algumas respostas que foram divididas n\~ao pudemos dividir o texto pois a resposta completa era necess\'ario para entender ambas as categoriza\c{c}\~oes, nesses casos, mantivemos a resposta original, mesmo que duplicada, e categorizamos cada uma de forma diferente. Ap\'os estas divis\~oes, as 313 respostas iniciais se tornaram 388 sendo cada uma com apenas uma categoria de \textit{smell}. 

Ao final de todas as etapas, conclu\'imos com 388 respostas sobre boas e m\'as pr\'aticas distribu\'idas em 47 categorias. 

\subsection{An\'alise e Defini\c{c}\~ao dos Smells}
\label{sub:analisys-definition}

Nosso objetivo nesta etapa foi entender quais \textit{smells}, dos 47 identificados, eram mais recorr\^entes. Para isso, contabilizamos cada \textit{smell}, em quantas respostas ele aparecia, ou seja, se duas respostas foram categorizadas com o \textit{smell} "A" ent\~ao diz-se que esse \textit{smell} tem contagem 2. Depois da contagem, elaboramos intervalos que indicam recorr\^encia ALTA, maior ou igual a 20, M\'EDIA, dentre 6 e 19 e BAIXA, dentre 3 e 5. Abaixo de 3 classificamos como IRRELEVANTE. Obtemos o seguinte resultado:

\begin{table}[h]
\centering
\caption{Smells identificados vs. recorr\^encia percebida pelos participantes do question\'ario}
\begin{tabular}{p{3cm}|p{4cm}}
\textbf{Recorr\^encia} & \textbf{Quantidade de Smells} \\
\hline
Alta  			& 5 \\
M\'edia 		& 17 (1 \textit{smell} tradicional) \\
Baixa			& 5 \\
Irrelevante 	& 20 \\
\end{tabular}
\label{tab:DadosDemograficos2}
\end{table}

Os irrelevantes foram desconsiderados nesta etapa de defini\c{c}\~ao. Os demais foram definidos com a ajuda das respostas dos participantes. Para cada \textit{smell} definimos os seguintes t\'opicos: 

\begin{itemize} 
	\item[$\textasteriskcentered$] \textbf{Quando ocorre}. Indicamos os motivos pelo qual foi considerado uma m\'a pr\'atica.   
	\item[$\textasteriskcentered$] \textbf{Contexto/exemplo}. Indicamos algum exemplo ou contexto pr\'atico. 
	\item[$\textasteriskcentered$] \textbf{Elementos afetados}. Eventualmente algum \textit{smell} afeta mais de um elemento, nesse t\'opico abordamos em quais os elementos este \textit{smell} pode estar presente. 
	\item[$\textasteriskcentered$] \textbf{Solu\c{c}\~ao}. Indicamos poss\'iveis refatora\c{c}\~oes para reduzir ou eliminar o \textit{smell}. \\
\end{itemize}

\'E importante ressaltar que todas as defini\c{c}\~oes dos \textit{smells} foram apoiadas nas respostas dos participantes. Se os participantes n\~ao indicaram algum t\'opico para algum \textit{smell}, o mesmo n\~ao foi definido. \\


\section{Resultados}
% -*- root: article.tex -*-
% \lettrine[nindent=0em,lines=3]{L}orem ipsim ult assh
\subsection{Interpreta\c{c}\~ao dos Dados}

As vezes o que era indicado como boa pr\'atica para um elemento era um smell percebido em outro elemento, por exemplo, a R1 diz que \textit{"Sempre que noto ter mais de um [layout] resource usando o mesmo estilo eu tento mov\^e-lo [o estilo] para meu recurso de estilo"} (tradu\c{c}\~ao livre) ao responder sobre boa pr\'atica para o style resource, por\'em esta resposta foi considerada para definir o smell Duplicated Styles Attributes que \'e percebido em recursos de layout ou styles.


\subsection{Afirma\c{c}\~ao sobre o \textit{front-end} Android}

Uma opini\~ao que foi un\^anime em muitas respostas foi que de fato, desenvolvedores tratam \textsc{Activitys}, \textsc{Fragments} e \textsc{Adapters} como elementos do front-end Android, conforme constatamos na se\c{c}\~ao 3.1. Isso pode ser observado diversas vezes com respostas por exemplo, P25 indicou como boa pr\'atica na Activity "Nenhuma l\'ogica [de neg\'ocio] aqui" (tradu\c{c}\~ao livre), o P40 afirma sobre m\'a pr\'atica em adapter \'e \textit{"L\'ogica de neg\'ocio em adapters \'e n\~ao-n\~ao"} (tradu\c{c}\~ao livre), ao falarem sobre fragments, muitos indicaram \textit{"O mesmo da Activity"}. Ou seja, primeiramente estas respostas refor\c{c}am nossa defini\c{c}\~ao inicial sobre elementos que compo\^em o front-end Android, e por outro lado, vimos que muitas vezes fragments s\~ao tratados como Activitys, ao se falar de boas e m\'as pr\'aticas de c\'odigo.



\section{Discuss\~ao}
% -*- root: article.tex -*-
% \lettrine[nindent=0em,lines=2]{L}
\blindtext % Dummy text

\blindtext % Dummy text

A text reference \cite{L}.

\blindtext % Dummy text



\section{Trabalhos Futuros}
% -*- root: article.tex -*-
% \lettrine[nindent=0em,lines=2]{L}
\blindtext % Dummy text

\blindtext % Dummy text

A text reference \cite{L}.

\blindtext % Dummy text



\section{Conclus\~ao}
% -*- root: article.tex -*-
% \lettrine[nindent=0em,lines=2]{L}
\blindtext % Dummy text

\blindtext % Dummy text

A text reference \cite{L}.

\blindtext % Dummy text



% \input{samplebody-conf}

\bibliographystyle{ACM-Reference-Format}
\bibliography{sigproc,dissertation} 

\end{document}
