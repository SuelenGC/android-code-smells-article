% -*- root: article.tex -*-
% \lettrine[nindent=0em,lines=3]{L}orem ipsim ult as 
Escrever c\'odigo com qualidade tem se tornado cada vez mais importante com o advento da tecnologia. Existem diferentes t\'ecnicas que auxiliam os desenvolvedores a escreverem c\'odigo com qualidade incluindo \textit{design patterns} e \textit{code smells}. Defeitos de software, ou \textit{bugs}, podem custar a empresas quantias significativas de dinheiro, especialmente quando conduzem a falhas de software \cite{Nagappan:2005, briand1993modeling}. Evolu\c{c}\~ao e manuten\c{c}\~ao de software tamb\'em j\'a se provaram como os maiores gastos com aplica\c{c}\~oes \cite{RefactoringAndImprovements:10}.

\textit{Code smells} desempenham um importante papel na busca por qualidade de c\'odigo. Seu mapeamento possibilita a defini\c{c}\~ao de heur\'isticas que por sua vez, possibilitam a implementa\c{c}\~ao de ferramentas que os identificam de forma autom\'atica no c\'odigo. PMD, Checkstyle e FindBugs por exemplo, s\~ao ferramentas que identificam automaticamente alguns tipos de \textit{code smells} em c\'odigos Java.

Determinar o que \'e ou n\~ao um \textit{code smells} \'e subjetivo e pode variar de acordo com tecnologia, desenvolvedor, metologia de desenvolvimento dentre outros aspectos []. Alguns estudos t\^em buscado por \textit{code smells} tradicionais em projetos android, por exemplo Verloop \cite{MobileSmells:13} analisou se classes derivadas do SDK android eram mais ou menos propensas a code smells tradicionais do que classes puramente Java. Linares et al. \cite{DomainMatters} usaram o m\'etodo DECOR para realizar a detec\c{c}\~ao de 18 \textit{anti-patterns} orientado a objetos em aplicativos m\'oveis. Outros estudos identificaram \textit{code smells} espec\'ificos android porém relacionados ao consumo intelig\^ente de recursos do dispositivo como bateria e mem\'oria, usabilidade, dentre outros \cite{EnergyAndroidSmells, ReimannBrylski2013}. Nossa pesquisa complementa as anteriores no sentido de que tamb\'em vamos buscar \textit{code smells} android, e se difere delas pois estamos buscando \textit{smells} relacionados a qualidade do c\'odigo android no sentido de responder quest\~oes como: quais s\~ao as m\'as pr\'aticas ao lidar com \texttt{android resources}, ou ao lidar com \texttt{activity}s, \texttt{fragment}s, \texttt{adapter}s e \texttt{listener}s. Nossa pesquisa objetiva definir \textit{code smells} android baseado em o que desenvolvedores desta plataforma percebem como boas e m\'as pr\'aticas em elementos espec\'ificos da plataforma.

Nas se\c{c}\~oes seguintes deste artigo, discutiremos primeiro alguns trabalhos relacionados (Se\c{c}\~ao 2) e os m\'etodos utilizados em nosso estudo (Se\c{c}\~ao 3). A Se\c{c}\~ao 4 apresenta os resultados e as amea\c{c}as \`a validade do nosso estudo. Na Se\c{c}\~ao 5 discutimos e conclu\'imos... e terminamos com uma discuss\~ao de trabalhos futuros (Se\c{c}\~ao 6).
