% -*- root: article.tex -*-
% \lettrine[nindent=0em,lines=3]{L}orem ipsim ult as 
Escrever código com qualidade tem se tornado cada vez mais importante com o aumento da complexidade de tecnologias e anseio dos usuários por novas funcionalidade e atualizações \cite{Hecht2015,MobileSmells:13}. Existem diferentes técnicas que auxiliam os desenvolvedores a escreverem código com qualidade, incluindo \textit{design patterns} \cite{gof} e cheiros de código \cite{Refactoring:99}. A falta de qualidade pode resultar em defeitos de software que podem custar a empresas quantias significativas, especialmente quando conduzem a falhas de software \cite{Nagappan:2005, briand1993modeling}. Evolução e manutenção de software também já se provaram como os maiores gastos com aplicações \cite{RefactoringAndImprovements:10}.

Uma das formas de aumentar a qualidade de software é identificar trechos de códigos ruins e refatorá-los, ou seja, alterar o código sem alterar o comportamento \cite{Refactoring:99}. Desta forma, temos que cheiros de código são aliados importantes na busca por qualidade de código pois, representam sintomas que podem indicar problemas mais profundos no software, não necessariamente, sendo o problema em si \cite{CodeSmell:06}. Seu mapeamento possibilita a definição de heurísticas que, por sua vez, possibilitam a implementação de ferramentas que os identificam de modo automático no código. PMD \cite{PMD2016}, Checkstyle e FindBugs são exemplos de ferramentas que identificam automaticamente alguns tipos de cheiros de código em projetos Java.

Determinar o que é ou não um cheiros de código é subjetivo e pode variar de acordo com a tecnologia, desenvolvedor, metodologia de desenvolvimento dentre outros aspectos \cite{WikiCodeSmell}. Em particular, Aniche et al. \cite{MvcSmells:16,aniche2016satt} mostraram que a arquitetura do software é um fator importante e que deve ser levada em conta ao analisar a qualidade de um sistema. Alguns estudos têm buscado por cheiros de código tradicionais em projetos Android. Por exemplo, Verloop \cite{MobileSmells:13} analisou se classes derivadas do SDK Android são mais ou menos propensas a cheiros de código tradicionais do que classes puramente Java. Linares et al. \cite{DomainMatters} usaram o método DECOR para realizar a detecção de 18 \textit{anti-patterns} orientado a objetos em aplicativos móveis. Outros estudos identificaram cheiros de código específicos Android relacionados ao consumo inteligente de recursos do dispositivo, como bateria e memória, usabilidade, dentre outros \cite{EnergyAndroidSmells, ReimannBrylski2013}. 

Nossa pesquisa complementa as anteriores no sentido de que também buscamos por cheiros de código Android. E se difere delas pois buscamos cheiros de código relacionados à qualidade, em termos de manutenabilidade e legibilidade, específico dessa plataforma. Por exemplo \textsc{Activities}, \textsc{Fragments} e \textsc{Adapters} são classes usadas na construção de telas e \textsc{listeners} são responsáveis pelas interações com os usuários. Buscamos entender, por exemplo, \textit{``quais são as \textbf{boas e más} práticas ao lidar com \textsc{Activities}, \textsc{Fragments}, \textsc{Adapters} e \textsc{listeners}?''} ou \textit{``quais são as \emph{boas e más} práticas no desenvolvimento da interface visual Android?''}. 

Apesar dos trabalhos citados, Umme et al. \cite{Mannan_Dig_Ahmed_Jensen_Abdullah_Almurshed} recentemente levantaram que, das principais conferências de manutenção de software (ICSE, FSE, OOPSLA/SPLASH, ASE, ICSM/ICSME, MRS e ESEM), dentre 2008 a 2015, apenas 9,6\% dos artigos consideraram em suas pesquisas, projetos Android. Nenhuma outra plataforma móvel foi considerada. Enquanto que cheiros de código em projetos Java já foram extensivamente estudados \cite{Riel, Refactoring:99, Martin:2008:CCH:1388398}, ainda há muito a se pesquisar sobre cheiros de código em projetos Android.

Para limitar nosso objeto de estudo, optamos por focar em cheiros de código relacionados ao \textit{front-end} Android pois encontramos pesquisas com abordagem similar, porém relacionadas ao \textit{front-end} de tecnologias web, como CSS \cite{CSSCodeSmell}, Javascript \cite{BB} e o arcabouço Spring MVC \cite{FinavaroAniche2016}. 

% E enquanto que cheiros de código em projetos Java já foram extensivamente estudados \cite{Riel, Refactoring:99, Martin:2008:CCH:1388398}, o \textit{front-end} Android ainda carece de estudo e possui peculiaridades não encontradas em código Java tradicional \cite{Mannan_Dig_Ahmed_Jensen_Abdullah_Almurshed}. Alguns exemplos dessas peculiaridas são o ciclo de vida de \textsc{Activities} e \textsc{Fragments} e a criação da interface visual que é feita através de arquivos XML chamados de \textsc{layout resources}.

Para definirmos quais elementos representam o \textit{front-end} Android, fizemos umas extensa revisão da documentação oficial e chegamos nos seguintes itens: \textsc{Activities}, \textsc{Fragments}, \textsc{Listeners}, \textsc{Adapters} e os recursos do aplicativo, que são arquivos XML ou imagens utilizados na interface visual como por exemplo \textsc{Drawables}, \textsc{Layouts}, \textsc{Styles} e \textsc{Colors}. Como existem muitos tipos de recursos do aplicativo \cite{AndroidResourcesOverview}, com o objetivo de limitar o tamanho do questionário e foco da pesquisa, selecionamos quatro: \textsc{Layout}, \textsc{Styles}, \textsc{String} e \textsc{Drawable}. Optamos por esses recursos pois os mesmos estão presentes no template padrão do Android Studio \cite{FirstApp2017}, IDE oficial para desenvolvimento de projetos da plataforma Android \cite{AndroidStudio}. 

Os dados iniciais foram obtidos através de um questionário online com perguntas sobre boas e más práticas no desenvolvimento do \textit{front-end} Android. As análise do questionário resultou num com a 23 más práticas no desenvolvimento do \textit{front-end} Android. Validamoss a percepção dos desenvolvedores sobre essas más práticas através de outro questionário online. Com isso, pretendemos responder as seguintes questões de pesquisa: \\

\textbf{QP1} O que desenvolvedores consideram boas e más práticas no desenvolvimento Android? \\

\textbf{QP2} Códigos afetados por estas más práticas são percebidos pelos desenvolvedores como problemáticos? \\

Esperamos que este catálogo de más práticas possa contribuir com ideias iniciais para a definição de cheiros de código e heurísticas para a detecção sistematizada das mesmas em projetos Android, além de contribuir com sugestões de como mitigá-las.

As contribuições deste trabalho são: \\

\textbf{1)} Catálogo com 23 más práticas e sugestões de soluções no desenvolvimento do \textit{front-end} Android, derivadas a partir dos resultados obtidos com a aplicação de um questionário online respondido por 45 desenvolvedores. \\

\textbf{2)} A percepção de desenvolvedores sobre as quatro más práticas mais recorrentes através de um questionário online com 11 desenvolvedores. \\

\textbf{3)} Apêndice online \footnote{Apêndice online: http://suelengc.com/android-code-smells-article} com roteiros dos quetionários e outras informações da pesquisa para que outros pesquisadores possam replicar nosso estudo. \\


% \begin{enumerate}
% 	\item Catálogo com 23 más práticas no desenvolvimento do \textit{front-end} Android, derivadas a partir dos resultados obtidos com a aplicação de um questionário online respondido por 45 desenvolvedores.
% 	\item A percepção de desenvolvedores sobre as quatro más práticas mais recorrentes através de um questionário online com 11 desenvolvedores..
% 	\item Apêndice online \footnote{Apêndice online: http://suelengc.com/android-code-smells-article} com roteiros dos quetionários e outras informações da pesquisa para que outros pesquisadores possam replicar nosso estudo.
% \end{enumerate}

As seções seguintes deste artigo estão organizadas da seguinte forma: a Seção \ref{metodologia} aborda a metodologia de pesquisa. A Seção \ref{resultados} apresenta os resultados. A Seção \ref{discussao} discute pontos relevantes. A Seção \ref{ameacas} aborda as ameaças à validade do estudo. A Seção \ref{relacionados} discute os trabalhos relacionados e o estado da arte sobre Android e cheiros de código. E por fim, a Seção \ref{conclusao} conclui.
