% -*- root: article.tex -*-
Neste artigo investigamos a existência de boas e más práticas no \textit{front-end} de projetos Android. Fizemos isso através de um estudo exploratório qualitativo com 45 desenvolvedores, onde mapeamos 23 más práticas Android. Após, validamos a percepção de desenvolvedores Android sobre as quatro más práticas mais recorrêntes. Fizemos isso através de um experimento online respondido com 20 desenvolvedores Android. Respondemos a \textbf{QP1} com um catálogo com 23 más práticas no \textit{front-end} Android. Respondemos a \textbf{QP2} através da validação com sucesso da percepção de desenvolvedores sobre 2 das más práticas de alta recorrência.

% Neste artigo investigamos a existência de boas e más práticas em elementos usados para implementação de \textit{front-end} de projetos Android: \textsc{Activities}, \textsc{Fragments}, \textsc{Listeners}, \textsc{Adapters}, \textsc{Layout}, \textsc{Styles}, \textsc{String} e \textsc{Drawable}. Fizemos isso através de um estudo exploratório qualitativo onde coletamos dados por meio de um questionário online respondido por 45 desenvolvedores Android. A partir deste questionário mapeamos 23 más práticas e sugestões de solução, quando mencionado por algum participante. Após, validamos a percepção de desenvolvedores Android sobre as quatro mais recorrêntes dessas más práticas: \textsc{Lógica em Classes de UI}, \textsc{Nome de Recurso Despadronizado}, \textsc{Recurso Mágico e Layout Profundamente Aninhado}. Fizemos isso através de um experimento online respondido por 20 desenvolvedores Android, onde os participantes eram convidados a avaliar 6 códigos com relação a qualidade. \\

% \textbf{QP1.} \textbf{O que desenvolvedores consideram boas e más práticas no desenvolvimento Android?} 

% Entendemos que sim pois, ao questionar os 45 desenvolvedores sobre o que eles consideravam como boas e más práticas em elementos específicos do Android, conseguimos consolidar um catálogo com 23 más práticas onde, para cada uma delas apresentamos uma descrição textual e exemplos de frases usadas nas respostas que nos levaram a sua definição. \\

% \textbf{QP2.} \textbf{Códigos afetados por estas más práticas são percebidos pelos desenvolvedores como problemáticos?}

% Nós validamos a percepção de desenvolvedores sobre apenas quatro das 23 más práticas definidas. Essas foram selecionadas por serem as com maior recorrência. Neste estudo concluímos que códigos afetados por duas das más práticas, \textsc{Lógica em Classes de UI} e \textsc{Layout Profundamente Aninhado}, são percebidos por desenvolvedores como problemáticos. As outras duas, \textsc{Nome de Recurso Despadronizado} e \textsc{Recurso Mágico}, não obtivemos dados suficientes para afirmar que são percebidas ou não.




