% -*- root: article.tex -*-
Nesta pesquisa objetivamos investigar a percep\c{c}\~ao de desenvolvedores android sobre boas e m\'as pr\'aticas de c\'odigo em projetos android. Desta forma, pretendemos responder as seguintes quest\~oes de pesquisa: \\

\textbf{RQ1} O que desenvolvedores consideram boas e m\'as pr\'aticas no desenvolvimento android? \\

\textbf{RQ2} Poderiam estas m\'as pr\'aticas serem consideradas \textit{code smells} android? \\

\textbf{RQ3} \textit{Code smells} tradicionais s\~ao percebidos em c\'odigo android? \\

Nossos resultados nos permitiram definir de forma textual um cat\'alogo com 21 android \textit{code smells}. Este cat\'alogo ir\'a contribuir como base para i) a defini\c{c}\~ao de heur\'isticas para detec\c{c}\~ao ii) automatiza\c{c}\~ao de ferramentas de detec\c{c}\~ao autom\'atica desses smells.

Para obter as ideias iniciais para esta pesquisa, publicamos um question\'ario online sobre boas e m\'as pr\'aticas em determinados elementos da plataforma android. Perguntas dissertativas foram usadas para permitir respostas completas e n\~ao submeter o participante a nenhum vi\'es. O survey foi escrito em ingl\^es por\'em informava o participante que respostas em ingl\^es ou portugu\^es eram aceitas. Antes da divulga\c{c}\~ao, fizemos um piloto com 3 desenvolvedores android, com o feedback deles fizemos alguns ajustes relacionados a obrigatoriedade de algumas perguntas. As respostas do piloto foram desconsideradas para reduzir vi\'es. 

O survey foi dividido em 3 sess\~oes. A primeira continham perguntas para mapeamento demogr\'afico. A segunda continham perguntas relacionadas a boas e m\'as pr\'aticas em elementos de projetos android. A terceira e \'ultima sess\~ao continha perguntas para obter alguma ideia que n\~ao fio coletada na segunda sess\~ao al\'em do email do participante caso o mesmo se disponibilizasse a participar de outras etapa da pesquisa. 

A seguir, a sess\~ao 3.1 abroda sobre a elabora\c{c}\~ao do question\'ario, na 3.2 falamos sobre os participantes e a sess\~ao 3.3 aborda a an\'alise para a defini\c{c}\~ao dos smells.

\subsection{Question\'ario}

Android utiliza a linguagem Java para desenvolvimento de aplicativos. Ent\~ao a primeira quest\~ao \'e: por que buscar por smells android sendo que j\'a existem tantos para Java? A resposta para esta pergunta \'e que projetos android possuem caracter\'isticas diferentes de projetos java [] e pesquisas veem mostrando que tecnologias diferentes podem apresentar \textit{code smells} espec\'ificos, como por exemplo Aniche et al. identificou 6 \textit{code smells} espec\'ificos ao framework Spring MVC, um framework Java para desenvolvimento web. 

Tamb\'em notamos que diversas pesquisas sobre \textit{code smells} veem sendo feitas sobre tecnologias usadas no desenvolvimento de front-end web como CSS [] e Javascript []. Estas pesquisas nos deram motiva\c{c}\~ao para buscas entender se existem \textit{code smells} no front-end android, pois \'e nesta parte em que projetos android mais se distinguem de projetos Java pois seu front-end \'e composto de c\'odigo java mesclado com diversos recursos que podem ser arquivos xml e imagens [].

Para definirmos quais seriam os elementos usados para desenvolver a camada de apresenta\c{c}\~ao do android fizemos uma extens\~ao revis\~ao da documenta\c{c}\~ao oficial [] e chegamos a seguinte lista:

\begin{itemize} 
	\item[$\circ$] Classes do tipo \textsc{Activity};
	\item[$\circ$] Classes do tipo \textsc{Fragment};
	\item[$\circ$] Classes do tipo \textsc{Listener};
	\item[$\circ$] Classes do tipo \textsc{Adapter};
	\item[$\circ$] e recursos do aplicativo com \textsc{drawbles}, \textsc{layout}, \textsc{styles}, \textsc{color}, dentre outros.
\end{itemize}

Como existem muitos tipos de recursos [], com o objetivo de limitar o tamanho do question\'ario, selecionamos os quatro recursos mais utilizados: layout, styles, string e drawable.

A segunda sess\~ao do question\'ario abordou perguntas sobre boas e m\'as pr\'aticas sobre cada um dos elementos supra-citados. Baseamos nossa estrutura de perguntas usada por Aniche et al. ao pesquisar sobre smells no framework Spring MVC. Por exemplo, para Activitys fizemos as seguintes perguntas: \\

\begin{itemize} 
	\item[$\circ$] Do you have any good practices to deal with Activities?
	\item[$\circ$] Do you have any bad practices to deal with Activities? 
\end{itemize}

A terceira sess\~ao objetivou captar qualquer \'ultima ideia sobre boas e m\'as pr\'aticas n\~ao captadas nas quest\~oes anteriores. Para isso fizemos as seguintes perguntas: \\

\begin{itemize} 
	\item[$\circ$] Are there any other *GOOD* practices in Android Presentation Layer we did not asked you or you did not said yet?
	\item[$\circ$] Are there any other *BAD* practices in Android Presentation Layer we did not asked you or you did not said yet?
\end{itemize}


\subsection{Participantes}

Com o question\'ario obtivemos 45 respostas. O survey foi divulgado em grupos de desenvolvedores android como por exemplo o Slack AndroidDevBR, maior grupo de desenvolvedores android do Brasil contando hoje com mais de 2500 participantes. Coletamos 36 respostas do Brasil, 7 de pa\'ises europeus e 1 dos Estados Unidos, 1 participantes preferiu n\~ao responder. A tabela \ref{tab:DadosDemograficos} apresenta dados sobre anos de experi\^encia dos participantes com desenvolvimento de sofware e com desenvolvimento android. Notamos que 67\% dos participantes possuem mais de 5 anos de experi\^encia com desenvolvimento de software e que 71\% possuem 3 anos ou mais de experi\^encia com desenvolvimento android.

\begin{table}[h]
\centering
\caption{Anos de experi\^encia com desenvolvimento de software e desenvolvimento android dos participantes.}
\begin{tabular}{c|p{2cm}p{2cm}}
& Experi\^encia com Software & Experi\^encia com Android \\
\hline
1-2 anos &	5  &	13 \\
3-5 anos &	10 &	21 \\
6-10 anos &	30 &	11 \\
\end{tabular}
\label{tab:DadosDemograficos}
\end{table}

% \begin{tikzpicture}
% \begin{axis}[
%     ybar,
%     enlargelimits=0.15,
%     legend style={at={(0.5,-0.15)},
%       anchor=north,legend columns=-1},
%     ylabel={participantes},
%     symbolic x coords={1-2 anos,2-5 anos,6-10 anos},
%     xtick=data,
%     nodes near coords,
%     nodes near coords align={vertical},
%     ]
% \addplot coordinates {(1-2 anos,5) (2-5 anos,10) (6-10 anos,30)};
% \addplot coordinates {(1-2 anos,13) (2-5 anos,21) (6-10 anos,11)};
% \legend{experi\^encia com software,experi\^encia com android}
% \end{axis}
% \end{tikzpicture}

\subsection{Categoriza\c{c}\~ao e Identifica\c{c}\~ao dos Smells}

A an\'alise partiu da listagem das 45 respostas ao question\'ario. A partir desta listagem realizamos o processo que denominamos como verticaliza\c{c}\~ao onde foi poss\'ivel extrair 810 respostas sobre boas e m\'as pr\'aticas em 8 tipos de elementos android (Activity, Fragment, Adapter, Listener, layout, string, style e drawable resources) e boas e m\'as pr\'aticas gerais em android. O n\'umero 810 refere-se a 8 elementos android x 2 perguntas (uma questionando sobre boas pr\'aticas e outra sobre m\'as pr\'aticas) + 2 perguntas gen\'ericas sobre boas e m\'as pr\'aticas, ou seja: 18 perguntas sobre boas e m\'as pr\'aticas respondidas por 45 participantes, verticalizando, ou seja, cada resposta de boa ou m\'a pr\'atica tornando-se um registro a ser analisado, resulta em 810 respostas sobre boas ou m\'as pr\'aticas.

Nosso primeiro passo foi realizar a limpeza dos dados. Esta etapa consistiu em remover respostas obviamente n\~ao \'uteis como as que constavam em branco, continham frases como "N\~ao", "N\~ao que eu saiba", "I don't remember" e similares, as consideradas vagas como "Eu n\~ao tenho certeza se s\~ao boas praticas mas uso o que vejo por ai", as consideradas gen\'ericas como "Like all Java code, the overkill of Utils...". Foi criada uma coluna USABLE onde era informado "yes" ou "no". Das 810 boas e m\'as pr\'aticas, 352 foram consideradas e 458 desconsideradas. Das 352, 44,6\% foram apontadas como m\'as pr\'aticas e 55,4\% como boas pr\'aticas. 

Ap\'os a verticaliza\c{c}\~ao, foram realizadas diversas itera\c{c}\~oes nas respostas consideradas a fim de categoriz\'a-las em algum smell pr\'e-existente (j\'a catalogado) ou em algum android smell. Essas itera\c{c}\~oes consistiam em analisar resposta a resposta da lista e atribuir 1 ou mais categorias de algum poss\'ivel smell (pr\'e-existente ou android smell). Foram realizadas diversas itera\c{c}\~oes de categoriza\c{c}\~ao com o objetivo de normalizar as categorias, ou seja, evitar que categorias que apontem para padr\~oes similares recebam nomes diferentes, como por exemplo KAS (Keep Activity Simple) e KFS (Keep Fragment Simple) foram normalizadas para KXS (Keep Elemento Simple). Durante a an\'alise houveram 30 respostas que n\~ao eram triviais de identificar uma categoria ou mesmo de dizer se estas respostas deveriam ser consideradas. Essas respostas foram marcadas como maybe durante o processo. 9 respostas foram trocadas para "no". Foi criada a coluna WHY MAYBE OR NOT USABLE para indicar o motivo do "no" ou "maybe". Ao final, totalizavam em 313 respostas de fato consideradas, ou seja, marcadas com "yes" , e categorizadas.

Uma pr\'atica que consideramos importante para a normaliza\c{c}\~ao dos smells/categorias foi durante as itera\c{c}\~oes de categoriza\c{c}\~ao, cada categoria atribu\'ida era adicionada a uma lista de categorias e era inserido descri\c{c}\~oes pr\'oximas as respostas dadas que indicava que tipo de respostas estavam entrando naquela categoria. Esta pr\'atica eliminou a cria\c{c}\~ao de categorias com prop\'ositos similares e nomes diferentes e serviu como base para a defini\c{c}\~ao e avalia\c{c}\~ao da relev\^ancia dos smells escolhidos para serem usados nos pr\'oximos passos.

As respostas que apresentavam mais de uma categoria foram quebradas em duas ou mais linhas de respostas, de acordo com o n\'umero de categorias identificadas, de forma a ser poss\'ivel contabilizar a incid\^encia. Por exemplo, a resposta "N\~ao fazer Activities serem callbacks de execu\c{c}\~oes ass\'incronas. Herdar sempre das classes fornecidas pelas bibliotecas de suporte, nunca diretamente da plataforma" indica na primeira ora\c{c}\~ao a categoria Zumbi Referended Activity e na segunda, a categoria Inherit From Support Library Always, desta forma, ao separar foi mantido o texto (coluna opinion) apenas relativo a categoria, como se fossem duas respostas v\'alidas. Algumas respostas n\~ao estava t\~ao bem separadas qual frase indicava qual categoria, ou as vezes a resposta completa era necess\'ario para entender ambas as categoriza\c{c}\~oes, nestes casos, preferiu-se manter a resposta original, mesmo que duplicada, e categorizar cada uma de forma diferente. Ap\'os estas divis\~oes, as 313 respostas iniciais se tornaram 388.

Como \'ultima etapa do processo de categoriza\c{c}\~ao e identifica\c{c}\~ao dos smells, repassamos por todas as 30 respostas marcadas como maybe e foi decidido se seria usada ou n\~ao. 6 delas se tornaram "yes" e foram atribu\'idas categorias, 24 se tornaram "no" e o motivo foi indicado na coluna WHY MAYBE OR NOT USABLE. Ao final, a planilha de an\'alise cont\'em respostas marcadas como \'uteis ou n\~ao e nenhuma marcada como "maybe". Ao final, totalizamos 319 respostas consideradas, considerando as divis\~oes realizadas conforme mencionado acima, totalizamos com 394 respostas sobre boas e m\'as pr\'aticas categorizadas. Durante esta etapa final, um caso interessante \'e que foram identificadas diversas respostas indicando que n\~ao se deve usar Fragments por\'em sem grande argumenta\c{c}\~oes sobre o motivo, por exemplo: "Fragments are the spawn of satan" e "I try to avoid them". Estas respostas foram consideradas pela quantidade derepeti\c{c}\~oes em que apareceram.

Em resumo, o processo de categoriza\c{c}\~ao e identifica\c{c}\~ao dos smells seguiu as seguintes etapas:

Limpeza dos Dados \textrightarrow Verticaliza\c{c}\~ao \textrightarrow Itera\c{c}\~oes de Categoriza\c{c}\~ao \textrightarrow Divis\~oes \textrightarrow Elimina\c{c}\~ao das D\'uvidas (respostas marcadas com maybe)

Ao final desta etapa, conclu\'imos com uma lista com 46 categorias de smells. Para filtrarmos quais smells seriam usados nas pr\'oximas etapas. Nosso objetivo neste momento era entender quais s\~ao as categorias mais notadas pelos desenvolvedores que ale a pena refirnar e validar? Avaliamos a relev\^ancia contabilizando em quantas boas e m\'as pr\'aticas ele foi utilizando. Esta contabiliza\c{c}\~ao resultou em