% -*- root: article.tex -*-
Para obter as ideias iniciais para esta pesquisa n\'os publicamos um survey online com perguntas dissertativas sobre boas e m\'as pr\'aticas em determinados elementos da plataforma android. O objetivo das perguntas dissertativas foi obter o m\'aximo de informa\c{c}\~oes dos desenvolvedores. O survey foi escrito em ingl\^es por\'em era informado que tanto respotas ingl\^es ou portugu\^es eram aceitas. Antes da divulga\c{c}\~ao, validamos o survey com 3 desenvolvedores android, com o feedback deles fizemos alguns ajustes relacionados a obrigatoriedade de algumas respostas. 

O survey foi dividido em 3 sess\~oes. A primeira continha perguntas para mapeamento demogr\'afico e de perfil. A segunda continham perguntas relacionadas às boas e m\'as pr\'aticas em elementos da plataforma android. A terceira e \'ultima sess\~ao tentava obter alguma ideia n\~ao coletada na sess\~ao anterior al\'em do email do participante caso o mesmo se disponibilizasse a participar de outras etapa da pesquisa. Mais detalhes sobre a cria\c{c}\~ao do question\'ario ser\'a abordado da Sess\~ao 3.1. Ap\'os coletada as respostas, realizamos uma an\'alise qualitativa. Essa an\'alise \'e \'e detalhada na sess\~ao 3.2.

\subsection{Question\'ario}

O survey foi escrito em ingl\^es e continha uma mensagem informando que respotas em ingl\^es ou portugu\^es eram aceitas. Criamos e modificamos o survey coletando respostas experimentais; qualquer altera\c{c}\~ao feita no question\'ario foi baseada nas respostas que recebemos de nossos 3 participantes iniciais. Realizou-se a an\'alise qualitativa nas respostas do survey. Esse processo \'e discutido em detalhes na Se\c{c}\~ao III-F.

\subsection{Participantes}

Este survey foi respondido por 45 participantes. O survey foi divulgado abertamente em redes sociais. Coletamos 36 respostas do Brasil, 7 da Europa, 1 dos Estados Unidos e 1 preferiu n\~ao responder. Quarenta dos participantes tinham 3 anos ou mais de 10 anos de experi\^encia com desenvolvimento, destes, trinta e um participantes tinham de 3 a 8 anos de experi\^encia com android. Trinta e tr\^es dos participantes tinham de 25 a 34 anos, 8 de 19 a 24 e 3 de 35 a 54.

\subsection{Categoriza\c{c}\~ao e Identifica\c{c}\~ao dos Smells}

A an\'alise partiu da listagem das 45 respostas ao question\'ario. A partir desta listagem realizamos o processo que denominamos como verticaliza\c{c}\~ao onde foi possível extrair 810 respostas sobre boas e m\'as pr\'aticas em 8 tipos de elementos android (Activity, Fragment, Adapter, Listener, layout, string, style e drawable resources) e boas e m\'as pr\'aticas gerais em android. O n\'umero 810 refere-se a 8 elementos android x 2 perguntas (uma questionando sobre boas pr\'aticas e outra sobre m\'as pr\'aticas) + 2 perguntas gen\'ericas sobre boas e m\'as pr\'aticas, ou seja: 18 perguntas sobre boas e m\'as pr\'aticas respondidas por 45 participantes, verticalizando, ou seja, cada resposta de boa ou m\'a pr\'atica tornando-se um registro a ser analisado, resulta em 810 respostas sobre boas ou m\'as pr\'aticas.

Nosso primeiro passo foi realizar a limpeza dos dados. Esta etapa consistiu em remover respostas obviamente n\~ao \'uteis como as que constavam em branco, continham frases como "N\~ao", "N\~ao que eu saiba", "I don't remember" e similares, as consideradas vagas como "Eu n\~ao tenho certeza se s\~ao boas praticas mas uso o que vejo por ai", as consideradas gen\'ericas como "Like all Java code, the overkill of Utils...". Foi criada uma coluna USABLE onde era informado "yes" ou "no". Das 810 boas e m\'as pr\'aticas, 352 foram consideradas e 458 desconsideradas. Das 352, 44,6\% foram apontadas como m\'as pr\'aticas e 55,4\% como boas pr\'aticas. 

Ap\'os a verticaliza\c{c}\~ao, foram realizadas diversas itera\c{c}\~oes nas respostas consideradas a fim de categoriz\'a-las em algum smell pr\'e-existente (j\'a catalogado) ou em algum android smell. Essas itera\c{c}\~oes consistiam em analisar resposta a resposta da lista e atribuir 1 ou mais categorias de algum possível smell (pr\'e-existente ou android smell). Foram realizadas diversas itera\c{c}\~oes de categoriza\c{c}\~ao com o objetivo de normalizar as categorias, ou seja, evitar que categorias que apontem para padr\~oes similares recebam nomes diferentes, como por exemplo KAS (Keep Activity Simple) e KFS (Keep Fragment Simple) foram normalizadas para KXS (Keep Elemento Simple). Durante a an\'alise houveram 30 respostas que n\~ao eram triviais de identificar uma categoria ou mesmo de dizer se estas respostas deveriam ser consideradas. Essas respostas foram marcadas como maybe durante o processo. 9 respostas foram trocadas para "no". Foi criada a coluna WHY MAYBE OR NOT USABLE para indicar o motivo do "no" ou "maybe". Ao final, totalizavam em 313 respostas de fato consideradas, ou seja, marcadas com "yes" , e categorizadas.

Uma pr\'atica que consideramos importante para a normaliza\c{c}\~ao dos smells/categorias foi durante as itera\c{c}\~oes de categoriza\c{c}\~ao, cada categoria atribuída era adicionada a uma lista de categorias e era inserido descri\c{c}\~oes pr\'oximas as respostas dadas que indicava que tipo de respostas estavam entrando naquela categoria. Esta pr\'atica eliminou a cria\c{c}\~ao de categorias com prop\'ositos similares e nomes diferentes e serviu como base para a defini\c{c}\~ao e avalia\c{c}\~ao da relevância dos smells escolhidos para serem usados nos pr\'oximos passos.

As respostas que apresentavam mais de uma categoria foram quebradas em duas ou mais linhas de respostas, de acordo com o n\'umero de categorias identificadas, de forma a ser possível contabilizar a incid\^encia. Por exemplo, a resposta "N\~ao fazer Activities serem callbacks de execu\c{c}\~oes assíncronas. Herdar sempre das classes fornecidas pelas bibliotecas de suporte, nunca diretamente da plataforma" indica na primeira ora\c{c}\~ao a categoria Zumbi Referended Activity e na segunda, a categoria Inherit From Support Library Always, desta forma, ao separar foi mantido o texto (coluna opinion) apenas relativo a categoria, como se fossem duas respostas v\'alidas. Algumas respostas n\~ao estava t\~ao bem separadas qual frase indicava qual categoria, ou as vezes a resposta completa era necess\'ario para entender ambas as categoriza\c{c}\~oes, nestes casos, preferiu-se manter a resposta original, mesmo que duplicada, e categorizar cada uma de forma diferente. Ap\'os estas divis\~oes, as 313 respostas iniciais se tornaram 388.

Como \'ultima etapa do processo de categoriza\c{c}\~ao e identifica\c{c}\~ao dos smells, repassamos por todas as 30 respostas marcadas como maybe e foi decidido se seria usada ou n\~ao. 6 delas se tornaram "yes" e foram atribuídas categorias, 24 se tornaram "no" e o motivo foi indicado na coluna WHY MAYBE OR NOT USABLE. Ao final, a planilha de an\'alise cont\'em respostas marcadas como \'uteis ou n\~ao e nenhuma marcada como "maybe". Ao final, totalizamos 319 respostas consideradas, considerando as divis\~oes realizadas conforme mencionado acima, totalizamos com 394 respostas sobre boas e m\'as pr\'aticas categorizadas. Durante esta etapa final, um caso interessante \'e que foram identificadas diversas respostas indicando que n\~ao se deve usar Fragments por\'em sem grande argumenta\c{c}\~oes sobre o motivo, por exemplo: "Fragments are the spawn of satan" e "I try to avoid them". Estas respostas foram consideradas pela quantidade derepeti\c{c}\~oes em que apareceram.

Em resumo, o processo de categoriza\c{c}\~ao e identifica\c{c}\~ao dos smells seguiu as seguintes etapas:

Limpeza dos Dados > Verticaliza\c{c}\~ao > Itera\c{c}\~oes de Categoriza\c{c}\~ao > Divis\~oes > Elimina\c{c}\~ao das D\'uvidas (respostas marcadas com maybe)

Ao final desta etapa, concluímos com uma lista com 46 categorias de smells. Para filtrarmos quais smells seriam usados nas pr\'oximas etapas. Nosso objetivo neste momento era entender quais s\~ao as categorias mais notadas pelos desenvolvedores que ale a pena refirnar e validar? Avaliamos a relevância contabilizando em quantas boas e m\'as pr\'aticas ele foi utilizando. Esta contabiliza\c{c}\~ao resultou em