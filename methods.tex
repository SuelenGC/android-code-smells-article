% -*- root: article.tex -*-
Durante o processo de codificação das respostas para as 18 perguntas abertas sobre boas e m\'as pr\'aticas, 	
Conduzimos um estudo qualitativo e explorat\'orio onde os dados foram coletados atrav\'es de um question\'ario online com desenvoveldores Android. Esta se\c{c}\~ao descreve de forma detalhada a estrutura do question\'ario, os participantes e a an\'alise realizada sobre as respostas obtidas.

\subsection{Question\'ario}
\label{sub:questionario}

O question\'ario continha 25 quest\~oes divididas em tr\^es se\c{c}\~oes: a primeira se\c{c}\~ao continha 6 perguntas demogr\'aficas, a segunda se\c{c}\~ao continha 16 perguntas sobre boas e m\'as pr\'aticas relacionadas ao \textit{front-end} Android e a terceira se\c{c}\~ao continha 3 perguntas, 2 para obter \'ultimos pensamentos sobre boas e m\'as pr\'aticas e 1 solicitando email caso o participante tivesse interesse em etapas futuras da pesquisa. O question\'ario foi escrito em ingl\^es por\'em informava o participante que respostas em ingl\^es ou portugu\^es eram aceitas. Antes da divulga\c{c}\~ao, realizamos um piloto com 3 desenvolvedores Android. Todos estes dados est\~ao dispon\'iveis no nosso pacote de replica\c{c}\~ao\footnote{https://github.com/SuelenGC/android-code-smells-article}.

A primeira se\c{c}\~ao continha 6 quest\~oes demogr\'aficas obrigat\'orias de multipla escolha. Abordavam sobre idade (18 ou menos, 19 a 24, 25 a 34 e assim por diante at\'e 55 ou mais), estado de resid\^encia (foi dada uma lista com estados do Brasil, Estados Unidos e Europa), anos de experi\^encia com desenvolvimento de software, (1 anos ou menos, 2 anos, 3 anos, e assim por diante at\'e 10 ou mais), anos de experi\^encia com desenvolvimento Android (mesma escala de anos da quest\~ao anterior), uma quest\~ao sobre linguagens que o participante se considerava proeficiente (Java, Python, Ruby, Android, dentre outras) e sobre o \'ultimo grau de escolaridade (estudante de bacharelado, bacharelado, mestrado e doutorado). As quest\~oes sobre idade, regi\~ao, linguagens e grau de escolaridade continham a op\c{c}\~ao ``outros'' para o caso de nenhuma das outras op\c{c}\~oes atenderem, ao selecionar ``outros'' era poss\'ivel escrever uma resposta de forma manual.

A segunda se\c{c}\~ao continha 16 quest\~oes opcionais e dissertativas sobre boas e m\'as pr\'aticas relacionadas ao \textit{front-end} Android. Para cada elemento do \textit{front-end} Android foram feitas duas perguntas, uma sobre boas e outra sobre m\'as pr\'aticas percebidas pelos participantes. Por exemplo, para o elemento \textsc{Activity} foram feitas as perguntas:

\begin{itemize} 
	\item[$\textasteriskcentered$] Do you have any good practices to deal with Activities?
	\item[$\textasteriskcentered$] Do you have any bad practices to deal with Activities? 
\end{itemize}

A terceira se\c{c}\~ao continha 3 perguntas opcionais e dissertativas, 2 para captar qualquer \'ultima ideia sobre boas e m\'as pr\'aticas n\~ao captadas nas quest\~oes anteriores e 1 solicitando o email do participante caso o mesmo tivesse interesse em participar de etapas futuras da pesquisa. As perguntas sobre boas e m\'as pr\'aticas foram as a seguir: 

\begin{itemize} 
	\item[$\textasteriskcentered$] Are there any other *GOOD* practices in Android Presentation Layer we did not asked you or you did not said yet?
	\item[$\textasteriskcentered$] Are there any other *BAD* practices in Android Presentation Layer we did not asked you or you did not said yet?
\end{itemize}

Antes da divulga\c{c}\~ao, realizamos um piloto com 3 desenvolvedores Android e com o feedback deles fizemos alguns ajustes relacionados a obrigatoriedade das perguntas da segunda se\c{c}\~ao do question\'ario, onde todas tornaram-se opcionais. As respostas dos participantes piloto foram desconsideradas para efeitos de vi\'es. 

\subsection{Participantes}
\label{sub:participantes}

O question\'ario foi divulgado em redes sociais como Facebook, Twitter e Linkedin, em grupos de discuss\~ao sobre Android como \href{https://groups.google.com/forum/#!forum/androidbrasil-dev}{\textit{Android Dev Brasil}}, \href{https://groups.google.com/forum/\#!forum/android-brasil-projetos}{\textit{Android Brasil Projetos}} e o grupo do \href{http://slack.androiddevbr.org/}{\textit{Slack Android Dev Br}}, maior grupo de desenvolvedores Android do Brasil com 2622 participantes at\'e o momento da escrita deste artigo. 

O question\'ario esteve aberto por aproximadamente 3 meses e meio, de 9 de Outubro de 2016 at\'e 18 de Janeiro de 2017. Recebemos um total de 45 respostas sendo que 41 foram submetidas em Outubro, 3 no come\c{c}o de Novembro e 1 em Janeiro. Uma poss\'ivel explica\c{c}\~ao para praticamente n\~ao termos tido respostas nos meses de Novembro e Dezembro pode ser as festas comemorativas de final de ano. 

80\% dos participantes responderam pelo menos 3 perguntas sobre boas e m\'as pr\'aticas no \textit{front-end} Android (7 responderam de 3 a 6, 6 responderam de 8 a 10 e 23 responderam 13 ou mais, sendo que desses, 14 responderam todas) e apenas 20\% responderam uma (2 participantes) ou nenhuma (7 participantes). A pergunta solicitando o email foi respondida por 53\% dos participantes, o que pode indicar um interesse leg\'itimo da comunidade de desenvolvedores Android pelo tema, refor\c{c}ando a relev\^ancia do estudo. A Tabela \ref{tab:ResponseRate} apresenta o percentual de respostas obtido por cada uma das 18 perguntas sobre boas e m\'as pr\'aticas, podemos notar que a menor taxa de respostas foi na quest\~ao 18 e a maior na quest\~ao 1.

\begin{table*}[t]
\centering
\caption{Percentual de respostas das perguntas sobre boas e m\'as pr\'aticas.}
\small
\begin{tabular}{rp{12cm}c}
\toprule
\multicolumn{2}{l}{\textbf{Quest\~oes}} 	& 	\multicolumn{1}{c}{\textbf{Taxa de Respostas}} \\
\hline
Q1 & Do you have any good practices to deal with \textsc{Activities}?	& 	80 \% \\ 
Q2 & Do you have anything you consider a bad practice when dealing with \textsc{Activities}?	& 	78 \% \\ 
Q3 & Do you have any good practices to deal with \textsc{Fragments}?	& 	73 \% \\ 
Q4 & Do you have anything you consider a bad practice when dealing with \textsc{Fragments}?	& 	69 \% \\ 
Q5 & Do you have any good practices to deal with \textsc{Adapters}?	& 	67 \% \\ 
Q6 & Do you have anything you consider a bad practice when dealing with \textsc{Adapters}?	& 	60 \% \\ 
Q7 & Do you have any good practices to deal with \textsc{Listeners}?	& 	53 \% \\ 
Q8 & Do you have anything you consider a bad practice when dealing with \textsc{Listeners}?	& 	51 \% \\ 
Q9 & Do you have any good practices to deal with \textsc{Layout Resources}?	& 	62 \% \\ 
Q10 & Do you have anything you consider a bad practice when dealing with \textsc{Layout Resources}?	& 	51 \% \\ 
Q11 & Do you have any good practices to deal with \textsc{Styles Resources}?	& 	51 \% \\ 
Q12 & Do you have anything you consider a bad practice when dealing with \textsc{Styles Resources}?	& 	49 \% \\ 
Q13 & Do you have any good practices to deal with \textsc{String Resources}?	& 	62 \% \\ 
Q14 & Do you have anything you consider a bad practice when dealing with \textsc{String Resources}?	& 	51 \% \\ 
Q15 & Do you have any good practices to deal with \textsc{Drawable Resources}?	& 	53 \% \\ 
Q16 & Do you have anything you consider a bad practice when dealing with \textsc{Drawable Resources}?	& 	47 \% \\ 
Q17 & Are there any other *GOOD* practices in Android Presentation Layer we did not asked you or you did not said yet?	& 	49 \% \\ 
Q18 & Are there any other *BAD* practices in Android Presentation Layer we did not asked you or you did not said yet?	& 	44 \% \\ 
\toprule
\end{tabular}
\label{tab:ResponseRate}
\end{table*}

Com a an\'alise das quest\~oes demogr\'aficas foi poss\'ivel notar que atingimos com sucesso \textit{desenvolvedores Android com variados n\'iveis de experi\^encia e de diversas regi\~oes} pois: 1) 100\% dos participantes indicaram possuir alguma experi\^encia com desenvolvimento Android, 2) menos de 14\% indicaram possuir 1 anos ou menos de experi\^encia com Android e mais de 86\% indicaram 2 anos ou mais (15,5\% 2 anos, 13,3\% 4 anos, 6,5\% 5 anos, 15,5\% 6 anos, 4,4\% 7 anos e 4,4\% 8 anos), 4) 36 respostas foram do Brasil, 7 de pa\'ises europeus e 1 dos Estados Unidos (Calif\'ornia). Vale lembrar que a plarforma Android completa 10 anos em 2017, ou seja, 5 anos de experi\^encia nessa plataforma representa 50\% do tempo de vida dela desde seu an\'uncio em 2007. Esses dados, sobre a experi\^encia dos participantes, s\~ao apresentados na Tabela \ref{tab:DadosDemograficos}.

\begin{table}[h]
\centering
\caption{Experi\^encia dos participantes com desenvolvimento Android.}
\small
\begin{tabular}{l|c|r}
\toprule
\textbf{Anos de Experi\^encia} & \textbf{Participantes} & \multicolumn{1}{c}{\textbf{\%}}  \\
\hline
1 ano ou menos 	&	 6 		& 	13,3 \%	 \\
2 anos 			& 	 7 		& 	15,6 \%	 \\
3 anos 			& 	 12		& 	26,7 \%	 \\
4 anos 			& 	 6 		& 	13,3 \%	 \\
5 anos 			& 	 3 		& 	 6,7 \%	 \\
6 anos 			& 	 7 		& 	15,6 \%	 \\
7 anos 			& 	 2 		& 	 4,4 \%	 \\
8 anos 			& 	 2 		& 	 4,4 \%	 \\
\toprule
\end{tabular}
\label{tab:DadosDemograficos}
\end{table}

% \begin{tikzpicture}
% \begin{axis}[
%     ybar,
%     enlargelimits=0.15,
%     legend style={at={(0.5,-0.15)},
%       anchor=north,legend columns=-1},
%     ylabel={participantes},
%     symbolic x coords={1-2 anos,2-5 anos,6-10 anos},
%     xtick=data,
%     nodes near coords,
%     nodes near coords align={vertical},
%     ]
% \addplot coordinates {(1-2 anos,5) (2-5 anos,10) (6-10 anos,30)};
% \addplot coordinates {(1-2 anos,13) (2-5 anos,21) (6-10 anos,11)};
% \legend{experi\^encia com software,experi\^encia com Android}
% \end{axis}
% \end{tikzpicture}

\subsection{An\'alise dos Dados}
\label{sub:smells-definition}

O processo de an\'alise partiu da listagem das 45 respostas do question\'ario e se deu em 4 passos: verticaliza\c{c}\~ao, limpeza dos dados, codifica\c{c}\~ao e divis\~ao. 

O processo que denominamos de \textit{verticaliza\c{c}\~ao} consistiu em considerar cada resposta de boa ou m\'a pr\'atica como um registro individual a ser analisado. Ou seja, cada participante respondeu 18 perguntas sobre boas e m\'as pr\'aticas no \textit{front-end} Android (2 perguntas para cada elemento e mais duas perguntas gen\'ericas). Com o processo de \textit{verticaliza\c{c}\~ao}, cada uma dessas respostas se tornou um registro, ou seja, cada participante resultava em 18 respostas a serem analisadas, totalizando 810 respostas (18 perguntas multiplicado por 45 participantes) sobre boas e m\'as pr\'aticas.

O passo seguinte foi realizar a \textit{limpeza dos dados}. Esse passo consistiu em remover respostas obviamente n\~ao \'uteis como respostas em branco, que continham frases como \textit{"N\~ao"}, \textit{"N\~ao que eu saiba"}, \textit{"Eu n\~ao me lembro"} e similares, as consideradas vagas como \textit{"Eu n\~ao tenho certeza se s\~ao boas praticas mas uso o que vejo por ai"}, as consideradas gen\'ericas como \textit{"Como todo c\'odigo java..."} e as que n\~ao eram relacionadas a boas pr\'aticas de c\'odigo. Das 810 boas e m\'as pr\'aticas, 352 foram consideradas e 458 desconsideradas. Das 352, 44,6\% foram apontadas como m\'as pr\'aticas e 55,4\% como boas pr\'aticas. 

Em seguida realizamos a codifica\c{c}\~ao sobre as boas e m\'as pr\'aticas. Durante esse processo, categorias foram emergindo a partir das respostas. Durante a codifica\c{c}\~ao, houveram 30 respostas que n\~ao eram triviais de identificar uma categoria ou mesmo de dizer se essas respostas deveriam ser consideradas, sendo marcadas como \textit{"talvez"} durante o processo e reavaliadas ao final, onde 6 permaneceram e 24 foram desconsideradas. Ainda durante a codifica\c{c}\~ao, 9 respostas incialmente consideradas, foram desconsideradas. Para toda resposta desconsiderada foi indicado um motivo.