% -*- root: article.tex -*-
Nesta pesquisa objetivamos investigar a percep\c{c}\~ao de desenvolvedores android sobre boas e m\'as pr\'aticas de c\'odigo em projetos android. Desta forma, pretendemos responder as seguintes quest\~oes de pesquisa: \\

\textbf{RQ1} O que desenvolvedores consideram boas e m\'as pr\'aticas no desenvolvimento android? \\

\textbf{RQ2} Poderiam estas m\'as pr\'aticas serem consideradas \textit{code smells} android? \\

\textbf{RQ3} \textit{Code smells} tradicionais s\~ao percebidos em c\'odigo android? \\

Para obter as ideias iniciais para esta pesquisa, publicamos um question\'ario online sobre boas e m\'as pr\'aticas em determinados elementos da plataforma android. Perguntas dissertativas foram usadas para permitir respostas completas e n\~ao submeter o participante a nenhum vi\'es. O question\'ario foi escrito em ingl\^es por\'em informava o participante que respostas em ingl\^es ou portugu\^es eram aceitas. Antes da divulga\c{c}\~ao, fizemos um piloto com 3 desenvolvedores android, com o feedback deles fizemos alguns ajustes relacionados a obrigatoriedade de algumas perguntas. As respostas do piloto foram desconsideradas para efeitos de vi\'es. 

O question\'ario foi dividido em 3 se\c{c}\~oes. A primeira continham perguntas para mapeamento demogr\'afico. A segunda continham perguntas relacionadas a boas e m\'as pr\'aticas em elementos de projetos android. A terceira se\c{c}\~ao continham perguntas para obter alguma ideia que n\~ao tenha sido capturada nas quest\~oes anteriores, al\'em do email do participante caso o mesmo quisesse ser contactado para participar de outras etapas da pesquisa. 

Com os resultados obtidos foi poss\'ivel compilar um cat\'alogo com 21 android \textit{code smells}. Esse cat\'alogo ir\'a contribuir como base para i) a defini\c{c}\~ao de heur\'isticas para detec\c{c}\~ao ii) automatiza\c{c}\~ao de ferramentas de detec\c{c}\~ao autom\'atica desses smells. \\

A seguir, a se\c{c}\~ao \ref{sub:questionario} aborda sobre a elabora\c{c}\~ao do question\'ario, na \ref{sub:participantes} falamos sobre os participantes e a se\c{c}\~ao \ref{sub:smells-definition} aborda de forma detalhada a an\'alise para a defini\c{c}\~ao dos smells.

\subsection{Question\'ario}
\label{sub:questionario}

Aplicativos Android s\~ao escritos com a linguagem de programa\c{c}\~ao Java \cite{AndroidFundamentals}. Ent\~ao a primeira quest\~ao \'e: por que buscar por \textit{smells} android sendo que j\'a existem tantos \textit{smells} Java? Pesquisas t\^em demonstrado que tecnologias diferentes podem apresentar \textit{code smells} espec\'ificos, como por exemplo Aniche et al. identificou 6 \textit{code smells} espec\'ificos ao framework Spring MVC, um framework Java para desenvolvimento web. Outras pesquisas conluem que projetos android possuem caracter\'isticas diferentes de projetos java \cite{Hecht2015, Mannan_Dig_Ahmed_Jensen_Abdullah_Almurshed, ReimannBrylski2013}, por exemplo, o \textit{front-end} \'e representado por arquivos XML e o ponto de entrada da aplica\c{c}\~ao \'e dado por \textit{event-handler} \cite{AndroidActivities2016} como o m\'etodo \textsc{onCreate}. Encontramos tamb\'em diversas pesquisas pesquisas sobre \textit{code smells} v\^em sendo feitas sobre tecnologias usadas no desenvolvimento de \textit{front-end} web como CSS \cite{CSSCodeSmell} e Javascript \cite{BB}. Estas pesquisas nos inspiraram a buscar entender se existem \textit{code smells} no \textit{front-end} android.

Para definirmos quais seriam os elementos usados no desenvolvimento do \textit{front-end} do android fizemos uma extensa revis\~ao da documenta\c{c}\~ao oficial \cite{AndroidDeveloperSite2016} e chegamos na listagem a seguir:

\begin{itemize} 
	\item[$\circ$] classes do tipo \textsc{Activity};
	\item[$\circ$] classes do tipo \textsc{Fragment};
	\item[$\circ$] classes do tipo \textsc{Listener};
	\item[$\circ$] classes do tipo \textsc{Adapter};
	\item[$\circ$] recursos do aplicativo como \textsc{drawable}s, \textsc{layout}s, \textsc{style}s, \textsc{color}s dentre outros.
\end{itemize}

Como existem muitos tipos de recursos \cite{AndroidResourcesOverview}, com o objetivo de limitar o tamanho do question\'ario, selecionamos os quatro recursos mais frequentes em aplicativos: \textsc{layout}, \textsc{styles}, \textsc{string} e \textsc{drawable}.

A segunda se\c{c}\~ao do question\'ario abordou perguntas sobre boas e m\'as pr\'aticas sobre cada um dos elementos supra-citados. Baseamos nossa estrutura de perguntas usada por Aniche et al. ao pesquisar sobre \textit{smells} no framework Spring MVC. Por exemplo, para \textsc{Activity}s fizemos as seguintes perguntas: como por exemplo, seu \textit{front-end} precisa de diversos tipos de arquivos xmls dentre

\begin{itemize} 
	\item[$\circ$] Do you have any good practices to deal with Activities?
	\item[$\circ$] Do you have any bad practices to deal with Activities? 
\end{itemize}

A terceira se\c{c}\~ao objetivou captar qualquer \'ultima ideia sobre boas e m\'as pr\'aticas n\~ao captadas nas quest\~oes anteriores. Para isso fizemos as seguintes perguntas: 

\begin{itemize} 
	\item[$\circ$] Are there any other *GOOD* practices in Android Presentation Layer we did not asked you or you did not said yet?
	\item[$\circ$] Are there any other *BAD* practices in Android Presentation Layer we did not asked you or you did not said yet?
\end{itemize}


\subsection{Participantes}
\label{sub:participantes}

Com o question\'ario obtivemos 45 respostas. O question\'ario foi divulgado em grupos de desenvolvedores android como por exemplo o Slack AndroidDevBR, maior grupo de desenvolvedores android do Brasil contando hoje com mais de 2500 participantes. Coletamos 36 respostas do Brasil, 7 de pa\'ises europeus e 1 dos Estados Unidos, 1 participantes preferiu n\~ao responder. A tabela \ref{tab:DadosDemograficos} apresenta dados sobre anos de experi\^encia dos participantes com desenvolvimento de sofware e com desenvolvimento android. Notamos que 67\% dos participantes possuem mais de 5 anos de experi\^encia com desenvolvimento de software e que 71\% possuem 3 anos ou mais de experi\^encia com desenvolvimento android.

\begin{table}[h]
\centering
\caption{Experi\^encia dos participantes com desenvolvimento de software e desenvolvimento android.}
\begin{tabular}{c|p{2cm}p{2cm}}
& Experi\^encia com Software & Experi\^encia com Android \\
\hline
1-2 anos &	5  &	13 \\
3-5 anos &	10 &	21 \\
6-10 anos &	30 &	11 \\
\end{tabular}
\label{tab:DadosDemograficos}
\end{table}

% \begin{tikzpicture}
% \begin{axis}[
%     ybar,
%     enlargelimits=0.15,
%     legend style={at={(0.5,-0.15)},
%       anchor=north,legend columns=-1},
%     ylabel={participantes},
%     symbolic x coords={1-2 anos,2-5 anos,6-10 anos},
%     xtick=data,
%     nodes near coords,
%     nodes near coords align={vertical},
%     ]
% \addplot coordinates {(1-2 anos,5) (2-5 anos,10) (6-10 anos,30)};
% \addplot coordinates {(1-2 anos,13) (2-5 anos,21) (6-10 anos,11)};
% \legend{experi\^encia com software,experi\^encia com android}
% \end{axis}
% \end{tikzpicture}

\subsection{Categoriza\c{c}\~ao e Identifica\c{c}\~ao dos Smells}
\label{sub:smells-definition}

O processo de categoriza\c{c}\~ao e defini\c{c}\~ao dos smells seguiu as seguintes etapas: Verticaliza\c{c}\~ao, Limpeza dos Dados, Itera\c{c}\~oes de Categoriza\c{c}\~ao, Divis\~oes e Elimina\c{c}\~ao das D\'uvidas. Essas etapas s\~ao detalhadas a seguir.

A an\'alise partiu da listagem das 45 respostas do question\'ario. A partir desta listagem realizamos o processo que denominamos como verticaliza\c{c}\~ao, ou seja, cada resposta de boa ou m\'a pr\'atica se tornou um registro individual a ser analisado, resultando em 810 respostas sobre boas ou m\'as pr\'aticas em algum elemento do front-end android. O n\'umero 810 refere-se as 18 perguntas sobre boas e m\'as pr\'aticas multiplicado pelos 45 participantes.

Nosso segundo passo foi realizar a limpeza dos dados. Esta etapa consistiu em remover respostas obviamente n\~ao \'uteis como respostas em branco, que continham frases como \textit{"N\~ao"}, \textit{"N\~ao que eu saiba"}, \textit{"I don't remember"} e similares, as consideradas vagas como \textit{"Eu n\~ao tenho certeza se s\~ao boas praticas mas uso o que vejo por ai"}, as consideradas gen\'ericas como \textit{"Like all Java code, the overkill of Utils..."} e as que n\~ao eram relacionadas a boas pr\'aticas de c\'odigo. Das 810 boas e m\'as pr\'aticas, 352 foram consideradas e 458 desconsideradas. Das 352, 44,6\% foram apontadas como m\'as pr\'aticas e 55,4\% como boas pr\'aticas. 

Em seguida, realizamos diversas itera\c{c}\~oes nas respostas sobre boas e m\'as pr\'aticas consideradas a fim de categoriz\'a-las em algum novo smell android ou algum smell pr\'e-existente. Essas itera\c{c}\~oes consistiam em analisar resposta a resposta e atribuir uma ou mais categorias de algum poss\'ivel novo smell android ou pr\'e-existente. Foram realizadas diversas itera\c{c}\~oes de categoriza\c{c}\~ao com o objetivo de normalizar as categorias, ou seja, evitar sin\^onimos e hom\^onimos. Um sin\^onimo \'e o mesmo conceito com dois nomes diferentes e hom\^onimos s\~ao dois conceitos diferentes com o mesmo nome. 

Durante a categoriza\c{c}\~ao houveram 30 respostas que n\~ao eram triviais de identificar uma categoria ou mesmo de dizer se estas respostas deveriam ser consideradas, essas foram marcadas como \textit{"talvez"} durante o processo e reavaliadas ao final, onde 6 permaneceram e 24 foram desconsideradas. Uma situa\c{c}\~ao interessante \'e que diversas dessas respostas indicavam que n\~ao se deve usar Fragments por\'em n\~ao apresentavam nenhum argumento sobre o motivo, por exemplo: \textit{"Fragments are the spawn of satan"} e \textit{"I try to avoid them"}. Estas respostas inicialmente seriam desconsideradas, mas pela quantidade de repeti\c{c}\~oes obtidas, 10 no caso, optamos por considerar.

Tamb\'em durante a categoriza\c{c}\~ao, 9 respostas incialmente consideradas, foram desconsideradas. Para toda resposta desconsiderada foi indicado um motivo. Ao final da categoriza\c{c}\~ao, 313 boas e m\'as pr\'aticas foram de fato consideradas.

Durante a categoriza\c{c}\~ao, uma mec\^anica que consideramos importante para a normaliza\c{c}\~ao das categorias, foi a cria\c{c}ã\~ao de uma lista de categorias, onde a cada nova categoria atribu\'ida, increment\'avamos a lista e preench\'iamos com descri\c{c}\~oes que indicavam que tipo de boa ou m\'a pr\'atica estava recebendo aquela categoria. Esta mec\^anica ajudou a evitar hom\^onimos e sin\^onimos e serviu como base para a defini\c{c}\~ao e avalia\c{c}\~ao da relev\^ancia dos smells a serem trabalhados nos pr\'oximos passos.

Em seguida, passamos pela etapa de divis\~ao, ou seja, as respostas que receberam mais de uma categoria foram divididas em duas ou mais respostas, de acordo com o n\'umero de categorias identificadas. Por exemplo, a resposta \textit{"N\~ao fazer Activities serem callbacks de execu\c{c}\~oes ass\'incronas. Herdar sempre das classes fornecidas pelas bibliotecas de suporte, nunca diretamente da plataforma"} indica na primeira ora\c{c}\~ao a categoria de smell que denominamos \textbf{Zumbi Referended Activity} e na segunda ora\c{c}\~ao, a categoria de smell \textbf{Inherit From Support Library Always}. Ao divid\'i-la mantivemos o texto da resposta apenas relativo a categoria, como se fossem duas respostas v\'alidas. Em algumas respostas que foram divididas n\~ao pudemos dividir o texto pois a resposta completa era necess\'ario para entender ambas as categoriza\c{c}\~oes, nesses casos, mantivemos a resposta original, mesmo que duplicada, e categorizamos cada uma de forma diferente. Ap\'os estas divis\~oes, as 313 respostas iniciais se tornaram 388 sendo cada uma com apenas uma categoria de smell. 

Ao final de todas as etapas, conclu\'imos com 394 respostas sobre boas e m\'as pr\'aticas categorizadas em 46 smells. A seguir, descrevemos o processo de sele\c{c}\~ao dos smells trabalhados nas etapas seguintes. 