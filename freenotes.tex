% \subsubsection{God/Large Class} s\~ao \textit{code smells} previamente definidos nas literaturas de Arthur Riel (\textit{God Class}) \cite{Riel} e Martin Fowler (\textit{Large Class}) \cite{RefactoringFowler1999} e tratam de classes grandes que cont\'em muitas responsabilidades e depend\^encias. Percebemos que esse \textit{smell} tamb\'em \'e uma preocupa\c{c}\~ao dos desenvolvedores Android ao lidarem com \textsc{Activities} (8 participantes), \textsc{Fragments} (3 participantes), \textsc{Listeners} (1 participante) e \textsc{Style Resources} (1 participante). P8 indicou como m\'a pr\'atica em Activity ``Considerar Activities como \textit{God Classes}'' (tradu\c{c}\~ao livre). 


\subsubsection{Use Fragments Just If Needed} 
Esta categoria agrupa 10 respostas que indicam que se deve evitar a utiliza\c{c}\~ao de \textsc{Fragment}s e s\'o us\'a-los se for realmente necess\'ario. Por exemplo, P4 indica como boa pr\'atica sobre \textsc{Fragments} que ``Tente substituir Fragments o m\'aximo poss\'ivel por ViewGroups. Apenas use-os se tiver uma boa raz\~ao para isso'' (tradu\c{c}\~ao livre), P28 diz ``Se poss\'ivel, evitar a todo custo''. 

\subsubsection{Use Fragments} 
Esta categoria agrupa 11 respostas onde os participantes recomendam o uso de \textsc{Fragments}. Por exemplo, P14 diz ``gosto de sempre usar fragmento, mesmo que seja uma tela simples com somente um fragmento'', j\'a P19 diz ``[...] utilizar Fragments sempre que poss\'ivel''. 

\subsubsection{Nested Layout} 
Esta categoria agrupa 21 respostas onde os participantes indicam \textsc{layout resources} com aninhamento de elementos muito profundos como uma m\'a pr\'atica. De fato, esta categoria poderia ser considerada um android code smell, mas j\'a existe uma ferramenta oficial do Android [] que valida este sintoma, logo, j\'a \'e um problema endere\c{c}ado.



% \begin{table}[h]
% \centering
% \caption{Smells identificados vs. recorr\^encia percebida pelos participantes do question\'ario}
% \small
% \begin{tabular}{p{3cm}|p{4cm}}
% \toprule
% \textbf{Recorr\^encia} & \textbf{Quantidade de Smells} \\
% \hline
% Alta  			& 5 \\
% M\'edia 		& 17 (1 \textit{smell} tradicional) \\
% Baixa			& 5 \\
% Irrelevante 	& 20 \\
% \toprule
% \end{tabular}
% \label{tab:DadosDemograficos2}
% \end{table}

% Os irrelevantes foram desconsiderados nesta etapa de defini\c{c}\~ao. Os demais foram definidos com a ajuda das respostas dos participantes. Para cada \textit{smell} definimos os seguintes t\'opicos: 

% \begin{itemize} 
% 	\item[$\textasteriskcentered$] \textbf{Quando ocorre}. Indicamos os motivos pelo qual foi considerado uma m\'a pr\'atica.   
% 	\item[$\textasteriskcentered$] \textbf{Contexto/exemplo}. Indicamos algum exemplo ou contexto pr\'atico. 
% 	\item[$\textasteriskcentered$] \textbf{Elementos afetados}. Eventualmente algum \textit{smell} afeta mais de um elemento, nesse t\'opico abordamos em quais os elementos este \textit{smell} pode estar presente. 
% 	\item[$\textasteriskcentered$] \textbf{Solu\c{c}\~ao}. Indicamos poss\'iveis refatora\c{c}\~oes para reduzir ou eliminar o \textit{smell}. \\
% \end{itemize}

% \'E importante ressaltar que todas as defini\c{c}\~oes dos \textit{smells} foram apoiadas nas respostas dos participantes. Se os participantes n\~ao indicaram algum t\'opico para algum \textit{smell}, o mesmo n\~ao foi definido. \\


% \subsection{Android Code Smells}
% \label{sub:Android-code-smells}

\subsubsection{Duplicated Styles Attributes (DSA)}
Este smell foi identificado em 7 respostas. Segundo os participantes, ccorre quando mais de um view de um ou mais layouts usam o mesmo conjunto de atributos para definir sua apar\^encia. Por exemplo, segundo P34 \textit{``Sempre que eu noto que tenho mais de um recurso usando o mesmo estilo, eu tento mov\^e-lo para o meu \textit{style resource}.''} (tradu\c{c}\~ao livre). Passando a mesma ideia, P32 diz \textit{``Utilizar muitas propriedades em um \'unico componente. Se tiver que usar muitas, prefiro colocar no arquivo de styles.''}. Ou seja, os elementos afetados por este smell s\~ao layout ou style resources. E as solu\c{c}\~oes indicadas s\~ao extrair um novo estilo que agrupe o conjunto de atributos duplicados e passar a usar o estilo criado.

Por exemplo, considere o seguinte \textbf{contexto} onde o \texttt{TextView} abaixo \'e usado em dois momentos no mesmo layout (exemplo extra\'ido da documenta\c{c}\~ao do Android\footnote{https://developer.Android.com/guide/topics/ui/themes.html}):

\begin{lstlisting}[
	language=XML, 
	caption={\texttt{TextView} usado em mais de um layout e com estilo definido por atributos.}, 
	label={lst:AndroidManifest},
]
<TextView
  android:layout-width="match-parent"
  android:layout-height="wrap-content"
  android:textColor="#00FF00"
  android:typeface="monospace"
  android:text="@string/hello" />
\end{lstlisting}

Visto que o estilo de cor e formata\c{c}\~ao do texto ser\'a replicado, uma op\c{c}\~ao \'e extrair esse estilo para um style resource e utiliz\'a-lo no layout. A seguir, criamos o estilo \textit{CodeFont} e o aplicamos no \textsc{TextView}.

\begin{lstlisting}[
	language=XML, 
	label={lst:AndroidManifest},
	caption={main\_activity.xml},
]
<!-- ... -->
<TextView
  android:layout-width="match-parent"
  android:layout-height="wrap-content"
  android:textAppearance="@style/CodeFont"
  android:text="@string/hello" />
<!-- ... -->
\end{lstlisting}

Arquivo de estilo com o estilo estilo  criado:

\begin{lstlisting}[
	language=XML, 
	caption={styles.xml}, 
	label={lst:AndroidManifest},
]
<?xml version="1.0" encoding="utf-8"?>
<resources>
  <style name="CodeFont" ...>
    <item name="Android:textColor">
      #00FF00
    </item>
    <item name="Android:typeface">
      monospace
    </item>
  </style>
</resources>
\end{lstlisting}


\subsubsection{Suspicious Extra Knowledge About Behavior (SEKAB)}

\textbf{Ocorre quando} uma view possui um conhecimento maior do que o considerado aceit\'avel sobre os detalhes de seu comportamento. Para este smell identificamos por meio das respostas que tr\^es elementos da view \textbf{podem ser afetados} por ele, s\~ao: \textsc{activity}, \textsc{fragment} e \textsc{adapter}. E a depender de qual view est\'a se tratando, o que \'e considerado aceit\'avel pode mudar. Sendo assim, chegamos a seguinte escala de conhecimento aceit\'avel:

\begin{itemize} 
	\item[$\circ$] n\'ivel 1 - implementa\c{c}\~ao de classe concreta;
	\item[$\circ$] n\'ivel 2 - atrav\'es de \textit{implements} (polimorfismo);
	\item[$\circ$] n\'ivel 3 - atrav\'es de classe an\^onima;
	\item[$\circ$] n\'ivel 4 - atrav\'es de \textit{inner class}.
\end{itemize}

Para \textsc{Adapter}s, apenas o n\'ivel 1 foi considerado como aceit\'avel. Para \textsc{Activity}s e \textsc{Fragment}s, at\'e o n\'ivel 3 \'e considerado aceit\'avel, sendo que houveram indica\c{c}\~oes positivas e nenhuma negativa sobre o uso do n\'ivel 1, houveram indica\c{c}\~oes positivas e negativas sobre o n\'ivel 2 e houveram apenas indica\c{c}\~oes negativas sobre o n\'ivel 3 e 4. 

Como \textbf{solu\c{c}\~ao}, segundo as respostas R666 e R672, \textit{"adapters devem ser o mais est\'upido poss\'ivel"} (tradu\c{c}\~ao livre). Podemos dizer ent\~ao que se for necess\'ario atribuir algum comportamento para alguma view que est\'a sendo populada, isso dever\'a ser feito usando a m\'etodo descrito para o n\'ivel 1. J\'a para \textsc{Activity}s e \textsc{Fragment}s a toler\^ancia de conhecimento aumenta at\'e o n\'ivel 3, sendo 1 e 2 os mais indicados e 3 toler\'avel \'as vezes (aqui talvez existam dois thresholds como n\'umero de linhas do m\'etodo do evento ou mesmo quantidade de classes an\^onimas na classe). O n\'ivel 4 \'e inaceit\'avel para todos os elementos afetados por este smell.

Os \textbf{elementos afetados} s\~ao: \textsc{activity}, \textsc{fragment} e \textsc{adapters}. As \textbf{heur\'isticas} de detec\c{c}\~ao s\~ao:

\begin{itemize} 
	\item[$\circ$] [n\'ivel 4] se existe uma ou mais inner class de listener
	\item[$\circ$] [n\'ivel 3] se existe um certo n\'umero de classes an\^onimas
	\item[$\circ$] [n\'ivel 2 e 3] se LOC dentro do m\'etodo do listener - an\^onimo ou sobreescrito \'e maior que 1
\end{itemize}


\subsubsection{Flex Adapter (FA)}

\textbf{Ocorre quando} um adapter precisa de condicionais para decidir se determinada view deve estar visivel ou n\~ao ou quando um adapter recebe mais de um tipo de objeto para ser colocado na view.

Como \textbf{solu\c{c}\~ao}, R578 e R647 sugerem que, caso os condicionais sejam para decidir sobre a exibi\c{c}\~ao ou n\~ao de views, separa em duas ou mais views. Sobre adaptar mais de um tipo de objeto, a resposta R577 sugere a delega\c{c}\~ao para outros adapters especialistas em lidar com cada objeto.




\subsubsection{Messy String Resources (MStringR)} 

\textbf{Ocorre quando} se tem um \'unico arquivo strings.xml para guardar todas as strings da aplica\c{c}\~ao ou quando os arquivos string encontram-se desorganizados, sem seguir nenhum padr\~ao espec\'ifico. Como \textbf{solu\c{c}\~ao}, P28 e P42 sugerem separar as strings por arquivos que indiquem a qual tela aquelas strings pertencem. P42 tamb\'em sugere ter um arquivo de strings comuns a toda aplica\c{c}\~ao. P41 sugere ainda separar os arquivos de acordo com contextos como: urls, mensagens de erro. P32 sugere usar um \'unico arquivo por\'em separando-o em blocos, onde casa bloco indica uma tela.



\subsubsection{Messy Style Resources (MStyleR)} 

\textbf{Ocorre quando} se tem um \'unico arquivo styles.xml para guardar todos os estilos e temas da aplica\c{c}\~ao ou quando os arquivos styles encontram-se desorganizados, sem seguir nenhum padr\~ao espec\'ifico. Como \textbf{solu\c{c}\~ao}, P7, P42 e P40 sugerem criar um styles para estilos de componentes e outro para temas. P8 j\'a sugere quebrar em v\'arios arquivos estilos e um para o tema.



\subsection{Problemas Pontuais}

\subsection{Ferramentas Utilit\'arias}

\subsection{Conhecimento}

\subsection{Categorias de Baix\'issima Recorr\^encia}


\subsection{Interpreta\c{c}\~ao dos Dados}

As vezes o que era indicado como boa pr\'atica para um elemento era um smell percebido em outro elemento, por exemplo, a R1 diz que \textit{"Sempre que noto ter mais de um [layout] resource usando o mesmo estilo eu tento mov\^e-lo [o estilo] para meu recurso de estilo"} (tradu\c{c}\~ao livre) ao responder sobre boa pr\'atica para o style resource, por\'em esta resposta foi considerada para definir o smell Duplicated Styles Attributes que \'e percebido em recursos de layout ou styles.


% \subsection{Afirma\c{c}\~ao sobre o \textit{front-end} Android}

% Uma opini\~ao que foi un\^anime em muitas respostas foi que de fato, desenvolvedores tratam \textsc{Activitys}, \textsc{Fragments} e \textsc{Adapters} como elementos do front-end Android, conforme constatamos na se\c{c}\~ao 3.1. Isso pode ser observado diversas vezes com respostas por exemplo, P25 indicou como boa pr\'atica na Activity "Nenhuma l\'ogica [de neg\'ocio] aqui" (tradu\c{c}\~ao livre), o P40 afirma sobre m\'a pr\'atica em adapter \'e \textit{"L\'ogica de neg\'ocio em adapters \'e n\~ao-n\~ao"} (tradu\c{c}\~ao livre), ao falarem sobre fragments, muitos indicaram \textit{"O mesmo da Activity"}. Ou seja, primeiramente estas respostas refor\c{c}am nossa defini\c{c}\~ao inicial sobre elementos que compo\^em o front-end Android, e por outro lado, vimos que muitas vezes fragments s\~ao tratados como Activitys, ao se falar de boas e m\'as pr\'aticas de c\'odigo.







%------------------ rascunho do script em R abaixo

install.packages("vioplot")
install.packages("effsize")

library(vioplot)
library(effsize)
 
par(cex.lab=1.3, cex.axis=1.3)

# --------- XML smelly vs limpos por smell
nl_smelly <- c(5,5,4,1,3)
nl_clean <- c(2,0,0,0,0)

mr_smelly <- c(4,4,3,2,2,0,0,2,0,3)
mr_clean <- c(4,4,3,2,2,1,0,0,0)

rnp_smelly <- c(3,2,2,2,1,0,1,1,4)
rnp_clean <- c(4,3,3,3,0,2,0)
vioplot(nl_smelly, nl_clean, mr_smelly, mr_clean, rnp_smelly, rnp_clean,
        names=c("NL Cheiroso", "NL Limpo", "MR Cheiroso", "MR Limpo", "RNP Cheiroso", "RNP Limpo", ), col="gold")


# --------- all XML smelly vs limpos por smell
xml_smelly <- c(5,5,4,1,3)
xml_clean <- c(2,0,0,0,0)
vioplot(xml_smelly, xml_clean, names=c("Cheiroso", "Limpo"), col="gold")


# --------- NLIV Smelly vs NLIV limpo vs Smell Tradicionais
nliv_smelly <- c(4,4,3,2,2,0,0,2,0,3)
nliv_clean <- c(4,4,3,2,2,1,0,0,0)
traditional_smelly <- c()
vioplot(nliv_smelly, nliv_clean, traditional_smelly, names=c("NLIV Cheiroso", "NLIV Limpo", "Cheiros Tradicionais"), col="gold")


# # all smells
# severity_traditional <-  c(5,5,5,5,5,5,5,5,4,4,3,3,3,3,3,3,3,3,2,2,1,1,3,3,4,2,5,5,4,5)
# severity_smelly      <- c(5,5,5,5,4,4,4,4,3,3,2,2,2,2,2,1,1,1,0,0,0,0,2,0,1,1,2,4,3,4,3)
# severity_non_smelly <-  c(4,4,4,3,3,3,3,3,2,2,2,1,0,0,0,0,0,0,0,0,0,0,3,2,0,0,0,0,0,0,0,0)
# 
# # boxplot(severity_traditional, severity_non_smelly, severity_smelly, xlab="")
# labels <- c("Tradicional", "Limpos", "Android")
# axis(1, at=1:length(labels), labels=labels)
# vioplot(severity_non_smelly, severity_smelly, severity_traditional, names=c("Limpo", "Android", "Tradicional"), col="gold")
# 
# wilcox.test(severity_traditional, severity_non_smelly)
# cliff.delta(severity_traditional, severity_non_smelly)
# 
# wilcox.test(severity_traditional, severity_smelly)
# cliff.delta(severity_traditional, severity_smelly)
# 
# wilcox.test(severity_smelly, severity_non_smelly)
# cliff.delta(severity_smelly, severity_non_smelly)


#-------------- android java smelly VS android java N smelly VS java smelly
# severity_java_traditional <-  c(5,5,5,5,5,5,4,4,3,3,3,2,2,1,3,5,5,5)
# severity_androd_java_smelly <- c(5,5,4,1,0,2,4)
# severity_android_java_non_smelly <-  c(3,0,0,0,0,3,0,0,0,0,0)
# 
# # boxplot(severity_traditional, severity_non_smelly, severity_smelly, xlab="")
# labels <- c("Tradicional", "Android (NLIV)", "Limpo")
# axis(1, at=1:length(labels), labels=labels)
# vioplot(severity_java_traditional, severity_androd_java_smelly, severity_android_java_non_smelly, names=c("Tradicional", "Android (NLIV)", "Limpo"), col="gold")
# 
# wilcox.test(severity_java_traditional, severity_androd_java_smelly)
# cliff.delta(severity_java_traditional, severity_androd_java_smelly)
# 
# wilcox.test(severity_java_traditional, severity_android_java_non_smelly)
# cliff.delta(severity_java_traditional, severity_android_java_non_smelly)
# 
# wilcox.test(severity_androd_java_smelly, severity_android_java_non_smelly)
# cliff.delta(severity_androd_java_smelly, severity_android_java_non_smelly)


#-------------- android xml smelly VS android xml N smelly
severity_xml_traditional <- c(5,5,3,3,3,3,3,1,3,4,2,4)
severity_androd_xml_smelly <- c(5,5,4,4,4,3,3,2,2,2,2,2,1,1,0,0,0,2,0,1,1,4,3,3)
severity_android_xml_non_smelly <- c(4,4,4,3,3,3,3,2,2,2,1,0,0,0,0,0,0,2,0,0,0)

# boxplot(severity_traditional, severity_non_smelly, severity_smelly, xlab="")
labels <- c("Tradicional", "Android", "Limpo")
axis(1, at=1:length(labels), labels=labels)
vioplot(severity_xml_traditional, severity_androd_xml_smelly, severity_android_xml_non_smelly, names=c("Tradicional", "Android", "Limpo"), col="gold")

wilcox.test(severity_xml_traditional, severity_androd_xml_smelly)
cliff.delta(severity_xml_traditional, severity_androd_xml_smelly)

wilcox.test(severity_xml_traditional, severity_android_xml_non_smelly)
cliff.delta(severity_xml_traditional, severity_android_xml_non_smelly)

wilcox.test(severity_androd_xml_smelly, severity_android_xml_non_smelly)
cliff.delta(severity_androd_xml_smelly, severity_android_xml_non_smelly)


# per smell
no_logic_in_view <- c(5,5,4,1,0,2,4)
resource_name_pattern <- c(3,2,2,2,1,0,1,1,4)
magic_resource <- c(4,4,3,2,2,0,0,2,0,3)
nested_layout <- c(1,4,5,5,3)

boxplot(no_logic_in_view, resource_name_pattern, magic_resource, nested_layout, xlab="")
labels <- c("NLIV", "RNP", "MR", "NL")
axis(1, at=1:length(labels), labels=labels)
vioplot(no_logic_in_view, resource_name_pattern, magic_resource, nested_layout, names=c("NLIV", "RNP", "MR", "NL"), col="gold")



# per traditional smell
# complex_class <- c(4,0,0,0,0,0)
# long_method <- c(4,0,0,4,5,0,0,2,4)
# long_class <- c(3,3,4,3,4,3,5)
# cbo <- c(5,4,5,0,0)
# long_parameter_list <- c(2,0,3,4,0)
# god_class <- c(5,5,3,3,4,3,4)
#
# vioplot(complex_class, long_method, long_class, cbo, long_parameter_list, god_class, names=c("CC", "LM", "LC", "CBO", "LPL", "GC"), col="gold")

# removing only multiple queries
# severity_traditional <- c(3,5,5,5,1,3,5,2,5,5,4,4,5,3,3,5,2,3,1,3,3,3)
# severity_smelly      <- c(1,4,4,0,0,4,4,3,2,1,1,2,5,3,0,0,5,5,2,2,5,2)
# 
# wilcox.test(severity_traditional, severity_smelly)
# cliff.delta(severity_traditional, severity_smelly)


# --------------------------------------------
# pie plot praticas vs. recorrência
install.packages("plotrix")
library(plotrix)
slices <- c(27, 6, 15, 4) 
lbls <- c("Baixíssima", "Baixa", "Média", "Alta")
pie3D(slices,labels=lbls, theta=pi/4, explode=0, edges=0)

# --------------------------------------------
# participants experience Survey 2

# Survey 2 - Experiência Android
# exp=c(1,2,1,4,3)
# names(exp)=c("2-4 anos", "4-6 anos", "6-8 anos", "8-10 anos", "> 10 anos")
# barplot(exp, expex.names = 1, col="darkolivegreen3", ylim=c(0,6))

# Survey 2 - Experiência Android
# exp_web=c(1,2,6,2)
# names(exp_web)=c("1-2 anos", "2-4 anos", "4-6 anos", "6-8 anos")
# barplot(exp_web, exp_webex.names = 1, col="darkolivegreen3", ylim=c(0,6))



# --------------------------------------------
# participants experience Survey 1

# Survey 1 - Experiência Android 
# my_vector=c(6,19,9,9,2)
# names(my_vector)=c("< 1 ano", "2-3 anos", "4-5 anos", "6-7 anos", "8-9 anos")
# barplot(my_vector, my_vectorex.names = 1, col="darkolivegreen3", ylim=c(0,20))

# Survey 1 - Experiência Software
# my_vector2=c(3, 6, 6, 20, 3, 7)
# names(my_vector2)=c("< 1 ano", "2-3 anos", "4-5 anos", "6-7 anos", "8-9 anos", "> 10 anos")
# barplot(my_vector2, my_vector2ex.names = 1, col="darkolivegreen3", ylim=c(0,20))


#---------- Exp. Desenvolvedores Survey 1 (Stacks Bar)
library(ggplot2)
anos=c(rep("< 1 ano" , 2) , rep("2-3 anos" , 2) , rep("4-5 anos" , 2) , rep("6-7 anos" , 2), rep("8-9 anos" , 2), rep("> 10 anos" , 2) )
condição=rep(c("software" , "android") , 12)
valor=c(3,6,6,19,6,9,20,9,3,2,7,0)
dataOrder=data.frame(anos,condição,valor)
dataOrder
dataOrder$anos = factor(dataOrder$anos, levels=c("< 1 ano", "2-3 anos", "4-5 anos", "6-7 anos", "8-9 anos", "> 10 anos"))
# Stacked
ggplot(dataOrder, aes(fill=condição, y=valor, x=anos)) + geom_bar( stat="identity") + theme(text = element_text(size=14), legend.key = element_blank(), legend.title = element_blank()) + labs(x = NULL, y = NULL)


#---------- Exp. Desenvolvedores Survey 2 (Stacks Bar)
#library(ggplot2)
anos=c(rep("1-2 anos" , 2) , rep("2-4 anos" , 2) , rep("4-6 anos" , 2) , rep("6-8 anos" , 2), rep("8-10 anos" , 2), rep("> 10 anos" , 2) )
condição=rep(c("software" , "android") , 12)
valor=c(0,2,2,5,4,6,1,3,4,0,5,0)
dataOrder=data.frame(anos,condição,valor)
dataOrder
dataOrder$anos = factor(dataOrder$anos, levels=c("1-2 anos", "2-4 anos", "4-6 anos", "6-8 anos", "8-10 anos", "> 10 anos"))
# Stacked
ggplot(dataOrder, aes(fill=condição, y=valor, x=anos)) + geom_bar( stat="identity") + theme(text = element_text(size=14), legend.key = element_blank(), legend.title = element_blank()) + labs(x = NULL, y = NULL)#+ guides(fill=FALSE)

%------------------ fim rascunho do script em R abaixo

